\newglossaryentry{FiFo}{
  name={FIFO (First-In, First-Out)},
  description={Das FIFO-Prinzip (engl. First-In, First-Out) besagt, dass das Element, das zuerst zu einer Struktur hinzugefügt wurde, auch als erstes wieder entnommen wird. Man kann es sich wie eine Warteschlange vorstellen: Wer zuerst kommt, mahlt zuerst.}
}

\newglossaryentry{LIFO}{
  name={LIFO (Last-In, First-Out)},
  description={Das LIFO-Prinzip (engl. Last-In, First-Out) besagt, dass das Element, das zuletzt zu einer Struktur hinzugefügt wurde, als erstes wieder entnommen wird. Dies ist typisch für einen Stapelspeicher (Stack).}
}

\newglossaryentry{Stapelspeicher}{
  name={Stapelspeicher (Stack)},
  description={Ein Stapelspeicher, auch Stack genannt, ist eine abstrakte Datenstruktur, die nach dem LIFO-Prinzip (Last-In, First-Out) arbeitet. Elemente werden oben auf den Stapel gelegt (push) und auch von oben wieder entfernt (pop).}
}

\newglossaryentry{Deque}{
  name={Deque (Double-Ended Queue)},
  description={Eine Deque (ausgesprochen \textit{Deck}, kurz für Double-Ended Queue) ist eine Datenstruktur, die das Einfügen und Entfernen von Elementen an beiden Enden (Anfang und Ende) effizient erlaubt. Sie kann somit sowohl als Warteschlange (Queue, FIFO) als auch als Stapelspeicher (Stack, LIFO) verwendet werden.}
}

\newglossaryentry{vollstandigQualifiziert}{
  name={Vollständig Qualifiziert},
  description={Pakete und Klassen, welche über ihren vollen Namen angesprochen werden, nennt man 
    \textit{vollständig Qualifiziert}. So wird also anstelle von \lstinline{Date} 
    \lstinline{java.util.Date} verwendet. Klassen müssen vollständig Qualifiziert werden,
    wenn mehrere Klassen gleich heißen. Beispielsweise \lstinline{java.util.Date} und 
    \lstinline{java.sql.Date}.}
}