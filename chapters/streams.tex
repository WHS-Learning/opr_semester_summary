\chapter{Streams}

Ein Stream ist eine Folge von Elementen, auf die sequentielle und parallele
Operationen angewendet werden können.

Streams unterstützen eine Vielzahl an Operationen wie Filtern, Transformieren,
Aggregieren und vieles mehr. Zudem lassen sich Streams parallel verarbeiten,
wodurch die Laufzeit im Vergleich zu normalen Schleifen auf Mehrkern-CPUs
deutlich schneller ist.

\section{Beispiel}

\begin{lstlisting}[language=Java, caption={Beispiel für Streams}]
Stream.of("08/15", "4771", "501", "s04", "1250", "333", "475")
    .filter(s -> s.matches("[0-9]+")) // nur Strings welche ausschließlich Ziffern enthalten
    .map(s -> Integer.parseInt(s)) // Strings als Integer parsen
    .filter(n -> n >= 400) // nur Zahlen größer gleich 400
    .sorted() // Zahlen sortieren
    .forEach(System.out::println); // Zahlen ausgeben
\end{lstlisting}

Die Ausgabe des Streams ist dann `475, 501, 1250, 4771`.

\section{Arten von Streams}

Es gibt vier verschiedene Arten von Streams:
\begin{itemize}
    \item \texttt{Stream<T>}
    \item \texttt{IntStream}
    \item \texttt{DoubleStream}
    \item \texttt{LongStream}
\end{itemize}

Der Stream \lstinline{Stream<Integer>} ist \textbf{nicht} äquivalent zu dem
\lstinline{IntStream}. Der \lstinline{IntStream} besitzt einige Methoden wie
beispielsweise \lstinline{sum}, welche der \lstinline{Stream<Integer>} nicht
besitzen kann.

\subsection{IntStream Beispiel}

\begin{lstlisting}[language=Java, caption={Beispiel eines IntStreams}]
IntStream.rangeClosed(111, 999)
    .filter(n -> n % (n / 100 + n / 10 % 10 + n % 10) == 0)
    .forEach(System.out::println);
\end{lstlisting}

Hier wird ein \lstinline{IntStream} mit dem geschlossenen Intervall \((111,
999)\) erstellt, bei welchem dann alle Zahlen gefiltert werden, die durch die
Summe ihrer Ziffern teilbar sind. Diese werden anschließend in der Konsole
ausgegeben.

\section{Hinweis}

Ein Stream ist \textbf{keine} Datenstruktur. Sie verwalten keine Daten. Wenn
ein Stream basierend auf einer Collection erstellt wird und diese Collection
anschließend bearbeitet wird, dann verändert sich der Stream automatisch mit.

\begin{lstlisting}[language=Java, caption={Stream über Collection}]
ArrayList<Integer> list = new ArrayList<>();
list.add(10);
list.add(20);
list.add(30);
Stream s = list.stream();
list.remove(0);
s.forEach(System.out::println);
\end{lstlisting}
Es wird `20, 30` in der Konsole ausgegeben, da der Stream die Änderungen in der
zugrunde liegenden Collection widerspiegelt.

\section{Stream erzeugen (Häufige Prüfungsaufgabe)}

Ein Stream kann über mehrere Wege erzeugt werden:

\begin{itemize}
    \item Über die statische Methode in der Schnittstelle \lstinline{Stream<T>}:
          \lstinline{Stream<T> of(T... values)}
    \item Über die Instanzmethode der Schnittstelle \lstinline{Collection<E>}:
          \lstinline{Stream<E> stream()} \(\rightarrow\) \lstinline{ArrayList<String> list.stream()}
    \item Über die statische Methode in der Klasse \lstinline{Arrays}:
          \lstinline{Arrays.stream(new String[])}
    \item Über die Instanzmethode der Klasse \lstinline{BufferedReader}:
          \lstinline{BufferedReader br.lines()}
\end{itemize}

\subsection{Iterate}
\label{iterate}

Es kann eine Funktion verwendet werden, um einen unendlichen Stream zu
erstellen:

\begin{lstlisting}[language=Java, caption={Unendlicher Stream mit Iterate}]
Stream<Integer> s = Stream.iterate(2, n -> n + 2).takeWhile(n -> n > 0);
\end{lstlisting}

Somit wurde ein Stream erstellt, welcher alle positiven geraden Zahlen
beinhaltet. Das
\lstinline{takeWhile(n -> n > 0)} sorgt dafür, dass nur positive Zahlen ausgegeben werden und der Stream terminiert, sobald \texttt{n}
durch einen Integer-Overflow negativ wird.

\subsubsection{im IntStream}

\begin{lstlisting}[language=Java, caption={Unendlicher Stream mit Iterate im IntStream}]
IntStream is = IntStream.iterate(2, n -> n + 2).takeWhile(n -> n > 0);
\end{lstlisting}

\subsection{im LongStream}

\begin{lstlisting}[language=Java, caption={Unendlicher Stream mit Iterate im LongStream}]
LongStream ls = LongStream.iterate(2, n -> n + 2).takeWhile(n -> n > 0);
\end{lstlisting}

\subsection{Generate}
\label{generate}

Ein Stream kann mit einer \lstinline{generate}-Methode gekoppelt werden, welche
aufgerufen wird, um ein neues Element anzufragen:

\begin{lstlisting}[language=Java, caption={Unendlicher Stream mit Generate}]
// Beispiel für eine Methode in einer Klasse, die ein Integer zurückgibt
private int n = 0; // Das Feld muss im Kontext der Klasse definiert sein

public Integer nextInt() {
    return n++; // n zurückgeben und danach erhöhen
}

// Aufruf im Kontext der Klasse (z.B. in einer anderen Methode)
Stream<Integer> s = Stream.generate(this::nextInt).takeWhile(n -> n >= 0);
IntStream is = IntStream.generate(this::nextInt).takeWhile(n -> n >= 0);
LongStream ls = LongStream.generate(this::nextInt).takeWhile(n -> n >= 0);
\end{lstlisting}

Es wird ein Stream erzeugt, welcher über alle positiven \texttt{int}-Werte
läuft. Das \lstinline{takeWhile(n -> n > 0)} sorgt dafür, dass nur positive
Zahlen im Stream landen und der Stream bei einem möglichen Integer-Overflow
terminiert.

\subsection{IntStream.range}

Ein \lstinline{IntStream} und \lstinline{LongStream} kann auch über ein
Intervall erstellt werden. Hier gibt es die statischen Methoden
\lstinline{range(int startInclusive, int endExclusive)} und
\lstinline{rangeClosed(int startInclusive, int endInclusive)}. Die erstellen
ein Intervall von \texttt{start} bis \texttt{end}. Der Unterschied zwischen den
beiden Methoden ist, dass bei der \lstinline{range} Methode das Ende nicht mit
eingeschlossen ist und bei der \lstinline{rangeClosed} das ende mit
eingeschlossen ist.

\section{Operationen auf Streams}

Auf streams können eine vielzahl an Operationen durchgeführt werden. Die
wichtigsten sind:

\begin{itemize}
    \item \lstinline{limit(int limit)} Geht mit maximal so vielen Elementen
          über den Stream. Hat der Stream weniger Elemente als das limit dann
          hört er auf, bevor das limit erreicht wurde.
    \item \lstinline{filter(Predicate<? super T> predicate)} Filtert den Stream nach bestimmten
          kriterien. Beispiel: \lstinline{Stream.filter(i -> i > 3)}
    \item \lstinline{map(Function<? super T, extends R> mapper)} Mappt die Elemente des Streams.
          Beispiel: \lstinline{Stream.map(s -> Integer.parseInt(s))}
    \item \lstinline{mapToInt} Ist äquivalent zu \lstinline{Stream.map(s -> Integer.parseInt(s))},
          nur dass ein \lstinline{IntStream} und kein \lstinline{Stream<Integer>} zurückgegeben wird.
          Es gibt äquivalente Methoden \lstinline{mapToLong} und \lstinline{mapToDouble}
    \item \lstinline{min(Comperator<? super T> comperator)} und \lstinline{max(Comperator<? super T> comperator)}
          Gibt den Maximalen bzw Minimalen Wert zugrunde des Comperators. Comperator funktioniert
          wie das \lstinline{compareto}
    \item \lstinline{findFirst()} Gibt das Erste Element das gefunden aus. Hier wird ein
          \gls{optional} zurückgegeben.
    \item \lstinline{findAny()} Gibt irgend ein Element des Streams als \gls{optional} zurück.
    \item \lstinline{anyMatches(Predicate<? super T> predicate)} Gibt als Boolean aus, ob
          irgendein Element des Streams eine bedingung erfüllt. Beispiel:
          \lstinline{Stream.anyMatches(s -> s.length() < 3)}
    \item \lstinline{allMatches(Predicate<? super T> predicate)} äquivalent zu \lstinline{anyMatches},
          nur dass geprüft wird, ob Alle elemente des Streams diese bedingung erfüllen.
    \item \lstinline{noneMatch(Predicate<? super T> predicate)} äquivalent zu \lstinline{anyMatches},
          nur dass geprüft wird, ob kein Element der Liste diese bedingung erfüllt.
        \item \lstinline{sorted(<Comparator<? super T> comparator)} Sortiert den Stream entweder
        aufgrund des compareTo (wenn kein Parameter übergeben wird) oder anhand eines Comperator.
        Beispiel: \lstinline{Stream.sort((s1, s2) -> s1.toLowerCase().compareTo(s2.toLowerCase))}
\end{itemize}