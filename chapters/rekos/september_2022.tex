\chapter{Reko September 2022}

\section{Aufgabe 1 (18 Punkte)}

Klasse Zeichenobjekt soll Oberklasse von Klassen sein, deren Instanzen
geometrische Objekte einer zweidimensionalen Zeichenanwendung sind (Rechtecke,
Kreise..).

Realisieren Sie eine Klasse Zeichenobjekt welche folgende Forderungen erfüllt:

\begin{enumerate}
    \item Es sind keine Objekte von Klasse Zeichenobjekt erzeugbar
    \item Klasse Zeichenobjekt enthält keine Instanzvariablen
    \item Jedes Objekt von Zeichenobjekt kann durch eine Template-Methode „public String
          gibText()“ eine textuelle Information über sich liefern wie „F=<Fläche>,
          U=<Umfang>“
\end{enumerate}

Gegeben:
\begin{lstlisting}
________________________{ //Klassenkopf ausfüllen
    public final String gibText(){
        …..
    }
}
\end{lstlisting}

\section{Aufgabe 2 (8 Punkte)}

Gegeben:

\begin{lstlisting}
public class A {
    public int m(int n){
        return 5 * n;
    }
}
\end{lstlisting}

Eine Anwendung der Klasse A sehen Sie im folgenden Code-Stück. Die Ausgabe an
Stelle (1) ist 50:

\begin{lstlisting}
A a;
A = new A();
System.out.println(a.m(10)); (1)

a =
System.out.println(a.m(10)); (2)
\end{lstlisting}

Ist eine weitere Zuweisung an die Variable a möglich, sodass
\begin{itemize}
    \item an der Stelle (2) 60 ausgegeben wird
    \item und auch, wenn die Zahl 5 in der Klasse A durch eine andere Zahl ersetzt wird,
          die Ausgabe an der Stelle (2) grundsätzlich um 10 größer ist als die Ausgabe an
          der Stelle (1)?
\end{itemize}

Wenn ja, dann Code-Stück ergänzen und weiteren Code schreiben. Klasse A darf
nicht verändert werden.

\section{Aufgabe 3 (12 + 3 Punkte)}

Schreiben Sie eine statische Methode ereignisseNachJahren (Map<String,
Integer>) mit folgenden Anforderungen:

\begin{itemize}
    \item Der Parameter ist eine Zuordnung, in der den Ereignissen der Zeitgeschichte das
          Jahr dieser Ereignisse zugeordnet wird
    \item Methode ereignisseNachJahren soll als Ergebnis eine Zuordnung liefern, in der
          diese Ereignisse nach Jahren gruppiert sind
    \item Den Jahren (als Schlüsse/Key) sollen die Ereignisse der jeweiligen Jahre als
          Wert (Value) zugeordnet sein
    \item Es kann mehrere Ereignisse pro Jahr geben
    \item Ergebnistyp der Methode muss selber definiert werden
    \item Zusatzpunkte, wenn Methode so realisiert wird, dass die Iteration über die
          Ereignisse der einzelnen Jahre in alphabetischer Ordnung erfolgt
\end{itemize}

Gegeben:
\begin{lstlisting}
public static _______ereignisseNachJahren(Map<String, Integer> ereignisse){
}
\end{lstlisting}

\section{Aufgabe 4 (14 Punkte)}

Es ist eine Klasse Rechteck gegeben:

\begin{itemize}
    \item Der Parameter ist eine Zuordnung, in der den Ereignissen der Zeitgeschichte das
          Jahr dieser Ereignisse zugeordnet wird
    \item Methode ereignisseNachJahren soll als Ergebnis eine Zuordnung liefern, in der
          diese Ereignisse nach Jahren gruppiert sind
    \item Den Jahren (als Schlüsse/Key) sollen die Ereignisse der jeweiligen Jahre als
          Wert (Value) zugeordnet sein
    \item Es kann mehrere Ereignisse pro Jahr geben
    \item Ergebnistyp der Methode muss selber definiert werden
    \item Zusatzpunkte, wenn Methode so realisiert wird, dass die Iteration über die
          Ereignisse der einzelnen Jahre in alphabetischer Ordnung erfolgt
\end{itemize}

Gegeben:
\begin{lstlisting}
public static _______ereignisseNachJahren(Map<String, Integer> ereignisse){
}
\end{lstlisting}