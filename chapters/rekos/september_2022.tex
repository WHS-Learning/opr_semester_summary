\chapter{Reko September 2022}

\section{Aufgabe 1 (18 Punkte)}

Klasse Zeichenobjekt soll Oberklasse von Klassen sein, deren Instanzen
geometrische Objekte einer zweidimensionalen Zeichenanwendung sind (Rechtecke,
Kreise..).

Realisieren Sie eine Klasse Zeichenobjekt welche folgende Forderungen erfüllt:

\begin{enumerate}
    \item Es sind keine Objekte von Klasse Zeichenobjekt erzeugbar
    \item Klasse Zeichenobjekt enthält keine Instanzvariablen
    \item Jedes Objekt von Zeichenobjekt kann durch eine Template-Methode „public String
          gibText()“ eine textuelle Information über sich liefern wie „F=<Fläche>,
          U=<Umfang>“
\end{enumerate}

Gegeben:
\begin{lstlisting}
________________________{ //Klassenkopf ausfüllen
    public final String gibText(){
        ...
    }
}
\end{lstlisting}

\textbf{Antwort:}

\begin{lstlisting}
public abstract class Zeichenobject {
    public final String gibText(){
        return "F=" + this.getArea() + ", U=" + this.getCircumference();
    }

    public abstract double getArea();
    public abstract double getCircumference();
}
\end{lstlisting}

\section{Aufgabe 2 (8 Punkte)}

Gegeben:

\begin{lstlisting}
public class A {
    public int m(int n){
        return 5 * n;
    }
}
\end{lstlisting}

Eine Anwendung der Klasse A sehen Sie im folgenden Code-Stück. Die Ausgabe an
Stelle (1) ist 50:

\begin{lstlisting}
A a;
a = new A();
System.out.println(a.m(10)); (1)

a =
System.out.println(a.m(10)); (2)
\end{lstlisting}

Ist eine weitere Zuweisung an die Variable a möglich, sodass
\begin{itemize}
    \item an der Stelle (2) 60 ausgegeben wird
    \item und auch, wenn die Zahl 5 in der Klasse A durch eine andere Zahl ersetzt wird,
          die Ausgabe an der Stelle (2) grundsätzlich um 10 größer ist als die Ausgabe an
          der Stelle (1)?
\end{itemize}

Wenn ja, dann Code-Stück ergänzen und weiteren Code schreiben. Klasse A darf
nicht verändert werden.

\textbf{Antwort:}

\begin{lstlisting}
public class B extends A {
    @Override
    public int m(int n) {
        return super.m(n) + 10;
    }
}

A a;
a = new A();
System.out.println(a.m(10)); (1) // 50

a = new B();
System.out.println(a.m(10)); (2) // 60
\end{lstlisting}

\section{Aufgabe 3 (12 + 3 Punkte)}

Schreiben Sie eine statische Methode ereignisseNachJahren (Map<String,
Integer>) mit folgenden Anforderungen:

\begin{itemize}
    \item Der Parameter ist eine Zuordnung, in der den Ereignissen der Zeitgeschichte das
          Jahr dieser Ereignisse zugeordnet wird
    \item Methode ereignisseNachJahren soll als Ergebnis eine Zuordnung liefern, in der
          diese Ereignisse nach Jahren gruppiert sind
    \item Den Jahren (als Schlüsse/Key) sollen die Ereignisse der jeweiligen Jahre als
          Wert (Value) zugeordnet sein
    \item Es kann mehrere Ereignisse pro Jahr geben
    \item Ergebnistyp der Methode muss selber definiert werden
    \item Zusatzpunkte, wenn Methode so realisiert wird, dass die Iteration über die
          Ereignisse der einzelnen Jahre in alphabetischer Ordnung erfolgt
\end{itemize}

Gegeben:
\begin{lstlisting}
public static ____________ereignisseNachJahren(Map<String, Integer> ereignisse){
    HashMap<Integer, Set<String>> map = new HashMap<Integer, TreeSet<String>>();

    for (String ereignis : ereignisse.keySet()) {
        int year = ereignisse.get(ereignis);

        Set<String> set = map.getOrDefault(year, new TreeSet<String>);
        set.add(ereignis);
        map.put(year, set);
    }

    return map;
}
\end{lstlisting}

\textbf{Antwort:}

\begin{lstlisting}
public static Map<Integer, Set<String>> ereignisseNachJahren(Map<String, Integer> ereignisse){
    HashMap<Integer, Set<String>> map = new HashMap<Integer, Set<String>>();

    for (String ereignis : ereignisse.keySet()) {
        int year = ereignisse.get(ereignis);

        Set<String> set = map.getOrDefault(year, new TreeSet<String>());
        set.add(ereignis);
        map.put(year, set);
    }

    return map;
}
\end{lstlisting}

\section{Aufgabe 4 (14 Punkte)}

Es ist eine Klasse Rechteck gegeben:

\begin{lstlisting}
public class Rechteck {
    private int breite;
    private int laenge;
    public Rechteck(int breite, int laenge){
        this.breite = breite;
        this. Laenge = laenge;
    }
}
\end{lstlisting}

\begin{itemize}
    \item Klasse Rechteck soll so erweitert werden, sodass Hashset<Rechteck> keine 2
          Rechtecke enthält, die in ihren Flächen gleich sind.
    \item Wenn man über Mengen des Typs iteriert, in welcher Reihenfolge werden die
          Rechtecke durchlaufen? (ankreuzen)
          \begin{itemize}
              \item Keine spez. Reihenfolge
              \item In der Reihenfolge, in der die Rechtecke den jeweiligen Mengen hinzugefügt
                    worden sind
              \item Reihenfolge aufsteigender Flächen
              \item Reihenfolge absteigender Flächen
          \end{itemize}
    \item Ist es möglich dafür zu sorgen, dass kein Objekt erzeugt wird, wenn der
          Kontruktor von Rechteck mit einem oder zwei negativen Parametern aufgerufen
          wird? Wenn ja, den Konstruktor dementsprechend erweitern, wenn nein begründen
          warum.
\end{itemize}

\textbf{Antwort:}

\begin{lstlisting}
public class Rechteck {
    private int breite;
    private int laenge;
    public Rechteck(int breite, int laenge){
        if (breite < 0 || laenge < 0) {
            throw new IllegalArgumentException("Breite und Länge muss mindestens 0 sein.");
        }
        this.breite = breite;
        this. Laenge = laenge;
    }

    @Override
    public boolean equals(Object obj) {
        if (obj instanceof Rechteck r) {
            return r.breite == this.breite && r.laenge == this.laenge;
        }
        return false;
    }

    @Override
    public int hashCode() {
        return this.laenge + this.breite;
    }
}
\end{lstlisting}

Die Rechtecke sind in keiner Spezifischen Reihenfolge sortiert.

\section{Aufgabe 5 (18 Punkte)}

Aufgabe ist leider zu ungenau um sie zu bearbeiten.

(Ungenau) Klasse Textverarbeiter gegeben mit den Methoden verarbeite und lies.
Methode verarbeite hat einen Datenstrom bekommen, welcher dann innerhalb der Methode jeweils
Zeilen mit Wörtern an die zweite Methode weitergereicht hat. Die lies-Methode hat immer nur
zeilenweise Wörter bekommen und innerhalb dieser Methode wurde ein StringTokenizer benutzt um
einzelne Wörter zu bekommen. Die einzelnen Wörter sollten weiter verarbeitet werden, dafür sollte
ein Entwurfsmuster genutzt und eine Schnittstelle Wortverarbeiter mit einer passenden Methode
implementiert werden. Zusätzlich sollte dann eine Klasse geschrieben werden, die diese Schnittstelle
implementiert und die Anzahl der Wörter von der längsten Zeile ausgibt.

\section{Aufgabe 6 (15 Punkte)}

Schreiben Sie Ausdrücke.

\begin{enumerate}
    \item br ist eine Variable vom Typ BufferedReader Aufgabe: Wahrheitswert berechnen,
          ob alle Zeilen der Datenquelle aus mindestens 5 Wörtern bestehen, Worttrenner
          sind nur Leerzeichen. Hinweis: Methode int countTokens() d. Klasse
          StringTokenizer gibt an, wie viele Tokens der StringTokenizer noch liefern
          kann.
    \item buecher ist Variable des Typs List<Buch>. Aufgabe: Das erste Buch ermitteln,
          dessen Preis unterhalb von 10€ liegt. Gehen Sie davon aus, dass es eine Klasse
          Buch \& darin eine Methode float gibPreisInEuro() gibt.
    \item buecher wie in 2. Definiert Aufgabe: Summe der Preise aller Bücher berechnen,
          die mehr als 10€ kosten.
\end{enumerate}

\textbf{Antwort:}

\begin{lstlisting}
br.lines().allMatch(s -> s.split(" ").length >= 5);

buecher.stream().filter(b -> b.gibPreisInEuro() < 10.0f).findFirst()

buecher.stream().filter(b -> b.gibPreisInEuro() > 10.0f).mapToDouble(b -> b.gibPreisInEuro()).sum();
\end{lstlisting}

\section{Aufgabe 7 (10 Punkte)}

Methode boolean hatMindestlaenge(InputStream, long) realisieren, die true
liefert, wenn der übergebene Eingabestrom mindestens die angegebene Länge hat.

Die Methode soll auch für externe Datenquellen effizient arbeiten.

Hinweis: Datenquellen können von unendlicher Länge sein und das Zählen der
Bytes des Eingabestroms ohne Rücksicht auf die übergebene Mindestlänge ist
deshalb kein sinnvoller Ansatz. Sie können das Werfen von Ausnahmen deklarieren
wenn notwendig.

Gegeben:

\begin{lstlisting}
public static boolean hatMindestlaenge(InputStream is, long mindestLaenge){
\end{lstlisting}

\textbf{Antwort:}

\begin{lstlisting}
public static boolean hatMindestlaenge(InputStream is, long mindestLaenge) throws IOException {
    long length = 0;
    BufferedInputStream bis = new BufferedInputStream(is);
    int lastByte = bis.read();

    while (lastByte != -1 || mindestLaenge >= length) {
        length++;
        lastByte = bis.read();
    }

    return length >= mindestLaenge;
}
\end{lstlisting}

\section{Aufgabe 8 (8 Punkte)}

Anfang der Klasse FilteredInputStream gegeben.

\begin{itemize}
    \item Der Stream basiert auf einen InputStream und ermöglicht einen gefilterten
          Zugriff auf Bytes dieses Eingabestroms
    \item Das Filterkriterium soll durch IntPredicate übergeben werden (funktionale
          Schnittstelle der Methode)
    \item Realisieren Sie eine Methode int read()
\end{itemize}

Gegeben:

\begin{lstlisting}
public class FilteredInputStream{
    private final InputStream is;
    private final IntPredicate filter;
    public FilteredInputStream (InputStream is, IntPredicate filter){
        this.is = is;
        this.filter = filter;
    }

    /*Liefert nächsten Bytewert des Eingabestroms für den das
    Filterkriterium erfüllt ist. Wenn Kriterium nicht erfüllt oder kein
    Wert, dann ist die Ausgabe -1.*/
    public int read() throws IOException{
\end{lstlisting}

\textbf{Antwort:}

\begin{lstlisting}
public class FilteredInputStream{
    private final InputStream is;
    private final IntPredicate filter;
    public FilteredInputStream (InputStream is, IntPredicate filter){
        this.is = is;
        this.filter = filter;
    }

    /*Liefert nächsten Bytewert des Eingabestroms für den das
    Filterkriterium erfüllt ist. Wenn Kriterium nicht erfüllt oder kein
    Wert, dann ist die Ausgabe -1.*/
    public int read() throws IOException{
        int lb = is.read();
        while (lb != -1 && !filter(lb)) {
            lb = is.read();
        }
        return lb;
    }
}
\end{lstlisting}

\section{Aufgabe 9 (6 Punkte)}
Testmethode von boolean hatMindestLaenge(InputStream, long) aus Aufg. 7
realisieren.

\begin{itemize}
    \item Testfall realisieren: Anwendung hatMindestlaenge auf einen InputStream soll
          true liefern, wenn der 2. Parameter 6 ist
    \item setUp-Methode wird nicht benötigt
    \item Die statische Methode hat Mindestlaenge darf ohne Qualifizierung aufgerufen
          werden
\end{itemize}

\textbf{Antwort:}

\begin{lstlisting}
@Test
void testHatMindestLaenge1() {
    ByteArrayInputStream bais = new ByteArrayInputStream(new byte[]{1, 2, 3, 4, 5, 6, 7, 8});
    assertTrue(hatMindestLaenge(bais, 6))
}
\end{lstlisting}

\section{Aufgabe 10 (6 Punkte)}

Testklasse StringTokenizerTest realisieren:

\begin{itemize}
    \item StringTokenizer für Zeichenkette „E-Bike“ \& Trennzeichen „-„
    \item Bei 3x Anwendung der Methode nextToken wird erwartet, dass
          NoSuchElementException geworfen wird
\end{itemize}

Gegeben:
\begin{lstlisting}
public class StringTokenizerTest {
\end{lstlisting}

\textbf{Antwort:}

\begin{lstlisting}
public class StringTokenizerTest {
    @Test
    void testNextToken1() {
        StringTokenizer st = new StringTokenizer("E-Bike", "-");
        assertEquals("E", st.nextToken());
        assertEquals("Bike", st.nextToken());
        assertThrows(NoSuchElementException.class, () -> st.nextToken());
    }
}
\end{lstlisting}

\section{Aufgabe 11 (10 Punkte)}

Aufzählungstyp Studiengang mit IN\_BA, WI\_BA, WI\_DUAL, MI\_MA schreiben.

\begin{itemize}
    \item Instanzmethode boolean istBachelorstudiengang() schreiben, die angibt ob es
          sich um einen Bachelorstudiengang handelt (Für alle außer MI\_MA der Fall)
    \item Instanzmethode int regelstudienzeit() schreiben, die die Regelstudienzeit der
          Studiengänge angibt (6 Semester bei Bachelor und 4 Semester bei Master)
    \item Wird Aufzählungstyp um zusätzliche Studiengänge erweitert, funktionieren
          istBachelorstudiengang() und regelstudienzeit() ohne jede Codeänderung auch
          sinnvoll für die neuen Studiengänge?
\end{itemize}

\textbf{Antwort:}

\begin{lstlisting}
public enum Studieng {
    IN_BA(true, 6), WI_BA(true, 6), WI_DUAL(true, 6), MI_MA(false, 4);

    private boolean isBachelor;
    private int normalTime;

    private Module(boolean isBachelor, int normalTime) {
        this.isBachelor = isBachelor;
        this.normalTime = normalTime;
    }

    public istBachelorstudiengang() {
        return this.isBachelor;
    }

    public regelstudienzeit() {
        return this.normalTime;
    }
}
\end{lstlisting}

Für neue Studiengänge funktioniert der Code auch ohne weitere Codeänderung, die
Studiengänge müssen einfach mit den korrekten Parametern eingefügt werden.