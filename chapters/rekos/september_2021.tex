\chapter{Reko September 2021}

\section{Aufgabe 1 (8 Punkte)}

Gegeben sei die folgende Klasse A.

\begin{lstlisting}
public class A {
    public final int m(int a) {
        return a * this.n(a);
    }

    public int n(int a) {
        return 2 * a;
    }
}
\end{lstlisting}

Schreiben Sie zusätzlichen Programmcode, sodass sich das Codestück

\begin{lstlisting}
A a = new B();
System.out.println(a.m(10));
\end{lstlisting}

compilieren lässt und bei Ausführung die Ausgabe 300 erscheint. Die Klasse A
soll unverändert bleiben. Hinweis: durch den Modifikator final in der obigen
Methode m ist diese Methode nicht überschreibbar.

\textbf{Antwort:}

\begin{lstlisting}
public class B extends A {
    @Override
    public int n(int n) {
        return 3 * n;
    }
}
\end{lstlisting}

\section{Aufgabe 2 (10 Punkte)}

Betrachten Sie folgende Klasse:

\begin{lstlisting}
public class Buch extends Katalogartikel {
    private String titel;
    private String autor;
    private int jahr;
    public Buch(String titel, String autor, int jahr) {
        this.titel = titel;
        this.autor = autor;
        this.jahr = jahr;
    }
}
\end{lstlisting}

Beantworten Sie die folgenden beiden Fragen mit ja oder nein und begründen Sie
Ihre Antworten. Ohne Begründungen werden die Antworten nicht gewertet. Die
Antworten müssen insbesondere darauf eingehen, in welcher Beziehung die
Konstruktoren, über die Sie sprechen, zueinander ste- hen.

\begin{enumerate}
    \item Lässt sich die Klasse Buch fehlerfrei compilieren, wenn in der Klasse
          Katalogartikel kein Konstruktor definiert ist?
    \item Ist es möglich, dass in der Klasse Katalogartikel ein Konstruktor definiert ist
          und sich die Klasse Buch nicht fehlerfrei compilieren lässt?
\end{enumerate}

\textbf{Antwort:}

Ja, die Klasse lässt sich Fehlerfrei compilieren, da standardmäßig der
Super-Aufruf ohne Parameter eingefügt wird. Dieser ruft den Konstruktor ohne
Parameter der direkten Oberklasse auf.

Ja, wenn in der direkten Oberklasse kein Konstruktor ohne Parameter (auch nicht
impliztit) definiert ist, lassen die Unterklassen nicht compilieren, wenn sie
keinen Expliziten, korrekt genutzen, Super-Aufruf besitzen.

\section{Aufgabe 3 (15 Punkte)}

Gegeben seien die Klassen Temperatur und Temperaturverwaltung.

\begin{lstlisting}
public class Temperatur {
    private final float t;
    public Temperatur(float t) {
        this.t = t;
    }
    public float wert() {
        return t;
    }
}

public class Temperaturverwaltung {
    private final LinkedList<Temperatur> temperaturen;
    public Temperaturverwaltung() {
        temperaturen = new LinkedList<>();
    }
    public int fuegeHinzu(Temperatur t) {
        Iterator<Temperatur> ts = temperaturen.iterator();
        boolean schonEnthalten = false;
        while (ts.hasNext() && !schonEnthalten) {
            schonEnthalten = ts.next().wert() == t.wert();
        }
        if (!schonEnthalten) {
            temperaturen.add(t);
        }
        return temperaturen.size();
    }
}
\end{lstlisting}

\begin{itemize}
    \item Ändern Sie die Implementierung der Klasse Temperaturverwaltung, sodass sich die Effizienz
          der Methode fuegeHinzu verbessert, insbesondere wenn die Collection der Temperaturen
          (temperaturen) viele Elemente enthält. Sie dürfen dafür auch die Implementierung der
          Klasse Temperatur ergänzen.
          Das äußere Verhalten der Klasse Temperaturverwaltung soll unverändert bleiben, d. h. die
          Werte, die vor der Codeänderung bei Aufrufen von fuegeHinzu zurückgegeben werden, sol-
          len auch nach der Änderung zurückgegeben werden.
    \item Es existieren keine Temperaturen, die unterhalb von -273, 15 Grad Celsius
          liegen (sog. absoluter Nullpunkt). Ändern Sie die Klasse Temperatur derart,
          dass keine Objekte der Klasse für Argumente erzeugt werden können, die kleiner
          als dieser Wert sind.
\end{itemize}

\textbf{Antwort:}

\begin{lstlisting}
public class Temperatur {
    private final float t;
    public Temperatur(float t) {
        if (t < -273.15f) {
            throw new IllegalArgumentException("The temperature is below absolute zero.");
        }
        this.t = t;
    }
    public float wert() {
        return t;
    }

    @Override 
    public boolean equals(Object obj) {
        if (obj instanceof Temperatur t) {
            return t.t == this.t;
        }
    }

    @Override
    public int hashCode() {
        return (int) this.t;
    }
}

public class Temperaturverwaltung {
    private final HashSet<Temperatur> temperaturen;
    public Temperaturverwaltung() {
        temperaturen = new HashSet<>();
    }
    public int fuegeHinzu(Temperatur t) {
        temperaturen.add(t);

        return temperaturen.size();
    }
}
\end{lstlisting}

\section{Aufgabe 4 (6 Punkte)}

Das folgende Codestück lässt sich durch Streichen von Code vereinfachen, ohne
dass sich dadurch seine Funktion ändert. Geben Sie diese Stelle an und
begründen Sie, warum sie entfernt werden kann.

Es geht hier nur um das Vereinfachen durch Entfernen von Code, nicht um andere
Möglichkeiten der Vereinfachung. Das Entfernen von Leerzeichen, Klammern, usw.
ist keine Vereinfachung.

\begin{lstlisting}
public void fuegeHinzu(String schluessel, int zahl) {
    ArrayList<Integer> l = hm.get(schluessel);
    if (l != null) {
        l.add(zahl);
        hm.put(schluessel, l);
    }
}
\end{lstlisting}

Die Variable hm sei hierbei vom Typ TreeMap<String, ArrayList<Integer>>.

\textbf{Antwort:}

\begin{lstlisting}
public void fuegeHinzu(String schluessel, int zahl) {
    ArrayList<Integer> l = hm.get(schluessel);
    if (l != null) {
        l.add(zahl);
    }
}
\end{lstlisting}

Es wird ein Wert immer nur dann eingefügt, wenn der Schlüssel noch nicht in der
TreeMap Existert. Aufgrunddessen kann die putIfAbsent Methode verwendet werden,
um das Key-Value Paar nur dann einzufügen, wenn dieses noch nicht existiert.

\section{Aufgabe 5 (12 Punkte)}

Realisieren Sie eine Methode boolean minLaengeUndZeichen(Reader, short, short),
die ge- nau dann true liefert, wenn die übergebene zeichenorientierte
Datenquelle mindestens l Zeichen (2. Parameter) enthält, und diese l Zeichen
aus mindestens n verschiedenen Zeichen (3. Parameter) bestehen.

Beispiel: Die Datenquelle enthält die Zeichen a5a55ba5ba@bb. Der Aufruf der
Methode für l = 10 und n = 4 liefert false, weil die ersten 10 Zeichen nur aus
3 verschiedenen Zeichen bestehen. Bei l = 11 und n = 4 würde sie true liefern.

Deklarieren Sie das Werfen von Ausnahmen, wenn Sie es für erforderlich halten.

Ein Tipp: Ich habe mit Absicht den Parametertyp short gewählt, weil dadurch
gesichert ist, dass die aktuellen Parameterwerte nicht allzu groß sein können.

public static boolean minLaengeUndZeichen(Reader r, short l, short n) {

\textbf{Antwort:}

\begin{lstlisting}
public static boolean minLaengeUndZeichen(Reader r, short l, short n) throws IOException {
    BufferedReader br = new BufferedReader(r);

    HashSet<Integer> uniqueChars = new HashSet<>();
    int c = 0;
    int lastChar = br.read();

    while (n > l && lastChar != -1 && uniqueSymbols.size() <= n && c <= l) {
        c++;
        uniqueSymbols.add(lastChar);
        lastChar = br.read();
    }

    return uniqueSymbols.size() >= n && c >= l;
}
\end{lstlisting}

\section{Aufgabe 6 (5 Punkte)}

Realisieren Sie eine setUp-Methode, die sinnvoll zum Test der Methode
minLaengeUndZeichen aus der vorherigen Aufgabe verwendet werden kann. Ergänzen
Sie dafür das unten stehende Code- stück an der ...-Stelle. Sie müssen keine
Testmethode realisieren

In der Methode soll ein Reader definiert werden, die folgende Eigenschaften
besitzen:

\begin{itemize}
    \item Die Zeichenfolge des Reader hat mindestens die Länge 10.
    \item Das erwartete Ergebnis bei Aufruf von minLaengeUndZeichen(r, 8, 3) ist false.
    \item Das erwartete Ergebnis bei Aufruf von minLaengeUndZeichen(r, 9, 3) ist true.
\end{itemize}

\begin{lstlisting}
private Reader r;
@Before
public void setUp() {
    r = ...
}
\end{lstlisting}

\textbf{Antwort:}

\begin{lstlisting}
private Reader r;
@Before
public void setUp() {
    r = new StringReader("aaaaaaabcd");
}
\end{lstlisting}

\section{Aufgabe 7 (16 Punkte)}

Gegeben sei der unvollständige Code der Klasse Messwerteleser. Ein Objekt der
Klasse verarbei- tet eine Datenquelle, die Temperatur-Messwerte von
Messstationen enthält. Die Datenquelle wird im Konstruktor übergeben. Das
Einlesen der Messwerte erfolgt in der Methode lies()

\begin{lstlisting}
public class Messwerteleser {
    private Reader r;
    public Messwerteleser(Reader r) {
        this.r = r;
    }
    ... // hier befinden sich weitere Methoden
    public void lies() {
        while (this.gibtEsWeiterenMesswert()) {
            Messwert wert = this.gibNaechstenMesswert();
        }
    }
}
\end{lstlisting}

Objekte der Klasse Messwert repräsentieren einzelne Messwerte bestehend aus Ort
der Messung, Zeitpunkt der Messung und gemessener Temperatur.

\begin{lstlisting}
public class Messwert {
    ...
    public String gibOrt() {...}
    public Datum gibZeitpunkt() {...}
    public float gibTemperatur() {...}
}
\end{lstlisting}

Die Methode lies führt derzeit (abgesehen vom Einlesen) keine Verarbeitung der
Messwerte durch. Dies soll nun geändert werden. Damit die Art der Verarbeitung
flexibel festgelegt werden kann, bietet sich die Anwendung des
Strategie-Entwurfsmusters unter Einsatz einer Schnittstelle an.

\begin{itemize}
    \item Definieren Sie eine geeignete Schnittstelle Messwertverarbeiter. Ein
          Messwertverarbei- ter soll durch Aufruf einer Methode einen Messwert
          verarbeiten können. Den Namen der Methode können Sie selbst wählen.
    \item Ergänzen Sie den vorgegebenen Programmcode der Klasse Messwerteleser, sodass
          ein Objekt dieser Klasse mit jedem beliebigen Messwertverarbeiter-Objekt
          verwendet wer- den kann und ihm die einzelnen Messwerte zur weiteren
          Verarbeitung übergeben kann.
    \item Realisieren Sie eine Klasse HitzeOrt, die diese Schnittstelle implementiert.
          Ein Objekt die- ser Klasse zählt bezogen auf einen einzelnen Ort, wieviele
          Temperaturwerte oberhalb eines bestimmten Schwellwerts liegen

          Im Konstruktor von HitzeOrt wird der Ort und der Schwellwert übergeben. Sie
          benötigen natürlich auch eine Zugriffsmethode für den Zählwert.
    \item Ergänzen Sie das folgende Codestück, sodass aus den Messwerten der Datenquelle
          ermittelt wird, wieviele Temperaturwerte in Bonn oberhalb von 38 Grad liegen
          \begin{lstlisting}
/*
* Reader mit Messwerten; wir nehmen an, die Datenquelle ist endlich.
*/
Reader r = ...;
Messwerteleser leser = new Messwerteleser
leser.lies();
System.out.println("Hitze-Messwerte in Bonn: " +
\end{lstlisting}
\end{itemize}

\textbf{Antwort:}

\begin{lstlisting}
public interface Messwertverarbeiter {
    void compute(Messwert m);
}

public class Messwerteleser{
    private Reader r;
    private messwertverarbeiter mv;

    public Messwerteleser(Reader r, Messwertverarbeiter mv) {
        this.r = r;
        this.mv = mv;
    }
    ... // hier befinden sich weitere Methoden
    public void lies() {
        while (this.gibtEsWeiterenMesswert()) {
            Messwert wert = this.gibNaechstenMesswert();
            mv.compute(wert);
        }
    }
}

public class HitzeOrt implements Messwertverarbeiter {
    private final String ort;
    private final float minTemperature;
    private int temperaturesAboveMin;

    public HitzeOrt(String ort, float minTemperature) {
        this.minTemperature = minTemperature;
        this.temperaturesAboveMin = 0;
    }

    @Override
    public void compute(Messwert m) {
        if (m.gibOrt().equals(ort) && m.gibTemperatur() > minTemperature) {
            temperaturesAboveMin++;
        }
    }

    public int getTemperaturesAboveMin() {
        return this.temperaturesAboveMin;
    }
}

/*
* Reader mit Messwerten; wir nehmen an, die Datenquelle ist endlich.
*/
Reader r = ...;
Messwertverarbeiter mv = new Messwertverarbeiter("Bonn", 38);

Messwerteleser leser = new HitzeOrt(r, mv);
leser.lies();
System.out.println("Hitze-Messwerte in Bonn: " + ((HitzeOrt) mv).getTemperaturesAboveMin());
\end{lstlisting}

\section{Aufgabe 8 (12 Punkte)}

Implementieren Sie einen Aufzählungstyp Einheit, dessen Werte die
Längeneinheiten $\mu$m, mm, cm, dm, m und km repräsentieren. Der Aufzählungstyp
soll folgende Eigenschaften erfüllen

\begin{enumerate}
    \item Der Aufzählungstyp enthält Werte mit den Namen MUM, MM, CM und DM, M und KM
    \item Der Aufzählungstyp besitzt eine Instanzmethode Einheit geeigneteEinheit(long),
          de- ren Verhalten sich am besten an Beispielen erklären lässt.
          \begin{itemize}
              \item Die Länge 12000 cm lässt sich besser als 120 m darstellen, weil die Maßzahl 120
                    kürzer und lesbarer als 12000 ist. Die geeignetere Einheit für diese Länge ist
                    also m. Der Aufruf CM.geeigneteEinheit(12000) soll deshalb M liefern.
              \item Die Länge 2.000.000 mm lässt sich besser als 2 km darstellen. Die geeignetere
                    Einheit für diese Länge ist also km. Der Aufruf
                    MM.geeigneteEinheit(2\_000\_000) soll deshalb KM liefern.
          \end{itemize}
          Gesucht ist also diejenige Einheit, sodass die Länge mit möglichst kleiner, ganzzahliger Maß-
          zahl dargestellt werden kann
\end{enumerate}

Zwei Hilfestellungen für diese Aufgabe:

\begin{enumerate}
    \item Für diese Aufgabe muss man die Umrechnungsfaktoren zwischen den Einheiten
          kennen. Zur Erinnerung: 1 mm = 1000 $\mu$m, 1 dm = 10 cm, 1 m = 10 dm.
    \item Sinnvoll ist es, wenn jede Einheit ihren Umrechnungsfaktor zur nächstgrößeren
          Einheit kennt. Diese Faktoren können Sie in den Objekten des Aufzählungstyps
          verwalten. Wenn Sie bis hierhin kommen (und das richtig machen), erhalten Sie
          schon die Hälfte der Punkte.
\end{enumerate}

\textbf{Antwort:}

\begin{lstlisting}
public enum Einheit {
    MUM(1000), MM(10), CM(10), DM(10), M(1000), KM(-1);

    private final long toNextBigger;

    private Einheit(long toNextBigger) {
        this.toNextBigger = toNextBigger;
    }

    public Einheit geeigneteEinheit(long l) {
        int v = l;
        Einheit e = this;

        while (e.toNextBigger != -1 && v > e.toNextBigger && v % e.toNextBigger == 0) {
            v /= e.toNextBigger;
            e = Einheit.values[e.ordinal() + 1];
        }

        return e;
    }
}
\end{lstlisting}

\section{Aufgabe 9 (8 Punkte)}

Gegeben sei die Funktionale Schnittstelle PINregel

\begin{lstlisting}
@FunctionalInterface
public interface PINregel {
    boolean istZulaessig(String pin);
}
\end{lstlisting}

Die Schnittstelle spezifiziert eine Regel für die Zulässigkeit von PINs. Sie
wird innerhalb einer Online- Banking-Anwendung verwendet, um festzulegen,
welche Anforderungen die PINs der Bankkunden erfüllen müssen. Zur Übergabe der
Regel dient die Methode setzePINregel

\begin{lstlisting}
public class OnlineBanking {
    private PINregel regel;
    ...
    public void setzePINregel(PINregel regel) {
        this.regel = regel;
    }
    ...
}
\end{lstlisting}

\begin{enumerate}
    \item Ergänzen Sie das folgende Codestück derart, dass für das OnlineBanking-Objekt
          die Regel festgelegt wird, dass eine PIN genau dann zulässig ist, wenn sie
          mindestens die Länge 5 hat und das erste und letzte Zeichen ungleich sind.
          Verwenden Sie dafür einen Lambda-Ausdruck OnlineBanking ob = new
          OnlineBanking();
    \item Zu Testzwecken soll nun festgelegt werden, dass jede PIN unzulässig ist.
          Ergänzen Sie das folgende Codestück um eine entsprechende Regel. Verwenden Sie
          dafür einen Lambda-Ausdruck. OnlineBanking ob = new OnlineBanking();
\end{enumerate}

\textbf{Antwort:}

\begin{lstlisting}
OnlineBanking ob = new OnlineBanking.setzePINregel((s) -> s.length() >= 5 && s.charAt(0) != s.charAt(s.length() - 1));

OnlineBanking ob = new OnlineBanking.setzePINregel((s) -> false);
\end{lstlisting}

\section{Aufgabe 10 (8 Punkte)}

Es sei r eine Variable des Typs Reader. Realisieren Sie einen Ausdruck, der
prüft, ob es unter den Zeilen der Datenquelle, die mit @ beginnen, mindestens
eine gibt, die bis auf Leerzeichen am Ende mit "aha" endet (in beliebiger
Groß-/Kleinschreibung).

Beispiel: Enthält die Datenquelle eine Zeile "@Erdbeertörtchen␣123␣aHa␣␣␣",
dann liefert der Aus- druck true.

Verwenden Sie keine Schleifen und keine if-Anweisungen. Hilfreich sind
eventuell diese Methoden:

lines() in Klasse BufferedReader endsWith() und stripTrailing() in Klasse
String

\textbf{Antwort:}

\begin{lstlisting}
r.lines().filter(s -> s.length() >= 1 && s.charAt(0) == '@').anyMatch(s -> s.stripTrailing().toUpperCase().endsWith("AHA"))
\end{lstlisting}