\chapter{Reko July 2021}

\section{Aufgabe 1 (10 Punkte)}

Die Klasse Zeichenobjekt2D soll die Oberklasse von Klassen sein, deren
Instanzen Zeichenobjek- te einer zweidimensionalen Zeichenanwendung sind.
Beispiele für solche Objekte sind Rechtecke, Kreise, Polygone und gruppierte
Zeichenobjekte. Realisieren Sie die Klasse Zeichenobjekt2D so, dass folgende
Forderungen erfüllt sind:

\begin{enumerate}
    \item Man kann keine Objekte der Klasse Zeichenobjekt2D erzeugen.
    \item Die Klasse Zeichenobjekt2D enthält keine Instanzvariable.
    \item Jedes Objekt einer Klasse, die von Zeichenobjekt2D abgeleitet ist, kann durch
          die (Template-) Methode public String gibInfo() eine textuelle Information über
          sich produzieren. Die In- formation ist dreizeilig und hat folgendes Format
          (die konkreten Angaben sind nur Beispiele):
          \begin{lstlisting}
Gruppe
Breite: 100
Höhe: 25
\end{lstlisting}
          Die erste Zeile enthält die Art des Zeichenobjekts, die zweite und dritte seine
          Breite bzw. Höhe.
\end{enumerate}

\textbf{Antwort:}

\begin{lstlisting}
public abstract class Zeichenobjekte2D {
    public String gibInfo() {
        return "Gruppe" + this.getGroup() 
        + "\nBreite: " + this.getBreite() 
        + "\nHoehe: " + this.getHoehe();
    }

    public abstract int getBreite();
    public abstract int getHoehe();
    public abstract String getGroup();
}
\end{lstlisting}

\section{Aufgabe 2 (10 Punkte)}

\begin{enumerate}
    \item Schreiben Sie ein Codestück mit zwei Klassen A und B, wobei B direkte
          Unterklasse von A ist und sich beide Klassen fehlerfrei übersetzen lassen
    \item Ergänzen Sie den Programmcode von A um einen bestimmten Konstruktor, sodass es
          in B zu einem Compilefehler kommt. Die Klasse A soll sich weiterhin fehlerfrei
          übersetzen lassen.
    \item Begründen Sie, warum sich B nach dem Einfügen des Konstruktors in A nicht mehr
          fehlerfrei compilieren lässt.
\end{enumerate}

Notieren Sie die Lösung der Aufgabe wie in dieser Skizze dargestellt, sodass
erkennbar ist, wie der Programmcode nach Schritt (1) aussieht und welcher Code
in Schritt (2) hinzugefügt wird.

\textbf{Antwort:}

\textit{Nach schritt 1:}

\begin{lstlisting}
public class A {

}

public class B extends A {

}
\end{lstlisting}

\textit{Nach schritt 2:}

\begin{lstlisting}
public class A {
    public A(String s) {

    }
}

public class B extends A {

}
\end{lstlisting}

B lässt sich nicht mehr Compilen, da dieser implizit den Super-Aufruf ohne
Parameter aufruft. Die Klasse A hat jedoch keinen Konstruktor ohne Parametern.

\section{Aufgabe 3 (12 Punkte)}

Gegeben sei die Klasse Geldbetrag, deren Objekte Geldbeträge in bestimmten
Währungen reprä- sentieren.

Beispiele: new Geldbetrag("EUR", 12.75) repräsentiert 12,75 Euro. new
Geldbetrag("CHF", 20000) repräsentiert 20.000,- Schweizer Franken.

\begin{lstlisting}
public class Geldbetrag {
    private String waehrung;
    private double betrag;
    public Geldbetrag(String waehrung, double betrag) {
        this.waehrung = waehrung;
        this.betrag = betrag;
    }
}
\end{lstlisting}

Realisieren Sie in der Klasse Geldbetrag eine statische Methode
summiere(Collection<Geldbetrag>), die eine Ansammlung von Geldbeträgen
zusammen- fasst. In der Zusammenfassung ist jeder vorkommenden Währung die
Summe der Beträge in dieser Währung zugeordnet.

Beispiel: Enthält die Ansammlung drei EUR-Beträge 12 EUR, 8 EUR und 30 EUR,
dann enthält die Zusammenfassung einen Eintrag für die Währung EUR mit Wert 50.

Iteriert man in der Zusammenfassung über die Währungen, sollen sie alphabetisch
aufsteigend durchlaufen werden.

Den Ergebnistyp der Methode müssen Sie nach dieser Beschreibung selbst
festlegen.

\textbf{Antwort:}

\begin{lstlisting}
public static Map<String, Double> summe(Collection<Geldbetrag> betraege) {
    TreeMap<String, Double> map = new TreeMap<>();
    for (Geldbetrag g : betraege) {
        int v = map.getOrDefault(g.waehrung, 0);
        map.put(g.waehrung, v + g.betrag);
    }
    return map;
}
\end{lstlisting}

\section{Aufgabe 4 (12 Punkte)}

Verändern und erweitern Sie die Klasse Geldbetrag, sodass folgende
Anforderungen erfüllt sind:

\begin{enumerate}
    \item Es sollen sich Objekte der Klasse genau dann erzeugen lassen, wenn das
          Währungskürzel die Länge 3 hat. \newline Beispiel: ISKR ist kein zulässiges
          Währungskürzel, deshalb soll sich durch new Geldbetrag("ISKR", 100) kein Objekt
          der Klasse erzeugen lassen.
    \item Geldbeträge sollen als Schlüssel einer HashMap verwendbar sein. Geldbeträge
          sind dann gleich, wenn sie den gleichen Betrag in der gleichen Währung haben.
\end{enumerate}

\textbf{Antwort:}

\begin{lstlisting}
public Geldbetrag (String waehrung, double betrag) {
    if (waehrung.length() != 3) {
        throw new IllegalArgumentException("Währung muss 3 Zeichen lang sein");
    }

    this.waehrung = waehrung;
    this.betrag = betrag;
}

@Override
public boolean equals(Object obj) {
    if (obj instanceof Geldbetrag g) {
        return g.waehrung.equals(this.waehrung)
            && g.betrag == this.betrag;
    }
    return false;
}

@Override 
public int hashCode() {
    return this.waehrung.hashCode() + (int) this.betrag;
}
\end{lstlisting}

\section{Aufgabe 5 (10 Punkte)}
Realisieren Sie eine Methode boolean sindGleich(Reader, Reader, short), die
angibt, ob zwei Zeichenfolgen in ihren ersten n Zeichen (3. Parameter)
miteinander übereinstimmen. Ist min- destens eine der Zeichenfolgen kürzer als
n, soll die Übereinstimmung auf dem kürzeren Anfangs- stück überprüft werden.

Deklarieren Sie das Werfen von Ausnahmen, wenn Sie es für erforderlich halten.

\textbf{Antwort:}

\begin{lstlisting}
public static boolean sindGleich(Reader r1, Reader r2, Short s) throws IOException {
    boolean areEqual = true;
    BufferedReader br1 = new BufferedReader(r1);
    BufferedReader br2 = new BufferedReader(r2);

    int lastChar1 = br1.read();
    int lastChar2 = br2.read();

    while (areEqual && lastChar1 != -1 && lastChar2 != -1) {
        areEqual = lastChar1 == lastChar2;
        lastChar1 = br1.read();
        lastChar2 = br2.read();
    }

    return areEqual;
}
\end{lstlisting}

\section{Aufgabe 6 (5 Punkte)}
Realisieren Sie eine setUp-Methode, die sinnvoll zum Test der Methode
sindGleich aus der vor- herigen Aufgabe verwendet werden kann. Ergänzen Sie
dafür das unten stehende Codestück an den ...-Stellen.

In der Methode sollen zwei Reader definiert werden, die folgende Eigenschaften
besitzen:

\begin{itemize}
    \item Die Zeichenfolgen der Reader haben mindestens die Länge 10.
    \item Das erwartete Ergebnis für den Vergleich der ersten 8 Zeichen ist true, für den
          Vergleich der ersten 9 Zeichen false.
\end{itemize}

\begin{lstlisting}
private Reader r1;
private Reader r2;

@Before
public void setUp() {
    r1 = ...
    r2 = ...
}
\end{lstlisting}

\textbf{Antwort:}

\begin{lstlisting}
private Reader r1;
private Reader r2;

@Before
public void setUp() {
    r1 = new StringReader("ABCDEFGHIJ");
    r2 = new StringReader("ABCDEFGHXX");
}
\end{lstlisting}

\section{Aufgabe 7 (8 Punkte)}
Gegeben sei die Klasse XYZ mit der Methode m:

\begin{lstlisting}
public class XYZ {
    public XYZ() {
    
    }
    public List<String> m(String s) {
        ...
    }
}
\end{lstlisting}

Wenn man die Methode m mit Argument "Ich wünsche Ihnen viel Erfolg." auf ein
Ob- jekt der Klasse XYZ anwendet, wird als Ergebnis eine Liste der Länge 4 mit
den Werten "viel", "Erfolg", "Ich" und "Ihnen" erwartet (in dieser
Reihenfolge).

Realisieren Sie in der JUnit-Testklasse XYZTest eine Testmethode, die genau das
überprüft.

\textbf{Antwort:}

\begin{lstlisting}
public class XYZTest {
    private XYZ xyz;

    @BeforeEach
    public void setUp() {
        this.xyz = new XYZ();
    }

    @Test
    testM() {
        assertEquals(List.of("viel", "Erfolg", "Ich", "Ihnen"), xyz.m("Ich Wünsche Ihnen viel Erfolg"));
    }
}
\end{lstlisting}

\section{Aufgabe 8 (12 Punkte)}
Eine Zuweisungsanweisung v = Ausdruck wird vom Compiler nur dann akzeptiert,
wenn der Typ des Ausdrucks mit dem Typ der Variablen verträglich ist.

Geben Sie für jede der folgenden Zuweisungsanweisungen an, ob
Typverträglichkeit vorliegt und begründen Sie Ihre Entscheidung.

Die ersten beiden Zuweisungen dienen als Beispiel.

\begin{enumerate}
    \item OutputStream os = new ByteArrayOutputStream()
    \item long v = 12.5f
    \item Float v = 12.5f
    \item Object v = "Java".substring(3)
    \item BufferedReader v = m() \newline Ergebnistyp der Methode m sei Reader.
    \item HashSet<String> v = new HashMap<String, Integer>().keySet()
    \item List<String> v = new LinkedList<String>()
    \item Comparator<String> v = (s, t) -> s.length() - t.length()
\end{enumerate}

\textbf{Antwort:}

\begin{enumerate}
    \item verträglich: ByteArrayOutputStream ist Unterklasse von OutputStream.
    \item nicht verträglich: der Ausdruck hat Typ float. Der Typ ist breiter als long.
          Breitere Typen werden nicht implizit konvertiert.
    \item verträglich: Der Compiler umhüllt den float zu einem Float.
    \item verträglich: String ist Unterklasse von Object.
    \item Nicht verträglich: Reader ist keine Unterklasse von BufferedReader.
    \item Nicht verträglich: Der Ergebnistyp von keySet() ist ein Set. Set ist keine
          Unterklasse von HashSet.
    \item verträglich: LinkedList ist eine Unterklasse von List.
    \item Comparator ist eine Funktionale Schnittstelle, welche nur die Methode int
          compare(T type1, T type2) besitzt.
\end{enumerate}

\section{Aufgabe 9 (15 Punkte)}
Gegeben sei der unvollständige Code der Klasse Flugportal, die Sie sich als
zentrale Klasse einer Anwendung zur Suche und Buchung von Flügen vorstellen
können. Durch die Methode

List<Flug> sucheDirektfluege(String start, String ziel)

erhält man alle Nonstop-Flugverbindungen, die zwischen einem Start- und
Zielflughafen stattfinden. Nehmen Sie an, dass die Klasse Flug gegeben sind.
Die Details dieser Klasse sind für diese Aufga- be nicht wichtig.

\begin{lstlisting}
public class Flugportal {
    /**
    * Liefert alle Direktflüge vom Start- zum Zielflughafen.
    */
    public List<Flug> sucheDirektfluege(String start, String ziel) {
        List<Flug> verbindungen = ... // in Datenbestand suchen
    }
}
\end{lstlisting}

Das Flugportal soll nun so erweitert werden, dass zu statistischen Zwecken alle
Suchanfragen und ihre Ergebnisse erfasst werden.

Die Methode sucheDirektfluege soll deshalb für jede einzelne Suche den Start-
und Zielflug- hafen sowie die gefundenen Ergebnisse protokollieren. Damit die
Art des Protokolls variieren kann, bietet sich die Anwendung des
Strategie-Entwurfsmusters unter Einsatz einer Schnittstelle an.

\begin{enumerate}
    \item Definieren Sie eine geeignete Schnittstelle Protokollierer. Die Methode der
          Schnittstelle und die Anzahl und Typen ihrer Parameter ergeben sich aus der
          obigen Beschreibung. Den Namen der Methode können Sie selbst wählen.
          (Fortsetzung auf nächster Seite)
    \item Ergänzen Sie den vorgegebenen Programmcode der Klasse Flugportal, sodass ein
          Objekt dieser Klasse zusammen mit jedem beliebigen Protokollierer-Objekt
          verwendet werden kann. Hierzu reicht es, nur den zusätzlichen Code anzugeben.
          Machen Sie geeignet kenntlich, wo er eingefügt werden soll. Die Signatur und
          der Ergebnistyp der Methode sucheDirektfluege sollen unverändert bleiben.
    \item Realisieren Sie eine Klasse Wunschziele, die diese Schnittstelle implementiert.
          Ein Objekt dieser Klasse verwaltet bezogen auf einen einzelnen Startflughafen
          alle Ziele, zu denen die Suchanfrage keine Direktverbindungen finden konnte,
          und liefert diese über die Methode Set<String> gibWunschziele(). Im Konstruktor
          von Wunschziele wird der Startflughafen übergeben, auf den sich diese Pro-
          tokollierung bezieht.
    \item Ergänzen Sie das folgende Codestück, sodass nach den vielen Suchanfragen
          ausgegeben wird, welche Wunschziele es für Flüge ab Münster gibt.
          \begin{lstlisting}
/*
* Objekt der Klasse Wunschziele erzeugen. Hier etwas ergänzen.
*/

/*
* Flugportal erzeugen. Hier etwas ergänzen.
*/

Flugportal portal = new Flugportal

/*
* Es folgen viele Aufrufe der Methode sucheDirektfluege. Hier nichts ergänzen.
*/

portal.sucheDirektfluege(...);
...
portal.sucheDirektfluege(...);

/*
* Wunschziele für Startflughafen Münster ausgeben. Hier etwas ergänzen.
*/
System.out.println("Wunschziele ab Münster: " +);
\end{lstlisting}
\end{enumerate}

\textbf{Antwort:}

\begin{lstlisting}
public interface Protokollierer {
    void log(String start, String end, List<Flug> fluege);
}

public class Flugportal {
    private Protokoller p;

    public Flugportal(Protokoller p) {
        this.p = p;
    }

    /**
    * Liefert alle Direktflüge vom Start- zum Zielflughafen.
    */
    public List<Flug> sucheDirektfluege(String start, String ziel) {
        List<Flug> verbindungen = ... // in Datenbestand suchen
        p.log(start, ziel, verbindungen);    
        return verbindungen;
    }
}

public class Wunschziele implements Protokoller {
    private HashSet<String> destinations;
    private String start;

    public Wunschziele(String start) {
        this.start = start;
        this.destinations = new HashSet<>();
    }

    @Override
    public void Log(String start, String end, List<Flug> fluege) {
        if (this.start.equals(start) && fluege.isEmpty()) {
            destinations.add(end);
        }
    }

    public Set<String> gibWunschziele() {
        return fluege;
    }
}
\end{lstlisting}

\begin{lstlisting}
/*
* Objekt der Klasse Wunschziele erzeugen. Hier etwas ergänzen.
*/

Wunschziele ws = new Wunschziele("Münster");

/*
* Flugportal erzeugen. Hier etwas ergänzen.
*/

Flugportal portal = new Flugportal(ws);

/*
* Es folgen viele Aufrufe der Methode sucheDirektfluege. Hier nichts ergänzen.
*/

portal.sucheDirektfluege(...);
...
portal.sucheDirektfluege(...);

/*
* Wunschziele für Startflughafen Münster ausgeben. Hier etwas ergänzen.
*/
System.out.println("Wunschziele ab Münster: " + String.join(ws.gibWunschziele(), ", "));
\end{lstlisting}

\section{Aufgabe 10 (7 + 3 Punkte)}
Implementieren Sie einen Aufzählungstyp Zeiteinheit. Wenn der Aufzählungstyp
die erste und zweite Eigenschaft korrekt erfüllt, erhalten Sie 7 Punkte. Wenn
er alle drei Eigenschaften korrekt erfüllt, erhalten Sie 10 Punkte.

Eigenschaften:

\begin{enumerate}
    \item Der Aufzählungstyp enthält Werte mit den Namen TAG, STUNDE, MINUTE und SEKUNDE.
    \item Der Aufzählungstyp besitzt eine Instanzmethode int sekunden(), durch die eine
          Zeitein- heit angeben kann, welcher Anzahl Sekunden sie entspricht. Der
          Ausdruck STUNDE.sekunden() soll z. B. den Wert 3600 liefern. Die Methode
          enthält keine Fallunterscheidung.
    \item Der Aufzählungstyp besitzt eine Instanzmethode double in(Zeiteinheit), durch
          die ei- ne Zeiteinheit angeben kann, wie groß sie in der übergebenen Einheit
          ist. Der Ausdruck STUNDE.in(MINUTE) soll z. B. den Wert 60.0 liefern. Der
          Ausdruck MINUTE.in(STUNDE) soll 0.01666667 liefern, denn eine Minute ist 1/60
          einer Stunde. Die Methode enthält keine Fall- unterscheidung.
\end{enumerate}

\textbf{Antwort:}

\begin{lstlisting}
public enum Zeiteinheit {
    TAG(60*60*60*24), STUNDE(3600), MINUTE(60), SEKUNDE(1);

    private final int seconds;
    
    private Zeiteinheit(int seconds) {
        this.seconds = seconds;
    }

    public int sekunden() {
        return this.seconds;
    }

    public double in (Zeiteinheit z) {
        return (double) this.seconds / (double) z.seconds;
    }
}
\end{lstlisting}

\section{Aufgabe 11 (16 Punkte)}
Die Methode Stream<String> lines() der Klasse BufferedReader liefert die Zeilen
einer zeichenorientierten Datenquelle als Stream. Die Variablen r und l, auf
die unten Bezug genommen wird, seien wie folgt deklariert:

\begin{lstlisting}
    BufferedReader r;
    List<Integer> l;
\end{lstlisting}

\begin{enumerate}
    \item Realisieren Sie unter Verwendung von Streams einen Ausdruck, der die erste
          Zeile der Daten- quelle in r liefert, deren Länge größer als 40 ist. Geben Sie
          den Typ des Ausdrucks an.
    \item Realisieren Sie unter Verwendung von Streams einen Ausdruck, der angibt, ob
          alle Zeilen von r eine Länge >= 10 haben.
    \item Realisieren Sie unter Verwendung von Streams ein Codestück, das die Menge aller
          geradzah- ligen Elemente von l bildet.
    \item Realisieren Sie unter Verwendung von LongStream.iterate einen Stream mit den
          Zahlen 0, 1, 3, 7, 15, 31, 63, . . .. Ignorieren Sie, dass es irgendwann einen
          Überlauf gibt.
\end{enumerate}

\textbf{Antwort:}

\begin{lstlisting}
    r.lines().filter(s -> s.length() > 40).findFirst();
\end{lstlisting} \textit{Der Typ ist Optional<String>}
\begin{lstlisting}
r.lines().allMatch(s -> s.length() >= 10);
\end{lstlisting}
\begin{lstlisting}
l.stream().filter(i -> i % 2 == 0).toSet();
\end{lstlisting}
\begin{lstlisting}
LongStream.iterate(0, i -> i + i + 1);
\end{lstlisting}

