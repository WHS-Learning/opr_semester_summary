\chapter{Reko Februar 2021}

\section{Aufgabe 1 (10 Punkte)}

Die Klasse Zeichenobjekt soll die Oberklasse von Klassen sein, deren Instanzen
geometrische Objekte einer zweidimensionalen Zeichenanwendung sind. Beispiele
für solche Objekte sind Recht- ecke, Kreise und Polygone. Realisieren Sie die
Klasse Zeichenobjekt so, dass folgende Forderun- gen erfüllt sind:

\begin{enumerate}
    \item Man kann keine Objekte der Klasse Zeichenobjekt erzeugen.
    \item Die Klasse Zeichenobjekt enthält keine Instanzvariable.
    \item Jedes Objekt einer von Zeichenobjekt abgeleiteten Klasse kann durch die Methode
          \begin{lstlisting}
public boolean istGroesser(Zeichenobjekt)
\end{lstlisting}
          angeben, ob es eine echt größere Fläche als das übergebene Zeichenobjekt
          besitzt.
\end{enumerate}

\textbf{Antwort:}

\begin{lstlisting}
public abstract class Zeichenobject {
    public abstract double area();
    public boolean istGroesser(Zeichenobject z) {
        return this.area() > z.area();
    }
}
\end{lstlisting}

\section{Aufgabe 2 (6 Punkte)}

Gegeben sei die Klasse A. Muss man in einer direkten Unterklasse von A einen
Konstruktor implementieren? Begründen Sie Ihre Antwort. Ohne Begründung wird
die Antwort nicht gewertet.

\begin{lstlisting}
public class A {
    private int a;
    public A(int a) {
        this.a = a;
    }
    protected int quadrat() {
        return a * a;
    }
}
\end{lstlisting}

\textbf{Antwort:}

Ja, weil A keinen Parameterlosen Konstruktor besitzt. Besitzt eine Klasse
keinen Konstruktor, so wird der Konstruktor ohne Parameter mit dem Super-Aufruf
ohne Parameter vom Compiler eingefügt. Da der Konstruktor von Klasse A jedoch
einen Parameter besitzt, führt der Super-Aufruf ohne Parameter in direkten
Unterklassen von A zu einen Compilefehler.

\section{Aufgabe 3 (12 Punkte)}

Realisieren Sie eine statische Methode erzeugeIndex, die zu einem Feld von
Zeichenketten eine Zuordnung erzeugt.

Die Schlüssel der Zuordnung sind die Anfangszeichen der in dem Feld
vorkommenden Zeichenket- ten.

Als Werte sind den Schlüsseln Collections derjenigen Wörter zugeordnet, die mit
dem jeweiligen An- fangsbuchstaben beginnen. Die Collections dürfen kein Wort
doppelt enthalten. Auf die Reihenfolge der Wörter in den Collections kommt es
nicht an.

Sie dürfen davon ausgehen, dass alle Zeichenketten des Felds mindestens die
Länge 1 besitzen.

Vervollständigen Sie auch die Typangabe im Kopf der Methode.

Beispiel: Das Feld enthält die sechs Zeichenketten:

\begin{itemize}
    \item Fahrrad
    \item 123 Abflussreinigung
    \item Fahrstuhl
    \item frische Blumen
    \item Fahrrad
    \item Pfirsich
\end{itemize}

Dann enthält die resultierende Zuordnung folgende Einträge:

\begin{itemize}
    \item F → [Fahrstuhl, Fahrrad]
    \item 1 → [123 Abflussreinigung]
    \item f → [frische Blumen]
    \item P → [Pfirsich]
\end{itemize}

\textbf{Antwort:}

\begin{lstlisting}
public static Map<Character, Collection<String>> erzeugeIndex(String[] woerter) {
    HashMap<Character, Collection<String>> map = new HashMap<>();

    for (String s : woerter) {
        HashSet<String> list = map.getOrDefault(s.charAt(0), new HashSet<String>());

        list.add(s);

        map.put(s.charAt(0), list);
    }

    return map;
}
\end{lstlisting}

\section{Aufgabe 4 (12 Punkte)}

Realisieren Sie eine Klasse Zeitdauer. Die Klasse soll einen Konstruktor
Zeitdauer(int, int) besitzen, um ein Objekt der Klasse durch Angabe einer
Anzahl Stunden (1. Parameter) und einer Anzahl Minuten (2. Parameter) zu
erzeugen.

Die Klasse soll folgende Anforderungen erfüllen:

\begin{enumerate}
    \item Es sollen sich Objekte genau dann erzeugen lassen, wenn die Werte beider
          Parameter größer oder gleich 0 sind.
    \item Ein HashSet<Zeitdauer> enthält niemals zwei Objekte, die die gleiche Zeitdauer
          repräsen- tieren. Beispiel: 1 Stunde und 100 Minuten sind die gleiche Zeitdauer
          wie 2 Stunden und 40 Minuten.
\end{enumerate}

\textbf{Antwort:}

\begin{lstlisting}
public class Zeitdauer {
    private final int h;
    private final int m;

    public Zeitdauer(int h, int m) {
        if (h < 0 || m < 0) {
            throw new IllegalArgumentException("Hours and Minutes has to be at least 0");
        }

        this.h = h;
        this.m = m;
    }

    @Override 
    public boolean equals(Object obj) {
        if (obj instanceof Zeitdauer z) {
            return z.h * 60 + z.m == this.h * 60 + this.m;
        }
        return false;
    }

    @Override 
    public int hashCode() {
        return this.h + this.m;
    }
}
\end{lstlisting}

\section{Aufgabe 5 (6 Punkte)}

Die Methode hashCode einer Klasse K ist so implementiert, dass sie für alle
Objekte dieser Klasse denselben Hash-Code liefert. Entscheiden Sie sich für
eine der folgenden Antworten und begründen Sie sie.

Wenn Sie sich für die erste Antwort entscheiden, müssen Sie klar begründen, was
in diesen Fall nicht funktioniert oder welcher Fehler auftritt.

Wenn Sie sich für die zweite Antwort entscheiden, müssen Sie klar begründen,
worin sich der Nach- teil dieser Implementierung zeigt.

\textbf{Antwort:}

Die Implementierung ist zulässig, aber ungünsitg.

Der Hash-Code zweier Objekte muss immer gleich sein, wenn diese Objekte auch
inhaltlich gleich sind. Dies ist hier der Fall, da immer der selbe Hash-Code
gegeben ist.

Gleichzeitig ist es Empfehlenswert hashCode() möglichst so zu implementieren,
das bei inhaltlicher Ungleichheit der Objekte der hashCode() wahrscheinlich
auch ungleich ist. Dies ist hier nie der Fall, da immer der selbe Hash-Code
gegeben ist. Z.B. werden dadurch Collection-Klassen, wie HashMap und HashSet
beinträchtigt. Diese nutzen Hash-Codes, um Objekte schnell zu finden. Ist
hashCode immer gleich, kann dies nicht geschehen, da sich auf Unterschiede im
hashCode verlassen wird.

\section{Aufgabe 6 (6 Punkte)}

Realisieren Sie eine Methode boolean enthaelt(InputStream, Set<Byte>), die
angibt, ob der übergebene Eingabestrom alle byte-Werte aus der übergebenen
Menge enthält. Deklarieren Sie das Werfen von Ausnahmen, wenn Sie es für
erforderlich halten.

Hinweis: Ist der übergebene Eingabestrom unendlich lang, kann die Methode
entweder true liefern (falls alle byte-Werte im Eingabestrom vorkommen) oder
endlos laufen. Für einen unendlich langen Eingabestrom ist die Antwort false
nicht möglich.

Hinweis: Es ist zulässig, die übergebene Menge zu verändern, z. B. indem Sie
daraus im Laufe der Verarbeitung byte-Werte entfernen.

\textbf{Antwort:}

\begin{lstlisting}
public static boolean enthaelt(InputStream is, Set<Byte> bytes) throws IOException {
    BufferedInputStream bis = new BufferedInputStream(is);    
    int b = bis.read();

    while (bytes.isEmpty() && b != -1) {
        if (bytes.contains((byte) b)) {
            bytes.remove((byte) b);
        }

        b = bis.read();
    }

    return bytes.isEmpty();
}
\end{lstlisting}

\section{Aufgabe 7 (10 Punkte)}

Gegeben sei folgende Klasse:

\begin{lstlisting}
/**
* Ein ReaderTokenizer zerlegt den Inhalt einer endlichen, zeichen-
* orientierten Datenquelle in einzelne Wörter. Die Zerlegung erfolgt
* anhand der Trennzeichen, die im Konstruktor dieser Klasse übergeben werden.
*/
public class ReaderTokenizer {
    ...
    /**
    * Erzeugt ein Objekt dieser Klasse für den übergebenen Reader und
    * die übergebenen Trennzeichen.
    * @param r Reader, dessen Zeichenstrom in Wörter zerlegt wird.
    * Es wird vorausgesetzt, dass der Zeichenstrom endlich ist.
    * @param trennzeichen Jedes Zeichen dieser Zeichenkette ist ein
    * Trenner (analog zu Trennzeichen bei StringTokenizer).
    */
    public ReaderTokenizer(Reader r, String trennzeichen) {
        ...
    }

    /**
    * Liefert die Wörter des Zeichenstroms.
    */
    public Set<String> gibWoerter() {
        ...
    }
}
\end{lstlisting}

Realisieren Sie eine Testklasse ReaderTokenizerTest mit zwei Testfällen für die
Methode gibWoerter. Für beide Testfälle soll der Reader den Zeichenstrom Viel
Erfolg bei OPR.\ enthalten.

Im ersten Testfall sind die Trenner `. ' (Punkt und Leerzeichen). Erwartet
werden die vier Wörter des Textes.

Im zweiten Testfall sind die Trenner `.,!'. Das Leerzeichen ist somit kein
Trenner.

\textbf{Antwort:}

\begin{lstlisting}
public class ReaderTokenizerTest {
    private StringReader sr;

    @BeforeEach
    void setUp() {
        this.sr = new StringReader("Viel Erfolg bei OPR.");
    }

    @AfterEach
    void tearDown() {
        this.sr.close();
    }

    @Test
    void testGibWoerter1() {
        ReaderTokenizer rt = new ReaderTokenizer(sr, ". ");
        assertEquals(Set.of("Viel", "Erfolg", "bei", "OPR"), rt.gibWoerter());
    }

    @Test
    void testGibWoerter2() {
        ReaderTokenizer rt = new ReaderTokenizer(sr, ".,!");
        assertEquals(Set.of("Viel Erfolg bei OPR", rt.gibWoerter()));
    }
}
\end{lstlisting}

\section{Aufgabe 8 (6 Punkte)}

Bei Auswertung des Ausdrucks "Haus".charAt(4) kommt es zu einer
StringIndexOutOfBoundsException. Die Meldung dieser Ausnahme ist String index
out of range: 4.

Realisieren Sie einen JUnit-Testfall (d. h. eine Testmethode), der genau das
überprüft. Der Testfall soll erfolgreich sein, wenn bei Auswertung des
Ausdrucks die Ausnahme mit der angegebenen Meldung auftritt. Der Testfall soll
scheitern, wenn dies nicht der Fall ist. Sie können sich in dieser Aufgabe rein
auf die Testmethode konzentrieren. Das public class ... drum herum denken wir
uns.

\textbf{Antwort:}

\begin{lstlisting}
@Test
void testCharAt() {
    StringIndexOutOfBoundsException e = assertThrows(StringIndexOutOfBoundsException.class, () -> "Haus".charAt(4));
    assertEquals("String index out of Range 4", e.getMessage());
}
\end{lstlisting}

\section{Aufgabe 9 (10 + 2 Punkte)}

Ergänzen Sie die Klasse Rechteck, sodass durch

Collections.sort(rechtecke)

eine Liste

List<Rechteck> rechtecke;

wie folgt sortiert wird:

\begin{itemize}
    \item Rechtecke größerer Höhe stehen vor Rechtecken mit kleinerer Höhe.
    \item Bei Rechtecken mit gleicher Höhe stehen die mit kleinerer Breite vor denen mit
          größerer Breite.
\end{itemize}

Beispiel: $(b \times h)$ bezeichnet ein Rechteck der Breite b und Höhe h. Die
Rechtecke würden so sortiert: $(1 \times 8), (3 \times 8), (2 \times 6), (4
    \times 5), (6 \times 5)$

Sie erhalten zwei Zusatzpunkte, wenn auch folgende Anforderung erfüllt ist:
Erzeugt man ein TreeSet<Rechtec durch new TreeSet<Rechteck>(), dann wird es
niemals zwei Rechtecke mit gleicher Breite und Höhe enthalten.

\begin{lstlisting}
public class Rechteck                            {
    private int breite;
    private int hoehe;

    public Rechteck(int breite, int hoehe) {
        this.breite = breite;
        this.hoehe = hoehe;
    }
}
\end{lstlisting}

\textbf{Antwort:}

\begin{lstlisting}
public class Rechteck implements Comparable<Rechteck> {
    private int breite;
    private int hoehe;

    public Rechteck(int breite, int hoehe) {
        this.breite = breite;
        this.hoehe = hoehe;
    }

    @Override
    public int compareTo(Rechteck r) {
        return return r.hoehe == hoehe 
                ? breite - r.breite
                : r.hoehe - hoehe;
    }
}
\end{lstlisting}

\section{Aufgabe 10 (12 Punkte)}

Betrachten Sie folgende Klasse mit der statischen Methode

\begin{lstlisting}
List<String> filter(List<String>, char).

public class Listenfilter {
    public static List<String> filter(List<String> elemente, char anfang) {
        ArrayList<String> gefilterteElemente = new ArrayList<>();
        for (String element : elemente) {
            if (!element.isEmpty() && element.charAt(0) == anfang) {
                gefilterteElemente.add(element);
            }
        }
        return gefilterteElemente;
    }
}
\end{lstlisting}

Wendet man die Methode auf eine Liste von Zeichenketten und ein Zeichen an,
liefert die Methode eine Liste mit allen Elementen aus der übergebenen Liste,
die mit dem angegebenen Zeichen begin- nen. Die Filtermethode lässt also alle
Werte durch, die mit dem angegebenen Zeichen beginnen.

Möchte man Listen nach anderen Kriterien filtern, z. B. `lasse alle
Zeichenketten mit ungerader Länge durch' oder `lasse alle Zeichenketten durch,
die nur aus Großbuchstaben bestehen', könnte man zusätzliche Filter-Methoden
definieren.

Besser ist es jedoch, genau eine Filtermethode zu definieren, der man neben der
Liste auch das `Durchlass-Kriterium' als Parameter übergibt. Hierfür eignet
sich eine Schnittstelle.

\begin{itemize}
    \item Definieren Sie eine Schnittstelle Filterkriterium mit einer geeigneten Methode.
          Ein Filterkriterium soll entscheiden, ob ein Wert durchkommt oder nicht.
    \item Definieren Sie eine statische Methode List<String> filter(List<String>,
          Filterkriterium), die eine Liste aller Ele- mente aus der übergebenen Liste
          liefert, die vom Filterkriterium durchgelassen werden.
    \item Realisieren Sie eine Klasse, die Filterkriterium implementiert und die alle
          Zeichenketten durchlässt, deren Länge echt größer als 10 ist.
\end{itemize}

\textbf{Antwort:}

\begin{lstlisting}
public interface Filterkriterium {
    boolean filter (String s);
}

public static List<String> filter(List<String> strings, Filterkriterium f) {
    return strings.stream().filter(f::filter).toList();
}

public class LengthFilter implements Filterkriterium {
    @Override
    public boolean filter(String s) {
        return s.length() < 10;
    }
}
\end{lstlisting}