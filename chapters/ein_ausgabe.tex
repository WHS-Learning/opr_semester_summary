\chapter{Ein- und Ausgabe}

\section{InputStream}
\label{inputstream}

Der \lstinline{InputStream} ist die Oberklasse und Abstraktion alelr
byteorientierten Eingabeströme. Es gibt nur zwei Methoden zum Lesen von Bytes:
\begin{itemize}
    \item \lstinline{int read()} liest \textit{ein} Byte
    \item \lstinline{int read(byte[])} liest mehrere Bytes, Ergebniswert hat eine andere
          Bedeutung als der von \lstinline{read()}
\end{itemize}

Der Rückgabewert der \lstinline{read()} Methode gibt den gelesenen Byte zurück.
Dieser kann einen Wert zwischen 0 und 255 (dem gelesenen Byte) sein. Ein
Rückgabewert von -1 signalisiert, dass das Ende des Streams erreicht wurde und
keine weiteren Bytes verfügbar sind. Es wird also keine Fehlermeldung geworfen,
wenn das Ende der Datei erreicht wurde. Wenn bei dem Lesen allerdings ein
Fehler auftritt, wird eine \lstinline{IOException} geworfen.

\begin{lstlisting}[language=Java, caption={Beispiel für InputStream}]
InputStream is = new ByteArrayInputStream(new byte[]{1, 2, 3, 4, 5});

int b = is.read();
ArrayList<Integer> content = new ArrayList<>();

while (b != -1) {
    content.add(b);
}
\end{lstlisting}

Das einzelne Lesen von einzelnen Bytes aus einen \lstinline{InputStream} ist für externe Datensenken sehr
ineffizient. Dateien können sehr groß werden, schnell auch mehrere Gigabyte
groß. Dies entspricht millionen von \lstinline{read()} aufrufen und damit
millionen von \texttt{I/O}-Zugriffen. Aufgrunddessen ist die einfache
\lstinline{read()} Methode in \lstinline{InputStreams} oft eine schlechte
lösung. Abhilfe schafft hier die Klasse \lstinline{BufferedInputStream}.

\subsection{BufferedInputStream}

Der \lstinline{BufferedInputStream} kapselt einen \lstinline{InputStream} und liest intern aus diesen
Blockweise über die Methode
\lstinline{read(byte[])}. Hier ist die Methode \lstinline{read()} effizient und
kann gut genutzt werden.

\subsection{ByteArrayInputStream}

Dies ist beispielsweise essenziell für das Schreiben von Testklassen, da diese
niemals direkt von externen Dateien abhängen sollten, um in sich geschlossen
und reproduzierbar zu sein.

\section{Reader}

Ähnlich wie der \nameref{inputstream} bietet der Reader zwei Methoden zum Lesen von Zeichen:
\begin{itemize}
    \item \lstinline{int read()} liest \textit{ein} Zeichen
    \item \lstinline{int read(char[])} liest mehrere Zeichen, Ergebniswert hat eine andere
          Bedeutung als der von \lstinline{read()}
\end{itemize}

Auch hier ist die Methode \lstinline{read()} für externe Datenquellen
sehr ineffizient.

\subsection{BufferedReader}

Der \lstinline{BufferedReader} kapselt einen \lstinline{Reader} ein und liest intern auf diesem
blockweise. Die \lstinline{read()}-Methode ist in dieser Klasse durch das
interne Puffern auch für einzelne Zeichenlesevorgänge effizient.

Der \lstinline{BufferedReader} enthält außerdem die Methode \lstinline{String readLine()}
um eine ganze Zeile zu lesen. Dies kann auch der gesamte Inhalt der Datenquelle
sein. Außerdem enthält er die Methode \lstinline{Stream<String> lines()} um
einen Stream aller Zeilen zu erstellen.

\subsection{StringReader}

Die Klasse \lstinline{StringReader} eignet sich sehr gut um Methoden mit \lstinline{Reader}-Parametern
zu testen.

\section{Codierung}

Es gibt viel mehr Zeichen als Bytes. Deshalb müssen einige Zeichen durch
mehrere Bytes repräsentiert werden. Eine Zeichencodierung beschreibt die Art
der Codierung von Zeichen durch Bytes. Die am häufigsten verwendete Codierung
ist UTF-8. Die meisten westeuropäischen Zeichen werden hier durch ein Byte
dargestellt. Alle anderen Zeichen durch 2 bis 4 Byte. In UTF-16 werden alle
Zeichen durch mindestens zwei Byte dargestellt.

Um die Codierung eines InputStreams anzugeben, hat sein Konstruktor einen
zweiten Parameter. Über den Aufzaehlungstypen
\lstinline{StandardCharsets} kann hier dann die erwünschte Codierung angegeben
werden.

\begin{lstlisting}[language=Java, caption={Beispiel für Codierten InputStream}]
ByteArrayInputStream byteStream = new ByteArrayInputStream("Test mit ÄÖÜ".getBytes(StandardCharsets.UTF_8));
Reader reader = new InputStreamReader(byteStream, StandardCharsets.UTF_8);
StringBuilder result = new StringBuilder();
int character;

try {
    while ((character = reader.read()) != -1) {
    result.append((char) character);
    }
} finally {
    reader.close();
}
\end{lstlisting}

Verfügbare StandardCharsets sind:

\begin{itemize}
    \item \textbf{ISO\_8859\_1}
    \item \textbf{US\_ASCII}
    \item \textbf{UTF\_16}
    \item \textbf{UTF\_16BE}
    \item \textbf{UTF\_16LE}
    \item \textbf{UTF\_8}
\end{itemize}

Die Codierung eines \lstinline{Readers} kann nicht geändert werden, da dieser
bereits Zeichen liefert und eben keine Bytes. Allerdings kann über die
Instanzmethode \lstinline{byte[] getBytes()} der String Klasse die Codierung
angepasst werden:

\begin{lstlisting}[language=Java, caption={Beispiel für String Codierung ändern}]
String string = "examplestring";
byte[] bytes = string.getBytes();

String asciiEncodedString = new String(bytes, StandardCharsets.US_ASCII);

System.out.println(asciiEncodedString); 

String utf8EncodedString = new String(bytes, StandardCharsets.UTF_8);

System.out.println(utf8EncodedString);  

String utf16EncodedString = new String(bytes, StandardCharsets.UTF_16);

System.out.println(utf16EncodedString); 
\end{lstlisting}

Wenn bytes in einer Tabelle nicht gefunden werden, werden diese über ein \lstinline{?}
repräsentiert. Wird eine falsche Codierung angewandt, so werden die Zeichen
anders als erwartet interpretiert. Die Methode \lstinline{getBytes()} übersetzt den String
standardmäßig in die Standardcodierung der JVM, welche in Deutschland idr.
UTF-8 is. Diese Codierung kann allerdings über den Parameter explizit angegeben
werden.

\section{OutputStream}

Der \lstinline{OutputStream} ist die Oberklasse aller Byteorientierten Ausgabeströmen. Er
besitzt zwei Methoden:
\begin{itemize}
    \item \lstinline{write(int)}
    \item \lstinline{write(byte[])}
\end{itemize}

Die Methode \lstinline{write(int)} istbei externen Datensenken potentiell
Ineffizient. Aufgrunddessen gibt es hier die Klasse \lstinline{BufferedOutputStream}.

\begin{lstlisting}[language=Java, caption={Beispiel für OutputStream}]
OutputStream bs = new ByteArrayOutputStream();

byte[] bytes = new byte[]{1, 2, 3, 4, 5}

for (byte b : bytes) {
    bs.write(b);
}
\end{lstlisting}

\subsection{BufferedOutputStream}

Der \lstinline{BufferedOutputStream} kapselt einen \lstinline{OutputStream} und schreibt dort
blockweise hinen. Die \lstinline{write(int)} Methode dieser Klasse ist auch für
externe Datensenken effizient.

\subsection{ByteArrayOutputStream}

der \lstinline{ByteArrayOutputStream} eignet sich zum Testen von Methoden mit
\lstinline{OutputStream}-Parametern. Dieser besitzt unter anderem 
die Instanzmethoden \lstinline{size()} um die Größe des 
Buffers zu bekommen und \lstinline{toByteArray()} um den 
Inhalt des \lstinline{OutputStream} als byte Array zu erhalten.

\section{Writer}

Der \lstinline{Writer} ist de Oberklasse aller Zeichenorientierten Ausgabeströme. 
Er besitzt die Methoden:

\begin{itemize}
    \item \lstinline{write(int)}
    \item \lstinline{write(char[])}
\end{itemize}
Das \lstinline{write(int)} ist auch hier bei Externen Datensenken 
sehr ineffizient.

\subsection{BufferedWriter}

Der \lstinline{BufferedWriter} kapselt einen \lstinline{Writer} ein und 
schreibt dort Blockweise hinen. Die Methode \lstinline{write(int)} ist 
hier effizient.

\subsection{PrintWriter}

Der \lstinline{PrintWriter} kapselt einen \lstinline{Writer} und hat 
die \lstinline{write()} Methode mit allen Primitiven Datentypen überladen.
Über ihn kann komfortabler in den \lstinline{Writer} geschrieben werden.

\subsection{StringWriter}

Der \lstinline{StringWriter} eignet sich sehr gut zum Testen von Methoden mit \lstinline{Writer}-Parametern.

\section{mit \texttt{I/O}-Strömen arbeiten}

Der Zugriff auf \texttt{I/O}-Ströme ist durch die Außnahmen relativ komplitziert. Jeder kann beim Lesen/Schreiben eine \lstinline{IOException} werfen, welche möglichst behandelt werden sollte. Außerdem müssen \texttt{I/O}-Ströme immer geschlossen werden, wenn diese geöffnet wurden.

\begin{lstlisting}[language=Java, caption={Kompletter beispielzugriff auf InputStream}]
InputStream is = null;
try {
    is = new FileInputStream(path);
    is.read();
    ...
} catch (IOException e) {
    ...
} finally {
    try {
        if (is != null) {
            is.close();
        }
    } catch (IOException e) {
        ...
    }
}
\end{lstlisting}

Um dies zu vereinfachen, gibt es seit Java 7 ein Try mit Ressourcen. Dieser schließt die Ressourcen automatisch.

\begin{lstlisting}[language=Java, caption={Kompletter beispielzugriff auf InputStream}]
try (InputStream is = new FileInputStream(path)) {
    is.read();
    ...
} catch (Exception e) {
    ...
}
\end{lstlisting}