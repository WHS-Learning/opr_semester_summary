\chapter{Pakete in Java}
\label{chap:pakete}

\textit{Pakete schaffen Ordnung.} Sie bilden eine übersichtliche, hierarchische Struktur innerhalb von Java-Projekten.
Im Prinzip sind Pakete Ordnerstrukturen, die Klassen und weitere (Unter-)Pakete enthalten können.
Die Adressierung eines Paketes erfolgt durch seinen vollqualifizierten Paketpfad, der die Hierarchie widerspiegelt:
$p_1.p_2.\dots.p_n$. Dabei ist $p_n$ ein Paket innerhalb von $p_{n-1}$, welches wiederum in $p_{n-2}$ liegt, und so weiter.

Klassen werden analog dazu über ihren vollqualifizierten Namen angesprochen: $p_1.p_2.\dots.p_n.K$, wobei $K$ der Name der Klasse ist.

\begin{lstlisting}[language=Java, caption={Beispiele für Paket- und Klassenspezifikationen}, label=lst:paketbeispiele]
// Paket "regex", enthalten im Paket "util", das im Paket "java" liegt:
java.util.regex

// Klasse "StringTokenizer", enthalten im Paket "util", das im Paket "java" liegt:
java.util.StringTokenizer
\end{lstlisting}

Eine Klasse wird einem spezifischen Paket zugeordnet, indem die \texttt{package}-Anweisung als erste Zeile in der Quelldatei deklariert wird:
\begin{lstlisting}[language=Java, caption={Deklaration der Paketzugehörigkeit}]
package java.util; // Beispielhafter Paketpfad

public class Main {
  // ...
}
\end{lstlisting}
Diese Angabe ist für jede Java-Datei, die zu einem Paket gehört, obligatorisch. Vor der \texttt{package}-Anweisung dürfen lediglich Kommentare stehen
- keine anderen Klassendefinitionen, Variablen oder Importanweisungen.

\section{Importieren von Klassen und Paketen}
\label{sec:importieren}

Um Klassen aus anderen Paketen innerhalb der eigenen Klasse nutzen zu können, müssen diese \textit{importiert} werden.
Die \texttt{import}-Anweisungen stehen dabei stets am Anfang der Datei, nach der optionalen \texttt{package}-Deklaration und 
\textit{vor} dem Klassenkopf (der Klassendefinition).

In modernen Entwicklungsumgebungen (IDEs) werden \texttt{import}-Anweisungen häufig automatisch hinzugefügt oder vorgeschlagen. 
Dennoch kann es in bestimmten Szenarien notwendig sein, diese manuell anzupassen oder zu ergänzen.

\subsection{Importieren aller Klassen eines Pakets (Wildcard-Import)}
\label{ssec:wildcard_import}

Um alle öffentlichen Klassen eines bestimmten Pakets verfügbar zu machen, kann ein sogenannter Wildcard-Import verwendet werden. 
Dieser wird durch das Sternchen (\texttt{*}) symbolisiert. Beispielsweise importiert die Anweisung
\begin{lstlisting}[language=Java, caption={Wildcard-Import des java.util Pakets}]
import java.util.*;
\end{lstlisting}
alle öffentlichen Klassen aus dem Paket \texttt{java.util}. Die importierten Klassen können daraufhin so verwendet werden, 
als wären sie im aktuellen Paket definiert worden.

\paragraph{Vorsicht bei Namenskonflikten:}
Wenn zwei oder mehrere Klassen mit identischem Namen aus unterschiedlichen Paketen importiert werden 
(oder eine importierte Klasse denselben Namen wie eine Klasse im aktuellen Paket hat), entsteht ein Namenskonflikt. 
In solchen Fällen müssen die betroffenen Klassen vollständig Qualifiziert verwendet werden, um Eindeutigkeit zu gewährleisten.
Beispiel:
\begin{lstlisting}[language=Java, caption={Umgang mit Namenskonflikten}]
import java.util.Date;
import java.sql.Date;

public class DateTest {
  java.util.Date utilDate; // Vollständig qualifiziert
  java.sql.Date sqlDate;   // Vollständig qualifiziert

  public void example() {
    utilDate = new java.util.Date(); // Eindeutige Zuweisung
    // Date ambiguDate; // Fehler: Mehrdeutigkeit ohne Qualifizierung
  }
}
\end{lstlisting}

\section{Vollständige Qualifizierung}
Pakete und Klassen, welche über ihren vollen Namen angesprochen werden, nennt man 
\textit{vollständig Qualifiziert}. So wird also anstelle von \lstinline{Date} 
\lstinline{java.util.Date} verwendet. Klassen müssen vollständig Qualifiziert werden,
wenn mehrere Klassen gleich heißen. Beispielsweise \lstinline{java.util.Date} und 
\lstinline{java.sql.Date}. 

\section{Das Paket \texttt{java.lang}}
\label{sec:java_lang}

Das Paket \texttt{java.lang} nimmt eine Sonderstellung ein. Es enthält fundamentale Klassen, die für die grundlegende 
Funktionalität der Java-Sprache unerlässlich sind. Dazu gehören beispielsweise:
\begin{itemize}
    \item \texttt{Object}: Die Wurzelklasse aller Java-Klassen.
    \item \texttt{String}: Für die Verarbeitung von Zeichenketten.
    \item \texttt{System}: Bietet Zugriff auf systemabhängige Ressourcen und Funktionen.
    \item Wrapper-Klassen für primitive Datentypen (z.B. \texttt{Integer}, \texttt{Boolean}).
    \item \texttt{Math}: Stellt mathematische Funktionen bereit.
\end{itemize}
Klassen aus dem \texttt{java.lang}-Paket sind \textbf{immer automatisch verfügbar} und müssen nicht explizit importiert werden.

\subsection{Statische Importanweisungen}
\label{ssec:statische_importe}

Neben dem Import von Klassen ist es auch möglich, statische Methoden und Variablen direkt zu importieren. 
Dies geschieht mithilfe der \texttt{import static}-Anweisung.
Durch einen statischen Import können statische Mitglieder einer Klasse so aufgerufen werden, als wären sie 
in der aktuellen Klasse definiert, ohne den Klassennamen voranstellen zu müssen.

\paragraph{Beispiel für statischen Import:}
Die Anweisung
\begin{lstlisting}[language=Java, caption={Statischer Import der max-Methode}]
import static java.lang.Math.max;
\end{lstlisting}
importiert die statische Methode \texttt{max} aus der Klasse \texttt{Math} (welche sich im Paket \texttt{java.lang} befindet).
Anschließend kann die Methode direkt verwendet werden:
\begin{lstlisting}[language=Java, caption={Verwendung nach statischem Import}]
public class MathExample {
  public int biggerNumber(int a, int b) {
    return max(a, b); // Aufruf ohne "Math." Präfix
  }
}
\end{lstlisting}

Es ist auch möglich, alle statischen Mitglieder einer Klasse zu importieren:
\begin{lstlisting}[language=Java, caption={Statischer Import aller Mitglieder von Math}]
import static java.lang.Math.*;

public class MathExampleAll {
  public double circleSurfaceArea(double radius) {
    return PI * pow(radius, 2); // PI und pow direkt verfügbar
  }
}
\end{lstlisting}

\textbf{Wichtig:} Nicht-statische Variablen oder Methoden können nicht über \texttt{import static} importiert werden. 
Diese Art des Imports ist ausschließlich statischen Mitgliedern vorbehalten. Instanzvariablen (nicht statische Variablen) können
gar nicht importiert werden.
