\chapter{Hüllenklassen}

\section{Definition und Notwendigkeit}
Hüllenklassen umhüllen primitive Datentypen. Dies ist beispielsweise für die Collection Klassen notwendig, da diese nur Objekte als Parameter aufnehmen. Hüllklassen haben keine andere Aufgabe als einen primitiven Datentyp zu umhüllen.

\section{Übersicht der Hüllenklassen}
Die folgende Tabelle zeigt die primitiven Datentypen und ihre entsprechenden Hüllenklassen:

\begin{center}
\begin{tabular}{|l|l|}
\hline
\textbf{Primitiver Datentyp} & \textbf{Hüllklasse} \\
\hline\hline
boolean             & Boolean     \\
\hline
byte                & Byte        \\
\hline
char                & Character   \\
\hline
short               & Short       \\
\hline
int                 & Integer     \\
\hline
long                & Long        \\
\hline
float               & Float       \\
\hline
double              & Double      \\
\hline
\end{tabular}
\end{center}

\section{Erstellung von Hüllenklassen}
\label{sec:erstellung_huellenklassen}
Hüllenklassen werden wie jedes andere Objekt mittels des \texttt{new}-Operators instanziiert:

\begin{lstlisting}[caption={Erstellung von Hüllenklassen-Objekten}]
Integer i = new Integer(10);
Boolean b = new Boolean(true);
\end{lstlisting}

\section{Verwendung vor JDK 1.5}
\label{sec:verwendung_vor_jdk1_5}
Bis zum Java Development Kit (JDK) Version 1.4 war eine direkte Kombination von Hüllenklassen-Objekten mit primitiven Datentypen in arithmetischen oder logischen Ausdrücken nicht möglich.

\subsection{Ungültige Ausdrücke (Theoretisch)}
\label{ssec:ungueltige_ausdruecke_vor_jdk1_5}
Die folgenden Operationen wären ohne explizites Auspacken ungültig gewesen:
\begin{lstlisting}[caption={Theoretisch ungültige Ausdrücke vor JDK 1.5}]
Integer iObj = new Integer(10);
Boolean bObj = new Boolean(true);

// iObj + 20; // Compilerfehler: Operator + nicht anwendbar auf Integer, int
// bObj && true; // Compilerfehler: Operator && nicht anwendbar auf Boolean, boolean
\end{lstlisting}

\subsection{Explizites Entpacken (Unboxing)}
\label{ssec:explizites_entpacken}
Um den Wert eines Hüllenklassen-Objekts verwenden zu können, musste dieser explizit mittels einer entsprechenden Methode (z.B. \texttt{intValue()}) entpackt werden:
\begin{lstlisting}[caption={Explizites Entpacken des Wertes}]
Integer iObj = new Integer(10);
int j = iObj.intValue(); // j erhält den Wert 10

Boolean bObj = new Boolean(true);
boolean bVal = bObj.booleanValue(); // bVal erhält den Wert true
\end{lstlisting}

\section{Autoboxing und Autounboxing (seit JDK 1.5)}
\label{sec:autoboxing_autounboxing}
Seit dem JDK 1.5 führt der Java-Compiler das notwendige Umwandeln zwischen primitiven Typen und ihren Hüllenklassen automatisch durch. Dies wird als Autoboxing (primitiv zu Hüllklasse) und Autounboxing (Hüllklasse zu primitiv) bezeichnet.
Der Compiler fügt die notwendigen Aufrufe (z.B. \texttt{intValue()}) automatisch ein, wodurch der Code lesbarer wird:

\begin{lstlisting}[caption={Automatische Umwandlung durch Autoboxing/Autounboxing}]
Integer i = new Integer(10); // Explizite Erstellung (Boxing auch möglich: Integer i = 10;)
Boolean b = new Boolean(true); // Explizite Erstellung (Boxing auch möglich: Boolean b = true;)

int k = i + 20; // Autounboxing: i wird zu int entpackt
if (b && true) { // Autounboxing: b wird zu boolean entpackt
// ...
}
\end{lstlisting}

\section{Umwandlung von Strings in Hüllenklassenwerte}
\label{sec:umwandlung_strings}
Hüllenklassen bieten statische Methoden (z.B. \texttt{parseInt()}), um Strings, die Zahlen oder Wahrheitswerte repräsentieren, in den entsprechenden primitiven Datentyp oder direkt in ein Hüllenklassen-Objekt umzuwandeln.

\begin{lstlisting}[caption={Parsen von Strings zu primitiven Typen/Hüllenklassen}]
int i = Integer.parseInt("1337");
double d = Double.parseDouble("-1.25E-13");
boolean b = Boolean.parseBoolean("true");

// Erzeugt direkt Hüllenklassen-Objekte (oft weniger gebräuchlich als das Parsen zu primitiven Typen)
// Integer iObj = Integer.valueOf("1337");
// Double dObj = Double.valueOf("-1.25E-13");
// Boolean bObj = Boolean.valueOf("true");
\end{lstlisting}