\chapter{Klassenhierarchie}
\label{chap:klassenhierarchie}

In der objektorientierten Programmierung ermöglicht die Klassenhierarchie die
Schaffung von Beziehungen zwischen Klassen, wodurch Code-Wiederverwendung
gefördert und eine logische Struktur aufgebaut wird. Dieses Kapitel erläutert
die fundamentalen Konzepte der Vererbung, Sichtbarkeitsmodifikatoren und
Bindungsarten in Java.

\section{Vererbung: Das Fundament der Hierarchie}
\label{sec:erben}

Eine Klasse kann Eigenschaften und Methoden von einer anderen Klasse
übernehmen. Dieser Mechanismus wird als Vererbung bezeichnet und mit dem
Schlüsselwort \texttt{extends} realisiert.

\paragraph{Syntax der Vererbung}
Die grundlegende Syntax zur Definition einer Klasse, die von einer anderen
erbt, ist wie folgt:
\begin{lstlisting}[language=Java, caption={Deklaration einer abgeleiteten Klasse}]
class Unterklasse extends Oberklasse {
    // Zusätzliche Attribute und Methoden der Unterklasse
    // oder überschriebene Methoden der Oberklasse
}
\end{lstlisting}
In diesem Beispiel ist \texttt{Oberklasse} die direkte Superklasse (auch
Elternklasse genannt) von \texttt{Unterklasse}. Umgekehrt ist
\texttt{Unterklasse} eine direkte Subklasse (auch Kindklasse genannt) von
\texttt{Oberklasse}. Eine Klasse kann beliebig viele direkte Unterklassen
haben, jedoch stets nur \textbf{eine} direkte Oberklasse. Java unterstützt
keine Mehrfachvererbung von Klassen. Die Kette der Oberklassen bildet die
Vererbungshierarchie.

\paragraph{Was wird vererbt?}
Eine Unterklasse erbt von ihrer Oberklasse:
\begin{itemize}
  \item Alle \texttt{public} und \texttt{protected} Instanzmethoden und Klassenmethoden
        (\texttt{static}).
  \item Alle Instanzvariablen und Klassenvariablen (\texttt{static}). Auch
        \texttt{private} deklarierte Variablen der Oberklasse sind Teil des
        Speicherabbilds von Objekten der Unterklasse. Jedoch ist ein direkter Zugriff
        auf \texttt{private} Variablen der Oberklasse aus der Unterklasse heraus nicht
        möglich; dieser kann nur über geerbte \texttt{public} oder \texttt{protected}
        Methoden der Oberklasse erfolgen (Prinzip der Datenkapselung).
\end{itemize}
Die Vererbung ist transitiv: Eine Klasse erbt auch die Member, die ihre Oberklasse bereits von deren Oberklasse geerbt hat. Geerbte Member
verhalten sich so, als wären sie in der Unterklasse selbst deklariert, sofern sie nicht überschrieben werden.

\paragraph{Erweiterung und Spezialisierung durch die Unterklasse}
Unterklassen können die geerbten Funktionalitäten erweitern und spezialisieren:
\begin{itemize}
  \item Sie können neue Methoden und Variablen definieren, die nur für die Unterklasse
        spezifisch sind.
  \item Sie können geerbte (nicht-\texttt{final} und nicht-\texttt{private})
        Instanzmethoden \textbf{überschreiben} (\texttt{@Override}), um eine
        spezifische Implementierung bereitzustellen.
\end{itemize}

\paragraph{Nicht vererbte Elemente}
Folgende Elemente werden \textbf{nicht} an Unterklassen vererbt:
\begin{itemize}
  \item \textbf{Konstruktoren:} Jede Klasse muss ihre eigenen Konstruktoren definieren oder den Default-Konstruktor verwenden. Der Konstruktor
        der Oberklasse kann jedoch mittels \texttt{super()} aufgerufen werden (siehe Abschnitt~\ref{sec:super}).
  \item \textbf{Private Methoden:} Private Methoden der Oberklasse sind für die Unterklasse unsichtbar und werden daher nicht vererbt. Folglich
        können sie auch nicht im Sinne der Polymorphie überschrieben werden. Definiert eine Unterklasse eine Methode mit derselben Signatur wie eine
        private Methode der Oberklasse, so handelt es sich um eine vollkommen neue, unabhängige Methode. Es besteht keine polymorphe Beziehung zwischen
        diesen beiden Methoden; die Methode in der Unterklasse \textit{verdeckt} (shadows) lediglich die Methode der Oberklasse, falls diese z.B. über
        einen Cast auf den Oberklassentyp aufgerufen würde (was bei privaten Methoden aber ohnehin nicht direkt von außen geht).
\end{itemize}

\subsection{Der Sichtbarkeitsmodifikator \texttt{protected}}
\label{ssec:protected}

Der Modifikator \texttt{protected} stellt eine Sichtbarkeitsstufe zwischen
\texttt{public} und \texttt{private} dar. Mit \texttt{protected} deklarierte
Member (Methoden oder Variablen) sind sichtbar:
\begin{itemize}
  \item Innerhalb der eigenen Klasse.
  \item Innerhalb aller Unterklassen, auch wenn diese sich in anderen Paketen befinden.
  \item Innerhalb aller anderen Klassen desselben Pakets.
\end{itemize}
Obwohl \texttt{protected} für Instanzvariablen in Vererbungsszenarien nützlich erscheinen mag, sollten Instanzvariablen aus Gründen der Kapselung
und Wartbarkeit in den meisten Fällen als \texttt{private} deklariert werden. Der Zugriff sollte dann über \texttt{public} oder \texttt{protected}
Getter- und Setter-Methoden erfolgen.

\subsection{Bindung von Methodenaufrufen: Statisch vs. Dynamisch}
\label{ssec:bindungen}

Die Bindung legt fest, welche konkrete Methodenimplementierung bei einem Aufruf
ausgeführt wird.

\paragraph{Statische Bindung (Early Binding)}
Bei der statischen Bindung wird bereits zur Kompilierzeit festgelegt, welche
Methodenimplementierung ausgeführt wird. Dies geschieht typischerweise für:
\begin{itemize}
  \item \texttt{private}-Methoden
  \item \texttt{static}-Methoden
  \item \texttt{final}-Instanzmethoden
  \item Aufrufe mit \texttt{super.}
  \item Aufrufe, bei denen der Compiler den exakten Objekttyp eindeutig bestimmen kann.
\end{itemize}

\begin{lstlisting}[language=Java, caption={Beispiele für statische Bindung}]
class Vehicle {
  public static String getCategorie() {
    return "Allgemeines Fahrzeug";
  }
  public final void startMotor() {
    System.out.println("Motor gestartet (finale Methode aus Fahrzeug)");
  }
}

class Car extends Vehicle {
  // Diese Methode verdeckt getCategorie() von Fahrzeug, überschreibt sie aber nicht.
  public static String getCategorie() {
    return "Automobil";
  }
}

public class TestBinding {
  public static void main(String[] args) {
    // Aufruf statischer Methoden: Immer statische Bindung
    System.out.println(Vehicle.getCategorie()); // Gibt "Allgemeines Fahrzeug" aus
    System.out.println(Car.getCategorie());     // Gibt "Automobil" aus

    Car meinAuto = new Car();
    // Aufruf einer finalen Methode: Statische Bindung
    meinAuto.startMotor(); // Ruft Vehicle.startMotor()

    Vehicle meinFahrzeug = new Car();
    // Obwohl meinFahrzeug auf ein Car-Objekt zeigt, wird bei statischen Methoden
    // der deklarierte Typ der Referenz (Car) für die Bindung verwendet.
    // System.out.println(meinFahrzeug.getCategorie()); // Schlechter Stil! Sollte Vehicle.getCategorie() sein.
    // Würde "Allgemeines Fahrzeug" ausgeben.
  }
}
\end{lstlisting}

\paragraph{Dynamische Bindung (Late Binding)}
Bei der dynamischen Bindung wird erst zur Laufzeit entschieden, welche
Methodenimplementierung ausgeführt wird. Dies ist das Kernprinzip der
Polymorphie und tritt bei überschriebenen Instanzmethoden auf. Der Java Virtual
Machine (JVM) ermittelt den tatsächlichen Typ des Objekts, auf dem die Methode
aufgerufen wird, und wählt die entsprechende Implementierung aus.

\begin{lstlisting}[language=Java, caption={Beispiel für dynamische Bindung}]
class Animal {
  public void say() {
    System.out.println("Ein Tier gibt einen Laut von sich.");
  }
}

class Dog extends Animal {
  @Override
  public void say() {
    System.out.println("Wuff! Wuff!");
  }
}

class Cat extends Animal {
  @Override
  public void say() {
    System.out.println("Miau!");
  }
}

public class Zoo {
  public static void main(String[] args) {
    Animal[] tiere = new Animal[3];
    tiere[0] = new Dog();       // Hund-Objekt
    tiere[1] = new Cat();       // Katze-Objekt
    tiere[2] = new Animal();    // Tier-Objekt

    for (Tier aktuellesTier : tiere) {
      // Dynamische Bindung:
      // Zur Laufzeit wird die spezifische say()-Methode
      // des tatsächlichen Objekttyps aufgerufen.
      aktuellesTier.say();
    }
    // Ausgabe:
    // Wuff! Wuff!
    // Miau!
    // Ein Tier gibt einen Laut von sich.
  }
}
\end{lstlisting}
Der Compiler prüft lediglich, ob eine Methode \texttt{say()} im deklarierten
Typ \texttt{Animal} existiert. Welche konkrete Implementierung dann ausgeführt
wird, entscheidet die JVM basierend auf dem Laufzeittyp des Objekts in
\texttt{aktuellesTier}.

\section{Das Schlüsselwort \texttt{super} (Prüfungsrelevant!)}
\label{sec:super}

Das Schlüsselwort \texttt{super} hat in Java zwei primäre Verwendungszwecke im
Kontext der Vererbung:
\begin{enumerate}
  \item Aufruf des Konstruktors der direkten Oberklasse.
  \item Zugriff auf Member (Methoden oder Variablen) der Oberklasse, die in der
        aktuellen Klasse möglicherweise überschrieben oder verdeckt wurden.
\end{enumerate}

\paragraph{Aufruf des Oberklassen-Konstruktors mit \texttt{super()}}
Der Aufruf \texttt{super()} dient dazu, einen Konstruktor der direkten Oberklasse explizit auszuführen.
\begin{itemize}
  \item Dieser Aufruf \textbf{muss}, falls vorhanden, die \textbf{erste Anweisung} im
        Konstruktor der Unterklasse sein.
  \item Die Parameterliste von \texttt{super(...)} muss mit der Signatur eines
        Konstruktors der Oberklasse übereinstimmen.
\end{itemize}

\begin{lstlisting}[language=Java, caption={Expliziter Aufruf des Oberklassen-Konstruktors via \texttt{super()}}]
class BaseClass {
  String name;
  public BaseClass(String name) {
    this.name = name;
    System.out.println("Konstruktor Basisklasse aufgerufen mit: " + name);
  }
  public BaseClass() {
    this.name = "Default";
    System.out.println("Parameterloser Konstruktor Basisklasse aufgerufen");
  }
}

class ChildClass extends BaseClass {
  public ChildClass(String spezifischerName) {
    super(spezifischerName); // Ruft BaseClass(String name) auf
    System.out.println("Konstruktor ChildClass aufgerufen");
  }
  public ChildClass() {
    super(); // Ruft BaseClass() auf
    // Alternativ: super("DefaultAbgeleitet");
    System.out.println("Parameterloser Konstruktor ChildClass aufgerufen");
  }
}

public class TestSuper {
  public static void main(String[] args) {
    ChildClass obj = new ChildClass("TestObjekt");
    // Ausgabe:
    // Konstruktor BaseClass aufgerufen mit: TestObjekt
    // Konstruktor ChildClass aufgerufen

    ChildClass obj2 = new ChildClass();
    // Ausgabe:
    // Parameterloser Konstruktor BaseClass aufgerufen
    // Parameterloser Konstruktor ChildClass aufgerufen
  }
}
\end{lstlisting}

\paragraph{Impliziter Aufruf von \texttt{super()}}
Wenn im Konstruktor einer Unterklasse \textbf{kein expliziter Aufruf} von \texttt{super(...)} (oder \texttt{this(...)}, siehe unten)
als erste Anweisung steht, fügt der Java-Compiler automatisch einen parameterlosen Aufruf \texttt{super();} ein.
\begin{itemize}
  \item Dies gilt auch für Klassen, die nicht explizit von einer anderen Klasse erben -
        diese erben implizit von der Klasse \texttt{Object}, die einen parameterlosen
        Konstruktor besitzt.
  \item Wenn eine Klasse gar keinen Konstruktor definiert, fügt der Compiler einen
        öffentlichen, parameterlosen Default-Konstruktor ein, der ebenfalls implizit
        \texttt{super();} aufruft.
\end{itemize}

\begin{lstlisting}[language=Java, caption={Impliziter Konstruktor und \texttt{super()}-Aufruf durch den Compiler}]
class Ober {
  public Ober() {
    System.out.println("Konstruktor Ober");
  }
}

class Unter extends Ober {
  // Kein expliziter Konstruktor definiert
}
// Der Compiler generiert für Klasse Unter:
// public Unter() {
//   super(); // Impliziter Aufruf des Ober-Konstruktors
// }

public class TestImplizit {
  public static void main(String[] args) {
    Unter u = new Unter(); // Führt zu Ausgabe: "Konstruktor Ober"
  }
}
\end{lstlisting}

\paragraph{Kompilierfehler bei fehlendem passenden Oberklassen-Konstruktor}
Der implizite \texttt{super();}-Aufruf (oder ein expliziter ohne Argumente)
funktioniert nur, wenn die direkte Oberklasse einen zugänglichen,
parameterlosen Konstruktor besitzt. Ist dies nicht der Fall (z.B. weil die
Oberklasse nur Konstruktoren mit Parametern definiert), \textbf{muss} die
Unterklasse in jedem ihrer Konstruktoren explizit einen passenden Konstruktor
der Oberklasse mittels \texttt{super(...)} mit den erforderlichen Argumenten
aufrufen. Andernfalls entsteht ein Kompilierfehler. Der Compiler kann die
benötigten Argumente nicht "erraten".

\begin{lstlisting}[language=Java, caption={Kompilierfehler: \texttt{super()}-Aufruf scheitert bei parametrisiertem Superkonstruktor}]
class Elternteil {
  private final String id;
  public Elternteil(String id) { // Nur ein Konstruktor mit Parameter
    this.id = id;
  }
}

class KindValid extends Elternteil {
  public KindValid(String kindId, String elternId) {
        super(elternId); // Korrekt: Expliziter Aufruf des passenden Konstruktors
  }
}

class KindError1 extends Elternteil {
  public KindError1() {
    // FEHLER! Implizites super() würde Elternteil() suchen, gibt es aber nicht.
    // Explizites super("someID") wäre nötig.
  }
}

class KindError2 extends Elternteil {
  // FEHLER! Compiler fügt Default-Konstruktor ein, der super() aufruft.
  // Elternteil() existiert nicht.
}
\end{lstlisting}

\paragraph{Zugriff auf Member der Oberklasse mit \texttt{super.}}
Mit \texttt{super.methodenName()} oder \texttt{super.variablenName} kann auf eine Methode oder Variable der Oberklasse zugegriffen werden,
selbst wenn diese in der aktuellen Klasse überschrieben (Methode) oder verdeckt (Variable) wurde. Dies ist nützlich, um die Funktionalität
der Oberklasse zu erweitern, anstatt sie komplett zu ersetzen.

\begin{lstlisting}[language=Java, caption={Zugriff auf überschriebene Methode der Oberklasse via \texttt{super.}}]
class Basis {
  public void anzeige() {
    System.out.println("Anzeige aus Basis");
  }
  protected String info = "Info aus Basis";
}

class Erweitert extends Basis {
  @Override
  public void anzeige() {
    super.anzeige(); // Ruft anzeige() der Klasse Basis auf
    System.out.println("Anzeige aus Erweitert");
  }
    
  public void zeigeInfos() {
    System.out.println("Eigene Info: " + this.info); // this.info ist hier redundant, da info nicht neu deklariert wurde
    System.out.println("Basis Info: " + super.info); // Greift auf info der Basisklasse zu
  }

  // Beispiel für Variablenverdeckung (Shadowing) - generell vermeiden!
  // protected String info = "Info aus Erweitert"; 
  // public void zeigeInfosMitVerdeckung() {
  //     System.out.println("Eigene Info (verdeckt): " + this.info); // Info aus Erweitert
  //     System.out.println("Basis Info (via super): " + super.info); // Info aus Basis
  // }
}

public class TestSuperMember {
  public static void main(String[] args) {
    Erweitert ext = new Erweitert();
    ext.anzeige();
    // Ausgabe:
    // Anzeige aus Basis
    // Anzeige aus Erweitert
    ext.zeigeInfos();
  }
}
\end{lstlisting}

\section{Konstruktoraufrufe mit \texttt{this()}}
\label{sec:this_konstruktor}

Analog zum Aufruf eines Oberklassen-Konstruktors mit \texttt{super()} kann mit
\texttt{this()} ein anderer Konstruktor derselben Klasse aufgerufen werden.
Dies wird als Konstruktorverkettung (constructor chaining) bezeichnet und dient
dazu, Code-Duplizierung innerhalb verschiedener Konstruktoren einer Klasse zu
vermeiden.
\begin{itemize}
  \item Der Aufruf \texttt{this(...)} \textbf{muss}, falls vorhanden, die \textbf{erste
          Anweisung} im Konstruktor sein.
  \item Ein Konstruktor kann entweder einen \texttt{super(...)} oder einen
        \texttt{this(...)} Aufruf als erste Anweisung enthalten, aber nicht beide.
  \item Mindestens ein Konstruktor in der Kette muss (implizit oder explizit) den
        Konstruktor der Oberklasse via \texttt{super(...)} aufrufen.
\end{itemize}

\begin{lstlisting}[language=Java, caption={Aufruf eines anderen Konstruktors derselben Klasse via \texttt{this()}}]
class Benutzer {
  private final String benutzername;
  private final String email;
  private boolean istAktiv;

  public Benutzer(String benutzername, String email) {
    this.benutzername = benutzername;
    this.email = email;
    this.istAktiv = true; // Standardwert
    System.out.println("Benutzer erstellt: " + benutzername + ", Aktiv: " + istAktiv);
  }

  public Benutzer(String benutzername) {
    this(benutzername, benutzername + "@example.com"); // Ruft den ersten Konstruktor auf
    System.out.println("Benutzer (nur Name) erstellt, E-Mail generiert.");
  }
    
  public Benutzer() {
    this("gast"); // Ruft den zweiten Konstruktor auf, der dann den ersten aufruft
    this.istAktiv = false; // Überschreibt den Standardwert für Gast-Benutzer
    System.out.println("Gast-Benutzer erstellt, Inaktiv gesetzt.");
  }
}

public class TestThisKonstruktor {
  public static void main(String[] args) {
    Benutzer b1 = new Benutzer("alice", "alice@mail.com");
    System.out.println("---");
    Benutzer b2 = new Benutzer("bob");
    System.out.println("---");
    Benutzer b3 = new Benutzer();
  }
}
\end{lstlisting}

\section{Finale Klassen und Methoden}
\label{sec:final}

Das Schlüsselwort \texttt{final} hat im Kontext der Vererbung spezielle
Bedeutungen:

\paragraph{Finale Klassen}
Eine mit \texttt{final} deklarierte Klasse \textbf{kann nicht erweitert
  werden}, d.h., es können keine Unterklassen von ihr abgeleitet werden.
\begin{lstlisting}[language=Java, caption={\texttt{final} Klasse}]
public final class Unveraenderlich {
  // ...
}

// class VersuchAbleitung extends Unveraenderlich { } // KOMPILIERFEHLER!
\end{lstlisting}
Dies wird oft für Klassen verwendet, deren Implementierung als abgeschlossen
und nicht für Erweiterungen vorgesehen gilt (z.B. \texttt{String} oder
\texttt{Integer}).

\paragraph{Finale Methoden}
Eine mit \texttt{final} deklarierte Methode \textbf{kann in Unterklassen nicht
  überschrieben werden}.
\begin{lstlisting}[language=Java, caption={\texttt{final} Methode}]
class BasisMitFinal {
  public final void wichtigeOperation() {
    System.out.println("Diese Operation darf nicht geändert werden.");
  }
  public void andereOperation() {
    System.out.println("Diese Operation kann überschrieben werden.");
  }
}

class AbgeleitetVonFinal extends BasisMitFinal {
  // @Override
  // public final void wichtigeOperation() { } // KOMPILIERFEHLER!

  @Override
  public void andereOperation() {
    System.out.println("andereOperation in AbgeleitetVonFinal überschrieben.");
  }
}
\end{lstlisting}
Finale Methoden garantieren, dass das Verhalten dieser spezifischen Methode in
der gesamten Klassenhierarchie unterhalb ihrer Definition konstant bleibt. Sie
werden auch vom Compiler für Optimierungen herangezogen, da die Bindung
statisch erfolgen kann.

\section{Abstrakte Klassen}

Abstrakte Klassen definieren ein Verhalten, das von anderen Klassen
implementiert werden muss.
\begin{itemize}
  \item Objekte abstrakter Klassen können nicht instanziiert werden.
  \item Methoden können in abstrakten Klassen deklariert werden, wobei deren
        Implementierung in ableitenden Klassen erforderlich ist.
  \item Bereits implementierte Methoden können ebenfalls in abstrakten Klassen
        enthalten sein, welche von abgeleiteten Klassen verwendet oder überschrieben
        werden können.
\end{itemize}