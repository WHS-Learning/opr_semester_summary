\chapter{Collection-Klassen (Sehr Prüfungsrelevant)}
\label{chap:CollectionKlassen}

Arrays haben eine feste Größe, die sich während der Laufzeit nicht ändern kann.
Dies stellt häufig eine Einschränkung dar. Sogenannte Collection-Klassen (oder
Kollektionen) bieten hier flexiblere Alternativen. Es gibt verschiedene Arten
von Collection-Klassen, die sich grob in Listen (\texttt{List}), Mengen
(\texttt{Set}) und Abbildungen (\texttt{Map}) einteilen lassen. Jede Art und
ihre spezifischen Implementierungen haben eigene Vor- und Nachteile, die sie
für unterschiedliche Anwendungsfälle geeignet machen.

\section{Arrays}
Arrays sind streng genommen keine Collection-Klassen im Sinne des Java
Collections Frameworks. Dennoch ist es sinnvoll, sie hier zu betrachten, da sie
einen grundlegenden Mechanismus zur Gruppierung von Elementen darstellen und
als Vergleichsbasis für die eigentlichen Collection-Klassen dienen.

\subsection{Anwendungsbereiche}
Arrays eignen sich, wenn eine feste Anzahl von Elementen desselben Typs
geordnet gespeichert werden soll und sich diese Anzahl zur Laufzeit nicht
ändert.

\subsection{Vorteile}
Der Zugriff auf Elemente über ihren Index ist bei Arrays sehr schnell und
effizient (konstante Zeitkomplexität, $O(1)$).

\subsection{Nachteile}
Die Größe eines Arrays ist nach seiner Initialisierung unveränderlich. Eine
nachträgliche Größenänderung erfordert die Erstellung eines neuen Arrays und
das Kopieren der Elemente.

\subsection{Beispiel}
\begin{lstlisting}[caption={Beispiel für die Verwendung eines Arrays in Java}, label=lst:arrayExample]
int[] numbers = new int[5]; // Array der Größe 5 erstellen
numbers[3] = 42; // Dem Element am Index 3 den Wert 42 zuweisen
System.out.println(numbers[3]); // Wert am Index 3 ausgeben (Ausgabe: 42)
// numbers[5] = 7; // Dies würde eine ArrayIndexOutOfBoundsException auslösen
\end{lstlisting}

\section{ArrayList}
\texttt{ArrayList}s sind dynamische Arrays, d.h., sie können ihre Größe während der Laufzeit anpassen. Sie implementieren das \texttt{List}-Interface.

\subsection{Anwendungsbereiche}
\texttt{ArrayList}s sind dann sinnvoll, wenn Elemente geordnet gespeichert werden sollen und sich die Anzahl der Elemente dynamisch ändern kann. Sie bieten einen guten Kompromiss zwischen flexiblem Größenmanagement und schnellem Indexzugriff.

\subsection{Vorteile}
\begin{itemize}
    \item Die Größe der Liste kann zur Laufzeit dynamisch wachsen oder schrumpfen.
    \item Der Zugriff auf Elemente über ihren Index ist in der Regel schnell (amortisiert
          $O(1)$), ähnlich wie bei Arrays.
\end{itemize}

\subsection{Nachteile}
\begin{itemize}
    \item Das Einfügen oder Löschen von Elementen am Anfang oder in der Mitte der Liste
          kann langsam sein ($O(n)$), da nachfolgende Elemente verschoben werden müssen.
    \item Beim Überschreiten der internen Kapazität muss das zugrundeliegende Array
          vergrößert und die Elemente kopiert werden. Obwohl das Hinzufügen am Ende
          amortisiert $O(1)$ ist, kann diese Operation vereinzelt länger dauern.
\end{itemize}

\subsection{Beispiel}
\begin{lstlisting}[caption={Beispiel für die Verwendung einer ArrayList in Java}, label=lst:arrayListExample]
import java.util.ArrayList; // Import notwendig
import java.util.List; // Interface verwenden ist gute Praxis

List<Integer> list = new ArrayList<>(); // Neue, leere ArrayList erstellen
list.add(10); // Element 10 am Ende hinzufügen
list.add(20); // Element 20 am Ende hinzufügen
list.add(0, 5); // Element 5 am Index 0 einfügen (verschiebt andere Elemente)

System.out.println(list.get(1)); // Element am Index 1 ausgeben (Ausgabe: 10)
list.remove(0); // Element am Index 0 entfernen
\end{lstlisting}

\section{LinkedList}
Eine \texttt{LinkedList} speichert ihre Elemente in einer doppelt verketteten
Liste. Jeder Knoten enthält das Element selbst sowie Verweise auf das vorherige
und das nächste Element in der Sequenz. Sie implementiert ebenfalls das
\texttt{List}-Interface sowie das \texttt{Deque}-Interface (Double-Ended
Queue), siehe \gls{Deque}.

\subsection{Anwendungsbereiche}
\texttt{LinkedList}s eignen sich besonders gut, wenn häufig Elemente am Anfang oder Ende der Liste hinzugefügt oder entfernt werden müssen. Sie sind daher eine gute Wahl für die Implementierung von Stacks (\gls{Stapelspeicher}, \gls{LIFO}) oder Queues (Warteschlangen, \gls{FiFo}).

\subsection{Vorteile}
\begin{itemize}
    \item Sehr schnelles Einfügen und Löschen von Elementen am Anfang und Ende der Liste
          ($O(1)$).
    \item Dynamische Größenanpassung ohne die Notwendigkeit, große Speicherblöcke am
          Stück zu reservieren oder umzukopieren.
\end{itemize}

\subsection{Nachteile}
\begin{itemize}
    \item Der Zugriff auf ein Element anhand seines Indexes (z.B. mit
          \texttt{get(index)}) ist langsam ($O(n)$), da die Liste vom Anfang (oder Ende,
          je nachdem, was näher ist) bis zum gewünschten Element durchlaufen werden muss.
    \item Benötigt mehr Speicher pro Element als eine \texttt{ArrayList}, da für jeden
          Knoten zusätzliche Referenzen auf Vorgänger und Nachfolger gespeichert werden
          müssen.
\end{itemize}

\subsection{Beispiel}
\begin{lstlisting}[caption={Beispiel für die Verwendung einer LinkedList in Java}, label=lst:linkedListExample]
import java.util.LinkedList; // Import notwendig
import java.util.List;       // Interface verwenden

List<String> queueList = new LinkedList<>(); // Neue, leere LinkedList erstellen
// Für Queue/Deque-Operationen kann man auch direkt LinkedList oder Deque verwenden:
// Deque<String> queue = new LinkedList<>();

queueList.add("Erster"); // Element am Ende anfügen
((LinkedList<String>) queueList).addFirst("Neuer Erster"); // Spezifische LinkedList-Methode
((LinkedList<String>) queueList).offer("Letzter"); // Typisch für Queues (fügt am Ende hinzu)

System.out.println(queueList.get(1)); // Element am Index 1 ausgeben (Ausgabe: "Erster")
System.out.println(((LinkedList<String>) queueList).poll()); // Erstes Element abrufen und entfernen (Ausgabe: "Neuer Erster")
\end{lstlisting}

\subsection{Effizienz des Indexzugriffs bei LinkedLists (möglicherweise prüfungsrelevant)}
\label{subsec:LinkedListIndexAccess}
Der Zugriff auf ein Element an einem bestimmten Index (z.B. mittels \texttt{liste.get(index)}) ist bei einer \texttt{LinkedList} vergleichsweise langsam und hat eine Zeitkomplexität von $O(n)$ im schlechtesten und durchschnittlichen Fall. Der Grund hierfür liegt in der Struktur der verketteten Liste:
\begin{itemize}
    \item Anders als bei einem Array, wo die Speicheradresse eines Elements direkt aus
          dem Index berechnet werden kann, muss bei einer \texttt{LinkedList} die Kette
          von Verweisen vom ersten (oder letzten, je nach Implementierung und Index)
          Knoten bis zum gewünschten Knoten verfolgt werden.
    \item Jeder Knoten kennt nur seinen direkten Vorgänger und Nachfolger. Um zum i-ten
          Element zu gelangen, müssen i Schritte (oder n-i Schritte vom Ende) durchlaufen
          werden.
\end{itemize}
Die nachfolgende Abbildung~\ref{fig:linkedlistaccess} verdeutlicht diesen Prozess.

\begin{figure}[h!]
    \centering
    \begin{tikzpicture}[
            node distance=1.5cm and 1cm,
            listnode/.style={rectangle, draw, fill=blue!10, rounded corners, minimum height=1cm, minimum width=1.5cm, text width=1.3cm, align=center},
            arrow/.style={-Latex, thick},
            traversalarrow/.style={Latex-, thick, red, dashed, bend left=20, shorten >=2pt, shorten <=2pt}
        ]
        \node[listnode] (n0) {Index 0 \\ Wert A};
        \node[listnode, right=of n0] (n1) {Index 1 \\ Wert B};
        \node[listnode, right=of n1] (n2) {Index 2 \\ Wert C};
        \node[listnode, right=of n2] (n3) {Index 3 \\ Wert D};
        \node[right=0.5cm of n3] (dots) {\ldots};

        \draw[arrow] (n0.east) -- (n1.west);
        \draw[arrow] (n1.east) -- (n2.west);
        \draw[arrow] (n2.east) -- (n3.west);
        \draw[arrow] (n3.east) -- (dots.west);

        \node[above=0.3cm of n0, red] (start) {Start};
        \draw[traversalarrow] (n0.north) to (start.south);
        \path (n0.north east) edge[traversalarrow, bend left=30] node[above, midway, sloped, font=\tiny, red] {1. Sprung} (n1.north west);
        \path (n1.north east) edge[traversalarrow, bend left=30] node[above, midway, sloped, font=\tiny, red] {2. Sprung} (n2.north west);
        \node[above=0.3cm of n2, red] (target) {Ziel: Index 2};
        \draw[traversalarrow, bend left=0] (n2.north) to (target.south);

        \node[below=2cm of n1.south, text width=8cm, align=center] (captiontext)
        {Zugriff auf Index 2: Es müssen 2 Sprünge vom Start (Index 0) erfolgen, um das Element zu erreichen. Jeder Sprung folgt dem Verweis des aktuellen Knotens auf den nächsten Knoten.};

    \end{tikzpicture}
    \caption{Visualisierung des sequenziellen Zugriffs auf ein Element (hier Index 2) in einer LinkedList. Um zum Zielknoten zu gelangen, muss die Liste vom Anfangsknoten aus durchlaufen werden.}
    \label{fig:linkedlistaccess}
\end{figure}

Im Gegensatz dazu ist das sequentielle Durchlaufen aller Elemente einer
\texttt{LinkedList} mit einem Iterator (z.B. in einer for-each-Schleife)
effizient. Jeder Schritt des Iterators zum nächsten Element dauert $O(1)$,
sodass das Iterieren über die gesamte Liste eine Gesamtkomplexität von $O(n)$
hat. Problematisch ist also nicht das Iterieren an sich, sondern der wahlfreie
Zugriff per Index.

\section{HashSet}
Ein \texttt{HashSet} implementiert das \texttt{Set}-Interface und speichert
eine Sammlung von eindeutigen Elementen. Die Elemente in einem \texttt{HashSet}
haben keine garantierte Reihenfolge; die Iterationsreihenfolge kann sich sogar
ändern, wenn neue Elemente hinzugefügt werden. Die Implementierung basiert auf
einer Hashtabelle (intern wird oft eine \texttt{HashMap} verwendet, bei der die
Werte ignoriert werden).

\subsection{Anwendungsbereiche}
\texttt{HashSet}s eignen sich hervorragend, wenn:
\begin{itemize}
    \item Eindeutige Elemente gespeichert werden sollen und die Reihenfolge keine Rolle
          spielt.
    \item Schnell geprüft werden muss, ob ein Element bereits in der Sammlung vorhanden
          ist.
    \item Duplikate automatisch verhindert werden sollen.
\end{itemize}

\subsection{Vorteile}
\begin{itemize}
    \item Sehr schnelle durchschnittliche Zeitkomplexität von $O(1)$ (amortisiert) für
          die Operationen \texttt{add}, \texttt{remove} und \texttt{contains}.
    \item Effiziente Vermeidung von Duplikaten.
\end{itemize}

\subsection{Nachteile}
\begin{itemize}
    \item Keine garantierte Reihenfolge der Elemente. Die Reihenfolge bei der Iteration
          ist nicht vorhersagbar.
    \item Die Leistung kann bei einer schlechten Implementierung der
          \texttt{hashCode()}-Methode der gespeicherten Objekte oder bei einer sehr hohen
          Füllrate der internen Hashtabelle (viele Kollisionen) im schlechtesten Fall auf
          $O(n)$ für einzelne Operationen abfallen.
\end{itemize}

\subsection{Beispiel}
\begin{lstlisting}[caption={Beispiel für die Verwendung eines HashSet in Java}, label=lst:hashSetExample]
import java.util.HashSet;
import java.util.Set; // Interface verwenden

Set<String> uniqueNames = new HashSet<>();
uniqueNames.add("Alice");
uniqueNames.add("Bob");
uniqueNames.add("Alice"); // Wird ignoriert, da Duplikat (add() gibt false zurück)

System.out.println(uniqueNames.contains("Alice")); // true
System.out.println(uniqueNames.size()); // 2

for (String name : uniqueNames) {
    System.out.println(name); // Reihenfolge nicht garantiert, z.B. Bob, Alice
}
\end{lstlisting}

\section{TreeSet}
Ein \texttt{TreeSet} implementiert das \texttt{SortedSet}-Interface (welches
\texttt{Set} erweitert) und speichert eine Sammlung von eindeutigen Elementen
in sortierter Reihenfolge. Die Sortierung erfolgt entweder natürlich (wenn die
Elemente \texttt{Comparable} implementieren) oder durch einen beim Erstellen
des \texttt{TreeSet}s übergebenen \texttt{Comparator}. Die Implementierung
basiert typischerweise auf einem Rot-Schwarz-Baum.

\subsection{Anwendungsbereiche}
\texttt{TreeSet}s sind nützlich, wenn:
\begin{itemize}
    \item Eindeutige Elemente in einer sortierten Reihenfolge gespeichert und abgerufen
          werden müssen.
    \item Operationen wie das Finden des kleinsten/größten Elements oder das Abrufen von
          Teilmengen (Ranges) basierend auf der Sortierung benötigt werden.
\end{itemize}

\subsection{Vorteile}
\begin{itemize}
    \item Elemente werden automatisch sortiert gehalten.
    \item Effiziente Operationen ($O(\log n)$) für \texttt{add}, \texttt{remove} und
          \texttt{contains}.
    \item Ermöglicht den Zugriff auf das erste/letzte Element sowie auf Teilmengen in
          $O(\log n)$ oder $O(1)$ (für erste/letzte).
\end{itemize}

\subsection{Nachteile}
\begin{itemize}
    \item Langsamer als \texttt{HashSet} für die grundlegenden Operationen \texttt{add},
          \texttt{remove} und \texttt{contains}, da die Sortierreihenfolge
          aufrechterhalten werden muss.
    \item Erfordert, dass Elemente entweder \texttt{Comparable} implementieren oder ein
          \texttt{Comparator} bereitgestellt wird. \texttt{null}-Elemente sind
          standardmäßig nicht erlaubt.
\end{itemize}

\subsection{Beispiel}
\begin{lstlisting}[caption={Beispiel für die Verwendung eines TreeSet in Java}, label=lst:treeSetExample]
import java.util.TreeSet;
import java.util.Set; // Interface verwenden
import java.util.SortedSet; // Spezifischeres Interface

SortedSet<Integer> sortedNumbers = new TreeSet<>();
sortedNumbers.add(50);
sortedNumbers.add(10);
sortedNumbers.add(90);
sortedNumbers.add(10); // Wird ignoriert, da Duplikat

// Iteration erfolgt in sortierter Reihenfolge
for (Integer number : sortedNumbers) {
    System.out.println(number); // Ausgabe: 10, 50, 90
}

System.out.println(sortedNumbers.first()); // 10
System.out.println(sortedNumbers.last());  // 90
// System.out.println(sortedNumbers.higher(50)); // gäbe es nicht für SortedSet, nur für NavigableSet/TreeSet
System.out.println(((TreeSet<Integer>)sortedNumbers).higher(50)); // 90 (Casting auf TreeSet)
\end{lstlisting}

\section{HashMap}
Eine \texttt{HashMap} implementiert das \texttt{Map}-Interface und speichert
Schlüssel-Wert-Paare. Jedem Schlüssel kann höchstens ein Wert zugeordnet sein.
\texttt{HashMap} erlaubt \texttt{null}-Werte und einen einzelnen
\texttt{null}-Schlüssel. Die Reihenfolge der Einträge in einer \texttt{HashMap}
ist nicht garantiert und kann sich ändern. Die Implementierung basiert auf
einer Hashtabelle.

\subsection{Anwendungsbereiche}
\texttt{HashMap}s sind die Standardwahl für Map-Implementierungen, wenn:
\begin{itemize}
    \item Daten als Schlüssel-Wert-Paare gespeichert werden sollen.
    \item Ein schneller Zugriff auf Werte über ihre Schlüssel erforderlich ist.
    \item Die Reihenfolge der Einträge keine Rolle spielt.
\end{itemize}

\subsection{Vorteile}
\begin{itemize}
    \item Sehr schnelle durchschnittliche Zeitkomplexität von $O(1)$ (amortisiert) für
          die Operationen \texttt{put}, \texttt{get}, \texttt{remove} und
          \texttt{containsKey}.
    \item Flexibel durch die Erlaubnis eines \texttt{null}-Schlüssels und beliebig vieler
          \texttt{null}-Werte.
\end{itemize}

\subsection{Nachteile}
\begin{itemize}
    \item Keine garantierte Reihenfolge der Schlüssel oder Werte bei der Iteration.
    \item Ähnlich wie bei \texttt{HashSet} kann die Leistung bei schlechten \texttt{hashCode()}-Implementierungen der Schlüssel oder hoher Füllrate auf $O(n)$ für einzelne Operationen abfallen.
\end{itemize}

\subsection{Beispiel}
\begin{lstlisting}[caption={Beispiel für die Verwendung einer HashMap in Java}, label=lst:hashMapExample]
import java.util.HashMap;
import java.util.Map; // Interface verwenden

Map<String, Integer> ageMap = new HashMap<>();
ageMap.put("Alice", 30);
ageMap.put("Bob", 25);
ageMap.put("Charlie", 35);
ageMap.put("Alice", 32); // Wert für "Alice" wird aktualisiert

System.out.println(ageMap.get("Bob")); // 25
System.out.println(ageMap.containsKey("David")); // false
ageMap.put(null, 40); // Null-Schlüssel ist erlaubt
ageMap.put("Eve", null); // Null-Wert ist erlaubt

for (Map.Entry<String, Integer> entry : ageMap.entrySet()) {
    System.out.println(entry.getKey() + ": " + entry.getValue()); // Reihenfolge nicht garantiert
}
\end{lstlisting}

\subsection{Prüfungshinweis}
Dass die HashMap (und HashCode) Effizient funktioniert, muss die
\lstinline{hashcode} Methode (und damit die \lstinline{equals} Methode)
sinnvoll überschrieben werden.

\section{TreeMap}
Eine \texttt{TreeMap} implementiert das \texttt{SortedMap}-Interface (welches
\texttt{Map} erweitert) und speichert Schlüssel-Wert-Paare in sortierter
Reihenfolge der Schlüssel. Die Sortierung der Schlüssel erfolgt entweder
natürlich (wenn die Schlüssel \texttt{Comparable} implementieren) oder durch
einen beim Erstellen der \texttt{TreeMap} übergebenen \texttt{Comparator}.
\texttt{TreeMap} erlaubt keine \texttt{null}-Schlüssel (im Gegensatz zu
\texttt{HashMap}), aber \texttt{null}-Werte sind erlaubt. Die Implementierung
basiert typischerweise auf einem Rot-Schwarz-Baum.

\subsection{Anwendungsbereiche}
\texttt{TreeMap}s werden verwendet, wenn:
\begin{itemize}
    \item Schlüssel-Wert-Paare gespeichert werden müssen und eine Iteration über die
          Schlüssel in sortierter Reihenfolge erforderlich ist.
    \item Operationen benötigt werden, die von der Sortierung der Schlüssel abhängen, wie
          das Finden des Eintrags mit dem kleinsten/größten Schlüssel oder das Abrufen
          von Teilmengen basierend auf Schlüsselbereichen.
\end{itemize}

\subsection{Vorteile}
\begin{itemize}
    \item Schlüssel werden automatisch sortiert gehalten.
    \item Effiziente Operationen ($O(\log n)$) für \texttt{put}, \texttt{get},
          \texttt{remove} und \texttt{containsKey}.
    \item Ermöglicht den Zugriff auf den ersten/letzten Eintrag sowie auf Teil-Maps
          (Ranges) in $O(\log n)$ oder $O(1)$.
\end{itemize}

\subsection{Nachteile}
\begin{itemize}
    \item Langsamer als \texttt{HashMap} für die grundlegenden Operationen, da die
          Sortierreihenfolge der Schlüssel aufrechterhalten werden muss.
    \item Erfordert, dass Schlüssel entweder \texttt{Comparable} implementieren oder ein
          \texttt{Comparator} bereitgestellt wird.
    \item Erlaubt keine \texttt{null}-Schlüssel (wirft \texttt{NullPointerException}).
\end{itemize}

\subsection{Beispiel}
\begin{lstlisting}[caption={Beispiel für die Verwendung einer TreeMap in Java}, label=lst:treeMapExample]
import java.util.TreeMap;
import java.util.Map; // Interface verwenden
import java.util.SortedMap; // Spezifischeres Interface

SortedMap<Integer, String> sortedUserMap = new TreeMap<>();
// Alternativ: Map<Integer, String> sortedUserMap = new TreeMap<>();

sortedUserMap.put(102, "Alice");
sortedUserMap.put(100, "Bob");
sortedUserMap.put(105, "Charlie");
// sortedUserMap.put(null, "David"); // Würde NullPointerException werfen

// Iteration erfolgt in sortierter Reihenfolge der Schlüssel
for (Map.Entry<Integer, String> entry : sortedUserMap.entrySet()) {
    System.out.println(entry.getKey() + ": " + entry.getValue()); // 100: Bob, 102: Alice, 105: Charlie
}

System.out.println(sortedUserMap.firstKey()); // 100
// Für subMap etc. ist SortedMap ausreichend
SortedMap<Integer, String> subMap = sortedUserMap.subMap(100, 103); // Einträge für 100 (inkl.), 102 (inkl.), 103 (exkl.)
System.out.println("SubMap: " + subMap);
\end{lstlisting}

\section{Über Collections Iterieren (Prüfungshinweis)}
Über Collections, abgesehen von der \lstinline{ArrayList}, sollte \textit{immer} mittels
Iterator iteriert werden. Andere lösungen mit einer Zähl Variable und mehreren
\lstinline{.get} aufrufen funktionieren zwar, sind aber in der Praxis sehr Ineffizient
und führen zu Punktabzug.