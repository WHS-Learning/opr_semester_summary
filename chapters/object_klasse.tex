\chapter{Klasse Object}
\label{chap:Klasse Object}

Die klasse Object ist eine besondere Klasse. \textit{jede} Klasse erbt von der
Object Klasse. In der Object Klasse sind einige Methoden definiert, welche jede
Klasse besitzt. Zu ihnen gehöhren beispielsweise
\lstinline{equals} und \lstinline{hashcode}

\subsection{equals und hashcode}

Die \lstinline{equals} und \lstinline{hashcode} Methoden haben einen Vertrag.
Wenn eine von beiden überschrieben wird, muss die andere auch überschrieben
werden. Wenn Objekte im sinne von
\lstinline{equals} gleich sind, \textit{müssen} diese auch den selben
\lstinline{hashcode} besitzen. Wenn objekte jedoch den selben \lstinline{hashcode}
haben, müssen sie nicht zwingend im sinne von \lstinline{equals} gleich sein.

Standardmäßig prüft die
\lstinline{equals} methode auf gleichheit der Identität, die \lstinline{hashcode}
methode gibt die memory addresse zurück.

\subsection{toString}

Das Objekt wird textuell ausgegeben. Diese Methode wird beispielsweise
aufgerufen, wenn ein Objet über ein \lstinline{System.out.println} in die
Konsole ausgegeben wird.

Die Klasse Objekt hat noch viele weitere Methoden implementiert, allerdings
sind keien weiteren Prüfungsrelevant.