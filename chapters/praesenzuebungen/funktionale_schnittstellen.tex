\chapter{Präsenzübung Funktionale Schnittstellen}

\section{Aufgabe 1}

\begin{enumerate}
    \item Überlegen Sie, durch welche Schnittstelle aus dem Paket java.util.function man die
Schnittstelle Filter aus dem Codebeispiel dieses Kapitels ersetzen kann.
Verändern Sie die Klasse FilteredSequence entsprechend.
\item Schreiben Sie einen Ausdruck, um ein Objekt der Klasse FilteredSequence zu erzeugen,
das aus einer zugrunde liegenden StringSequence alle Elemente ausfiltert, die eine unge-
radzahlige Länge haben (sodass die Strings mit geradzahliger Länge durchkommen). Nehmen
Sie an, dass sich die zugrunde liegende StringSequence in der Variablen seq befindet.
\end{enumerate}

\section{Aufgabe 2}

Schreiben Sie einen Ausdruck, um ein TreeSet<String> zu erzeugen, sodass die Elemente bei
der Iteration über die Menge in lexikographisch aufsteigender Reihenfolge durchlaufen werden. Da-
bei soll jedoch nicht zwischen Groß- und Kleinschreibung unterschieden werden, und Leerzeichen
zwischen den Wörtern der Strings sollen ignoriert werden.

Beispiele:

\begin{itemize}
    \item Haus kommt vor hier
    \item haus kommt vor Hier
    \item hier kommt vor hi hi
\end{itemize}