\chapter{Präsenzübung Streams}

\section{Aufgabe 1}

Es sei s eine Variable des Typs Stream<String>.
Schreiben Sie einen Ausdruck für den Stream aller Wörter von s, solange diese eine Länge ungleich
100 haben. Der berechnete Stream endet somit mit demjenigen Wort aus s, auf das das erste Mal
ein Wort der Länge 100 folgt.
Schauen Sie sich dazu die Methode takeWhile in der Schnittstelle Stream an.

\section{Aufgabe 2}

Schreiben Sie einen Ausdruck des Typs LongStream, der den Stream der Zahlen 0, 1, 3, 7, 15, 31,
... berechnet. Der Stream soll nur die positiven Zahlen enthalten.

\section{Aufgabe 3}

Die Methode count der Schnittstelle Stream berechnet die Anzahl der Elemente eines Streams.
Schreiben Sie einen Ausdruck für die gleiche Berechnung (ohne count zu verwenden!).

\section{Aufgabe 4}

Die Folge $r: \mathbb{N}_0 \rightarrow \mathbb{R}$ konvergiert für $x \geq 0$ gegen $\sqrt{x}$. Sie kann als Grundlage für einen Algorithmus
verwendet werden, um Quadratwurzeln näherungsweise zu berechnen. $r$ ist wie folgt definiert:

\begin{align*}
    r_0 = 1 \\
    r_{n + 1} = \frac{1}{2} \cdot \left(r_n + \frac{x}{r_n}\right) \text{ für } n \geq 0
\end{align*}

Ergänzen Sie die folgende Methode sqrt zur näherungsweisen Berechnung der Quadratwurzel. Die
Berechnung der Wurzel von x soll enden, wenn sich das Quadrat des Näherungswerts nur noch um
weniger als 0.00001 von x unterscheidet.

\begin{lstlisting}
public static OptionalDouble sqrt(double x) {
return x < 0
? OptionalDouble.empty()
:
}
\end{lstlisting}