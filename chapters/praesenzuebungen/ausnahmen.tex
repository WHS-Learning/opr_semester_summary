\chapter{Präsenzübung Ausnahmen}

\section{Aufgabe 1}
In der Anwendung Staff gibt es in der Klasse Employee die Methode void setGeneralOrderLimit(int),
mit der ein Bestelllimit für alle Mitarbeiter gesetzt wird. Ein negativer Parameterwert ergibt für diese
Methode keinen Sinn.
Beschreiben Sie, wie man diese Methode ändern muss, sodass sie einem Aufrufer im Falle eines
negativen Parameterwerts signalisiert, dass sie mit einem unsinnigen Wert aufgerufen wurde.

\begin{lstlisting}
public void setGeneralOrderLimit(int limit) {
    if (limit < 0) {
        throw new IllegalArgumentException("limit has to be at least 1.");
    }
    ...
}
\end{lstlisting}

\section{Aufgabe 2}
Realisieren Sie eine Methode Number parse(String), die ein Hüllobjekt der Klasse Integer
liefert, wenn die Zeichenkette eine Zahl des Typs int repräsentiert. Wenn die Zeichenkette keinen
int-Wert, jedoch einen double-Wert repräsentiert, liefert die Methode ein entsprechendes Hüllobjekt der Klasse Double. Sonst wirft die Methode eine NumberFormatException.
In der Klasse Integer gibt es die statische Methode int parseInt(String), in Double die Methode double parseDouble(String). Beide Klassen sind Unterklassen von java.lang.Number.

\begin{lstlisting}
public Number parse(String s) {
    try {
        return Integer.parseInt(s);
    } catch (NumberFormatException e) {
        return Double.parseDouble(s);
    }
}
\end{lstlisting}