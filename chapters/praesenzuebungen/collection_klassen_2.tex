\chapter{Präsenzübung Collection \newline Klassen 2}

\section{Aufgabe 1}
Bei einer Umfrage gibt es für eine bestimmte Frage n vordefinierte
Antwortmöglichkeiten. Jede Kombination dieser Antwortmöglichkeiten ist
zulässig, einschließlich dem Ankreuzen keiner und dem Ankreuzen aller
Möglichkeiten. Eine Antwort auf die Frage wird durch ein Objekt der Klasse Vote
repräsentiert.

\begin{lstlisting}
public class Vote {
  private final int[] selectedOptions;

  public Vote(int... selectedOptions) {
    this.selectedOptions = selectedOptions;
  }
    ...
}
\end{lstlisting}

Die Klasse Evaluation dient der Auswertung der Antworten auf eine Frage. Beim
Erzeugen eines Evaluation-Objekts übergibt man eine Liste aller Antworten auf
die Frage. Ergänzen Sie die Klasse Evaluation (und gerne auch Vote), sodass
sich die Methode percentageOfVotes wie beschrieben verhält. Die Methode soll
keine Iteration enthalten.

\begin{lstlisting}
public class Evaluation {
  public Evaluation(LinkedList<Vote> votes) {
    ...
  }

  /**
   * Returns how often the given vote occurs among all votes.
   *
   * @return the relative quantity in percent
   */
  public double percentageOfVotes(Vote v) {
    ...
  }
}
\end{lstlisting}

\begin{lstlisting}
public class Vote {
  private final int[] selectedOptions;

  public Vote(int... selectedOptions) {
    this.selectedOptions = selectedOptions;
  }

  @Overrite
  public boolean equals(Object obj) {
    if (obj instanceof Vote v) {
        return Arrays.equals(v.selectedOptions, this.selectedOptions);
    }
    return false;
  }

  @Overrite
  public int hashCode() {
    int sum = 0;
    for (int v : selectedOptions) {
        sum += v;
    }
    return sum;
  }
}

public class Evaluation {
  private final LinkedList<Vote> votes;
  HashMap<Vote, Integer> votePercentages;

  public Evaluation(LinkedList<Vote> votes) {
    votes = new LinkedList<>();
    votePercentages = new HashMap<>();
    for (Vote v : votes) {
        votes.add(v);
        int amount = votePercentages.getOrDefault(v, 0);
        votePercentages.put(v, amount + 1);
    }
  }

  /**
   * Returns how often the given vote occurs among all votes.
   *
   * @return the relative quantity in percent
   */
  public double percentageOfVotes(Vote v) {
    return votePercentages.getOrDefault(v, 0) / votes.size();
  }
}
\end{lstlisting}