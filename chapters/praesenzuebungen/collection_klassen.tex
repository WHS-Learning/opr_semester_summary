\chapter{Präsenzübung Collection \newline Klassen 1}

\section{Aufgabe 1}
Die CatalogItem-Objekte des Code-Beispiels aus der Vorlesung sollen eine
(alphanummerische) Artikelnummer erhalten und es soll möglich sein, mit der
Methode CatalogItem getItem(String) einen Artikel des Katalogs gezielt anhand
seiner Nummer auszuwählen. Skizzieren Sie die notwendigen Änderungen an den
Klassen des Katalogbeispiels.

\begin{lstlisting}
public abstract class CatalogItem {
    ...
    private final String id;

    protected CatalogItem(float price, String id) {
        this.id = id;
        ...
    }

    public String getId() {
        return this.id;
    }
}

public class Catalog {
    CatalogItem getItem(String id) {
        for (CatalogItem item : items) {
            if (item.getId().equals(id)) {
                return item;
            }
        }

        throw new NoSuchElementException("No item found with ID " + id);
    }
}
\end{lstlisting}

\section{Aufgabe 2}
In der Klasse CatalogItem des Code-Beispiels aus der Vorlesung werden die
Ausschlusswörter in einem Feld (Array) verwaltet. Mit unserem jetzigen
Kenntnisstand ist das keine gute Wahl. Welche Klasse ist geeigneter, um die
Ausschlusswörter zu verwalten? Was ändert sich dadurch am Quellcode der Methode
\lstinline{boolean isSkipWord(String)}?

\textbf{Die Ausschlusswörter sollten in einem \lstinline{HashSet} verwaltet werden. Um zu überprüfen, ob ein Wort ein SkipWord ist, kann dann die \lstinline{contains()} Methode verwendet werden.}

\section{Aufgabe 3}
Die Klasse Address besitzt Instanzvariablen street und city vom Typ String.
Realisieren Sie in Address eine statische Methode 
\lstinline{... streetsByTown(ArrayList<Address>)} die die Adressen der übergebenen Liste als
Zuordnung liefert, bei der jeder Stadt die Straßen dieser Stadt zugeordnet
sind. Wenn man über die Straßen der einzelnen Städte iteriert, sollen sie
alphabe- tisch sortiert durchlaufen werden. Wählen Sie zuerst einen geeigneten
Ergebnistyp der Methode.

\begin{lstlisting}
HashMap<String, TreeSet<String>> streetsByTown(ArrayList<Address> addresses) {
    HashMap<String, TreeSet<String>> streets = new HashMap<>();

    for (Address address : addresses) {
        TreeSet<String> st = streets.getOrDefault(address.city, new TreeSet<String>());
        st.add(address.street);
        streets.put(adress.city, st);
    }

    return streets;
}
\end{lstlisting}
