\chapter{Präsenzübung Collection Klassen 1}

\section{Aufgabe 1}
Die CatalogItem-Objekte des Code-Beispiels aus der Vorlesung sollen eine (alphanummerische)
Artikelnummer erhalten und es soll möglich sein, mit der Methode CatalogItem getItem(String)
einen Artikel des Katalogs gezielt anhand seiner Nummer auszuwählen.
Skizzieren Sie die notwendigen Änderungen an den Klassen des Katalogbeispiels.

\section{Aufgabe 2}
In der Klasse CatalogItem des Code-Beispiels aus der Vorlesung werden die Ausschlusswörter
in einem Feld (Array) verwaltet. Mit unserem jetzigen Kenntnisstand ist das keine gute Wahl. Wel-
che Klasse ist geeigneter, um die Ausschlusswörter zu verwalten? Was ändert sich dadurch am
Quellcode der Methode boolean isSkipWord(String)?

\section{Aufgabe 3}
Die Klasse Address besitzt Instanzvariablen street und city vom Typ String.
Realisieren Sie in Address eine statische Methode
... streetsByTown(ArrayList<Address>)
die die Adressen der übergebenen Liste als Zuordnung liefert, bei der jeder Stadt die Straßen dieser
Stadt zugeordnet sind. Wenn man über die Straßen der einzelnen Städte iteriert, sollen sie alphabe-
tisch sortiert durchlaufen werden.
Wählen Sie zuerst einen geeigneten Ergebnistyp der Methode.

