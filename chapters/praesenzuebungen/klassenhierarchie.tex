\chapter{Präsenzübung Klassenhierarchie}

Ausgangspunkt dieser Aufgabe sind folgende Klassen A und B.

\begin{lstlisting}
public class B {
  private double d;
  public void setD(double v) {
    d = v;
  }
  public String asText() {
    return String.valueOf(d);
  }
}

public class A {
  private int n;
  public A(int n) {
    this.n = n;
  }
  public double m(int m) {
    return (double) n / m;
  }
}
\end{lstlisting}

\begin{enumerate}
    \item Ergänzen Sie den Quellcode, sodass A direkte Unterklasse von B ist.
    \item Ist nun folgendes Codestück compilierbar? \lstinline{B b = new A(n);}
    \item Ist dieses Codestück compilierbar? Wenn ja, was ist die Ausgabe? \newline
    \begin{lstlisting}
A a = new A(10);
a.setD(5);
System.out.println(a.asText());
\end{lstlisting}
\item Implementieren Sie eine Methode \lstinline{getD()}, die den Wert der Instanzvariablen d liefert. Die
Methode soll aus allen Unterklassen von B aufrufbar sein, nicht jedoch aus allen Klassen überhaupt.
\item Implementieren Sie in A eine Methode \lstinline{asText()}, die den gleich Wert liefert wie die gleichnamige Methode aus B, jedoch ergänzt um den Wert der Instanzvariablen n. Dies soll auch dann
gelten, wenn die Implementierung von \lstinline{asText} in B geändert wird.
\item Deklarieren Sie die Methode m so, dass sie in Unterklassen nicht überschrieben werden kann.
\end{enumerate}