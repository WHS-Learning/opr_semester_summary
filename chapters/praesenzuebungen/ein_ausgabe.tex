\chapter{Präsenzübung Ein-/Ausgabe}

\section{Aufgabe 1}

\begin{enumerate}
    \item Realisieren Sie eine Methode \lstinline{void onlyNumbers(Reader, Writer)}, die
          alle nicht-leeren Zeilen der übergebenen Datenquelle, die ausschließlich aus
          Ziffern bestehen, in die Datensenke schreibt. Hinweis: Sie können durch
          \lstinline{s.matches([0-9]+)} prüfen, ob die Zeichenkette in s nicht leer ist
          und nur aus Ziffern besteht.
          \begin{lstlisting}
public void onlyNumbers(Reader reader, Writer writer) {
    BufferedReader br = new BufferedReader(reader);
    BufferedWriter bw = new BufferedWriter(writer);

    String line = br.readLine();

    while (line != null && line.matches("[0-9]+")) throws IOException{
        bw.write(line);
        line = br.readLine();
    }
    br.close();
    bw.close();
}
\end{lstlisting}
    \item Von welchen Klassen erzeugen Sie Objekte für JUnit-Tests der Methode
          onlyNumbers? \newline \textbf{von StringReader und StringWriter}
    \item Wird die Methode immer terminieren? \newline \textbf{Nein. Reader können
              unendlich lange Daten beinhalten.}
\end{enumerate}

\pagebreak

\section{Aufgabe 2}
\begin{enumerate}
    \item Realisieren Sie eine Methode long lengthOfPart(Path, int), die die Datei zum
          angegebenen Pfad byte-orientiert durchläuft und zurückgibt, wieviele Bytes
          die Datei enthält bis zur Position, an der erstmals der übergebene Bytewert
          vorkommt. Diese Position soll nicht mitgezählt werden.
          \begin{lstlisting}
public long lengthOfPart(Path path, int i) {
    File f = path.toFile();
    FileInputStream fis = new FileInputStream(f);
    BufferedInputStream bis = new BufferedInputStream(fis);

    int b = bis.read();
    
    int size = 0;

    while (b != -1 && b != i) {
        size++;
        b = bis.read();
    }

    bis.close();
    fis.close();

    return size();
}
\end{lstlisting}
    \item Ist der Parametertyp der Methode gut gewählt, um die Methode mit JUnit zu
          testen? \newline 
          \textbf{Nein, Teste sollten in sich geschlossen sein und nicht auf externe Daten basieren.}
    \item Überprüfen Sie, ob die Methode auch für größere Dateien effizient arbeitet. Ändern Sie ggfs.
          die Realisierung. \newline \textbf{Ja funktioniert gut wegen des BufferedInputStreams.}
\end{enumerate}