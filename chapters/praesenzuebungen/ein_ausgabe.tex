\chapter{Präsenzübung Ein-/Ausgabe}

\section{Aufgabe 1}

\begin{enumerate}
    \item Realisieren Sie eine Methode void onlyNumbers(Reader, Writer), die alle nicht-leeren
Zeilen der übergebenen Datenquelle, die ausschließlich aus Ziffern bestehen, in die Datensen-
ke schreibt.
Hinweis: Sie können durch s.matches([0-9]+) prüfen, ob die Zeichenkette in s nicht leer
ist und nur aus Ziffern besteht.
\item Von welchen Klassen erzeugen Sie Objekte für JUnit-Tests der Methode onlyNumbers?
\item Wird die Methode immer terminieren?
\end{enumerate}

\section{Aufgabe 2}
\begin{enumerate}
    \item Realisieren Sie eine Methode long lengthOfPart(Path, int), die die Datei zum ange-
gebenen Pfad byte-orientiert durchläuft und zurückgibt, wieviele Bytes die Datei enthält bis
zur Position, an der erstmals der übergebene Bytewert vorkommt. Diese Position soll nicht
mitgezählt werden.
\item Ist der Parametertyp der Methode gut gewählt, um die Methode mit JUnit zu testen?
\item Überprüfen Sie, ob die Methode auch für größere Dateien effizient arbeitet. Ändern Sie ggfs.
die Realisierung.
\end{enumerate}