\chapter{Präsenzübung Schnittstellen}

\section{Aufgabe 1}

Das Carsharing-Unternehmen, für das in der praktischen Aufgabe Car Sharing die
Klasse VehicleManager entwickelt wurde, möchte die Anwendung um eine
Protokollfunktion erweitern. Die konkrete Form des Protokolls ist noch unklar
und sie soll sich im Laufe der Zeit ändern können. Fest steht jedoch, dass sich
die Protokollierung auf Buchungsversuche (Methode bookVehicle) bezieht und auf
folgenden Informationen beruhen soll: Name des Fahrzeugs, das versucht wurde zu
buchen; gewünschter Zeitraum; Kennzeichen, ob der Buchungsversuch erfolgreich
war. Beschreiben Sie, was zusätzlich benötigt wird (Klassen und/oder
Schnittstellen), um die Anforderung zu erfüllen und was an der Klasse
VehicleManager geändert werden muss. Als Protokoll interessiert sich das
Carsharing-Unternehmen anfangs für den Anteil der erfolgreichen Buchungen in
Bezug auf alle Buchungsversuche. Beschreiben Sie, was dafür zu tun ist.

\begin{lstlisting}
public interface Protocol {
    void log(String name, TimeInterval ti, boolean successful);
}

public class SuccessProtocol implements Protocol{
    private int successful;
    private int general;
    
    public SuccessProtocol() {
        this.successful = 0;
        this.general = 0;
    }

    @Override
    public void log(String name, TimeInterval ti, boolean successful) {
        general++;
        if (successful) {
            this.successful++;
        }
    }
}

public class VehicleManager {
    Protocol protocol;

    ...

    public VehicleManager() {
        this.protocol = new SuccessProtocol();
    }

    public boolean bookVehicle(String name, String from, String to) {
        TimeInterval ti = new TimeInterval(from, to)
        boolean success = vehicles.get(name).book(TimeInterval, from);
        this.protocol.log(name, ti, success);
    }
}
\end{lstlisting}