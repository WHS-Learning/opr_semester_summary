\chapter{Schnittstellen (Prüfungsrelevant)}

Schnittstellen geben ein Versprechen, dass Klassen bestimmte Fähigkeiten besitzen, ohne vorzugeben, wie sich eine Klasse zu verhalten hat. Sie definieren, welche Methoden eine Klasse implementieren muss.

\begin{lstlisting}[language=Java, caption={Beispiel einer Schnittstelle}]

public interface Salable {
    boolean matches(String searchPattern);
    String asText();
    float getPrice();
}
\end{lstlisting}

\section{Eigenschaften und Hinweise}
\begin{itemize}
    \item Schnittstellen können \textbf{nicht} \lstinline{protected} oder \lstinline{private} sein.
    \item Es gibt keine oberste Schnittstelle, analog zur Klasse \lstinline{Object}.
    \item In Schnittstellen können \lstinline{public static final}-Variablen (Konstanten) deklariert werden.
    \item Schnittstellen können \lstinline{static} Methoden besitzen, die keinen Bezug zu Objekten haben.
    \item Wenn eine Klasse eine Schnittstelle implementiert (\lstinline{implements}), muss sie entweder \texttt{abstract} sein oder alle nicht-statischen Methoden der Schnittstelle definieren.
\end{itemize}

\section{Schnittstellen als Typ}
Schnittstellen können als Typdeklaration und als Ergebnistyp von Methoden verwendet werden. Dies ermöglicht die Angabe eines Objekts jeder Klasse, die die Schnittstelle implementiert.
\begin{lstlisting}[language=Java, caption={Schnittstelle als Typ}]
Salable item;

public void add(Salable item) {
    ...
}
\end{lstlisting}

\section{Funktionale Schnittstellen}
Eine Schnittstelle ist funktional, wenn sie genau eine nicht-statische Methode definiert. Sie kann optional mit der Annotation \texttt{@FunctionalInterface} versehen werden.

\section{Lambda-Ausdrücke (sehr sehr Prüfungsrelevant)}
Lambda-Ausdrücke sind namenlose Funktionen, die über die Syntax \texttt{(lambdaParameter) -> { Anweisung }} definiert werden.

\subsection{Vereinfachung von Lambda-Ausdrücken}
Lambda-Ausdrücke können vereinfacht werden:
\begin{itemize}
    \item Der Typ der Parameter kann weggelassen werden, da er aus dem Kontext hervorgeht.
    \item Geschweifte Klammern können weggelassen werden, wenn der Ausdruck nur eine Anweisung enthält.
    \item Runde Klammern können weggelassen werden, wenn der Ausdruck nur einen Parameter hat.
\end{itemize}
\begin{lstlisting}[language=Java, caption={Vereinfachung von Lambda ausdrücken}]
(String s) -> { return s.length() >= 4; } // Ursprünglich

(s) -> { return s.length() >= 4; }       // Parameter Typen weglassen
(s) -> s.length() >= 4;                  // Geschweifte Klammern weglassen
s -> s.length() >= 4;                    // Runde Klammern weglassen
\end{lstlisting}
Wenn in einem Lambda-Ausdruck nur eine Methode aufgerufen wird, kann eine Methodenreferenz verwendet werden:
\begin{lstlisting}[language=Java, caption={Methodenreferenz}]
s -> s.isEmpty();       // Lambda Ausdruck
String::isEmpty;        // Methodenreferenz
\end{lstlisting}

\subsection{Hinweise zu Lambda-Ausdrücken}
Lambda-Ausdrücke dürfen lokale Variablen nicht verändern, können aber deren Werte auslesen. Objekte können über Methodenaufrufe bearbeitet werden.