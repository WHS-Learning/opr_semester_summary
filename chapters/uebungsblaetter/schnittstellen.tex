\chapter{Übung Schnittstellen}

\section{Aufgabe 1}
Ist folgendes Code-Stück korrekt? Was ist ggfs.\ der Compile- oder
Laufzeitfehler?

\begin{lstlisting}
public class C {
  private Map<String, String> eineMap;

  public C() {
    eineMap = new HashMap<>();
  }

  public HashMap<String, String> m(String s) {
    eineMap.put(s, s);
    return eineMap;
  }
}
\end{lstlisting}

Das Code-Stück führt zu einem Compile-Fehler. Der Rückgabewert der Methode
\lstinline{m} ist eine \lstinline{HashMap}, die Instanzvariable
\lstinline{eineMap} ist allerdings eine \lstinline{Map}. Eine
\lstinline{HashMap} ist zwar eine \lstinline{Map}, aber eine
\lstinline{Map} ist nicht zwingend \lstinline{HashMap}

\pagebreak

\section{Aufgabe 2}

Gegeben seien folgende Klassen:

\begin{lstlisting}
public class Kreis {
  private double radius;

  public Kreis(double radius) {
    this.radius = radius;
  }

  public double gibUmfang() {
    return 2 * radius * Math.PI;
  }
}

public class Rechteck {
  private double laenge;
  private double breite;

  public Rechteck(double laenge, double breite) {
    this.laenge = laenge;
    this.breite = breite;
  }
}
\end{lstlisting}

Definieren Sie eine Schnittstelle Geo und ergänzen Sie die oben stehenden
Klassen, sodass folgender Programmcode compiliert und mit dem gewünschten
Verhalten ausgeführt werden kann.

\begin{lstlisting}
Geo g = new Kreis(2.0);
g = new Rechteck(2.0, 1.0);

// Figur in alle Richtungen um Faktor 3 vergrößern
g.skaliere(3);

// erwartete Ausgabe: Umfang = 18.0
System.out.println("Umfang = " + g.gibUmfang())
\end{lstlisting}

\begin{lstlisting}
public interface Geo {
    void skaliere(int i);
    double gibUmfang();
}

public class Kreis implements Geo{
  private double radius;

  public Kreis(double radius) {
    this.radius = radius;
  }

  public double gibUmfang() {
    return 2 * radius * Math.PI;
  }

  public void skaliere(int i) {
    this.radius *= i
  } 
}

public class Rechteck implements Geo{
  private double laenge;
  private double breite;

  public Rechteck(double laenge, double breite) {
    this.laenge = laenge;
    this.breite = breite;
  }
  public double gibUmfang() {
    return 2 * laenge + 2 * breite;
  }

  public void skaliere(int i) {
    this.laenge *= i;
    this.breite *= i;
  }
}
\end{lstlisting}

\section{Aufgabe 3}
Realisieren Sie eine auf JUnit basierende Testmethode für die Methode
getHierarchy aus der Übung zu Collection-Klassen. Da wir mittlerweile
Schnittstellen kennengelernt haben, gehen wir von folgender Definition der
Methode aus:

\begin{lstlisting}
/**
* Liefert aus den übergebenen Employees und allen 
* Superiors, auf die darüber zugegriffen werden 
* kann, eine Personalhierarchie. In der 
* Personalhierarchie sind alle Personen 
* enthalten, die oben genannt sind, und jeder 
* Person sind ihre direkt untergebenen Personen  
* zugeordnet.
*
* @param employees Mitarbeiter, von denen ausgehend 
* die Hierarchieinformation ermittelt wird.
*
* @return die Personalhierarchie basierend auf den 
* übergebenen Mitarbeitern. Die Schlüssel
* der Zuordnung sind alle direkten und indirekten 
* Vorgesetzten dieser Mitarbeiter.
* Jedem Vorgesetzten sind als Wert die direkten 
* Untergebenen zugeordnet.
*/
Map<Superior, Set<Employee>> getHierarchy(Collection<Employee> employees)
\end{lstlisting}

\pagebreak

\begin{lstlisting}
void testGetHierarchie() {
    Superior kriegesmann = new Superior("kriegesmann");
    Superior luis = new Superior("luis");
    Employee urban = new Employee("urban");
    Employee borsum = new Employee("borsum");
    luis.setSuperior(kriegsmann);
    urban.setSuperior(luis);
    borsum.setSuperior(luis);

    HashMap<Superior, ArrayList<Employee>> hierarchie = getHierarchy(List.of(kriegesmann, luis, urban, borsum));
    
    HashMap<Superior, ArrayList<Employee>> expectedHierarchie = new HashMap<>();
    expectedHierarchie.put(kriegesmann, List.of(luis));
    expectedHierarchie.put(luis, List.of(urban, borsum));
    expectedHierarchie.put(urban, new ArrayList<>());
    expectedHierarchie.put(borsum, new ArrayList<>());

    assertEquals(expectedHierarchie, hierarchie);
}
\end{lstlisting}

\section{Aufgabe 4}
In welcher Klasse (oder welchen Klassen) der praktischen Aufgabe Number
Sequence kann sinnvoll die Klasse PushBackSequence eingesetzt werden? Erläutern
Sie, zu welchem Zweck die Klasse dort eingesetzt wird.

MergeSequence und UniqueSequence

\section{Aufgabe 5}
An welchen Stellen innerhalb der Klassen der praktischen Aufgabe Number
Sequence ist es erforderlich oder sinnvoll, eine NoSuchElementException zu
erzeugen und zu werfen?

in Allen endlichen NumberSequences bei der \lstinline{.getNext()} Methode.