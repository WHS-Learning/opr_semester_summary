\chapter{Übung Funktionale Schnittstellen}

\section{Aufgabe 1}
Schreiben Sie ein Code-Stück, um ein Objekt der Klasse
\lstinline{TreeSet<Point>} zu erzeugen, bei dem die Punkte nach der folgenden
Regel geordnet sind. Am Ende des Code-Stücks soll die Menge über die Variable
points zugreifbar sein. Gehen Sie davon aus, dass es eine geeignete Klasse
Point gibt.

Sind p und q zwei Elemente der Menge, dann liegt p vor q, wenn gilt: Die
x-Koordinate von p ist kleiner als die x-Koordinate von q, oder die
x-Koordinaten beider Punkte sind gleich und die y-Koordinate von p ist kleiner
als die y-Koordinate von q

\begin{lstlisting}
TreeSet<Point> points = new TreeSet<>((p, q) -> p.getX() - q.getY() == 0? p.getX() - q.getY() : p.getY() - q.getY());
\end{lstlisting}

\section{Aufgabe 2}
Diese Aufgabe bezieht sich auf das Code-Beispiel „UserManagement“ der
Vorlesung. Definieren Sie die Schnittstelle PasswordValidator derart, dass bei
der Erzeugung eines UserManagement-Objekts ein PasswordValidator per
Lambda-Ausdruck übergeben werden kann. Erzeugen Sie auf diese Weise ein
UserManagement-Objekt, das Passwörter akzeptiert, die mit einer Ziffer beginnen
und mindestens die Länge 12 haben.

\begin{lstlisting}
new UserManager(s -> s.charAt(0) >= '0' && s.charAt(0) <= '9' && p.length() >= 12);
\end{lstlisting}

