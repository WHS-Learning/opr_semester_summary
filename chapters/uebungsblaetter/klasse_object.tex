\chapter{Übung Klasse Object}

\section{Aufgabe 1}
Gegeben sei folgende Klasse Zeitspanne, deren Objekte Zeitspannen bestehend aus
Stunden, Minuten und Sekunden repräsentieren.

\begin{lstlisting}
public class Zeitspanne {
    private final int stunden;
    private final int minuten;
    private final int sekunden;

    /**
    * Erzeugt ein Objekt dieser Klasse. Die Parameter dürfen
    * nicht negativ sind.
    */
    public Zeitspanne(int stunden, int minuten, int sekunden) {
        this.stunden = stunden;
        this.minuten = minuten;
        this.sekunden = sekunden;
    }
}
\end{lstlisting}

\begin{itemize}
    \item Erweitern Sie die Klasse derart, dass Objekte der Klasse mit beliebigen
          Objekten auf Gleichheit getestet werden können. Ein Objekt der Klasse
          Zeitspanne soll nur mit Objekten derselben Klasse gleich sein, und zwar genau
          dann, wenn es sich real um dieselbe zeitliche Spanne handelt. \newline Bsp.: 2
          Stunden, 17 Minuten und 10 Sekunden ist die gleiche Zeitspanne wie 0 Stunden,
          135 Minuten und 130 Sekunden. \newline
          \begin{lstlisting}
@Override
public boolean equals(Object obj) {
    if (obj instanceof Zeitspanne z) {
        return this.stunden * 3600 + this.minuten * 60 + this.sekunden == z.stunden * 3600 + z.minuten * 60 + this.sekunden;
    }
    return false;
}
\end{lstlisting}
    \item Erweitern Sie die Klasse derart, dass durch den Methodenaufruf
          \lstinline{System.out.print(z)} (z sei ein Ausdruck des Typs Zeitspanne) eine
          textuelle Darstellung des übergebenen Objekts im Format h:mm:ss ausgegeben
          wird. Hierbei soll der Minuten- und Sekundenanteil immer zweistellig und
          kleiner als 60 sein. Der Stundenanteil wird mit so vielen Stellen dargestellt,
          wie es erforderlich ist. \newline
          \begin{lstlisting}
@Override
public String toString() {
    int seconds = this.stunden * 3600 + this.minuten * 60 + this.sekunden;
    int hours = seconds / 3600;
    seconds = seconds % hours;
    int minutes = seconds / 60;
    seconds = seconds / minutes;
    return "" + hours + ":" + (minutes > 10 ? "" : "0") + minutes + ":" + (seconds > 10 ? "" : "0") + seconds;
}
\end{lstlisting}
\end{itemize}

\section{Aufgabe 2}
Jedes der folgenden Programmstücke enthält eine equals-Methode (und es kann
noch andere Methoden enthalten). Beantworten Sie für jedes Programmstück
folgende Fragen.

\begin{enumerate}
    \item Ist das Programmstück compilierbar? Falls nein, was ist der Fehler?
    \item Falls es compilierbar ist, gibt es Anwendungen der Methode, die zu einem
          Laufzeitfehler füh- ren?
    \item Falls Anwendungen der Methode zu einem Laufzeitfehler führen, ist dies
          `akzeptabel' oder sollte er durch Änderung der Methode vermieden werden?
    \item Fällt Ihnen sonst noch etwas Bemerkenswertes an dem Programmstück auf?
\end{enumerate}

\begin{lstlisting}
public class A {
  private int b;

  public boolean equals(Object obj) {
    return this.b == ((A) obj).b;
  }
}
\end{lstlisting}
\begin{enumerate}
    \item Ist das Programmstück compilierbar? Falls nein, was ist der Fehler? \newline
          \textbf{Ja}
    \item Falls es compilierbar ist, gibt es Anwendungen der Methode, die zu einem
          Laufzeitfehler führen? \newline \textbf{Ja, wenn das Object keine A-Instnaz
              ist}
    \item Falls Anwendungen der Methode zu einem Laufzeitfehler führen, ist dies
          `akzeptabel' oder sollte er durch Änderung der Methode vermieden
          werden?\newline \textbf{Nein}
    \item Fällt Ihnen sonst noch etwas Bemerkenswertes an dem Programmstück auf?\newline
          \textbf{-}
\end{enumerate}

\begin{lstlisting}
public class B {
  private int b;

  public boolean equals(B obj) {
    return this.b == obj.b;
  }
}
\end{lstlisting}
\begin{enumerate}
    \item Ist das Programmstück compilierbar? Falls nein, was ist der Fehler?\newline
          \textbf{Ja}
    \item Falls es compilierbar ist, gibt es Anwendungen der Methode, die zu einem
          Laufzeitfehler führen? \newline \textbf{wenn obj == null}
    \item Falls Anwendungen der Methode zu einem Laufzeitfehler führen, ist dies
          `akzeptabel' oder sollte er durch Änderung der Methode vermieden
          werden?\newline \textbf{Ja, die Anforderungen an die Methode sind nicht die
              selben wie die Anforderungen an die Equals Methode aus der Object-Klasse}
    \item Fällt Ihnen sonst noch etwas Bemerkenswertes an dem Programmstück auf?\newline
          \textbf{Die Methode überschreibt nicht die equals Methode aus der
              Object-Klasse}
\end{enumerate}
\pagebreak
\begin{lstlisting}
public class C {
  private int b;

  public void equals(Object obj) {
  }
}
\end{lstlisting}
\begin{enumerate}
    \item Ist das Programmstück compilierbar? Falls nein, was ist der Fehler? \newline
          \textbf{Nein, der Return Typ darf nicht geändert werden.}
    \item Falls es compilierbar ist, gibt es Anwendungen der Methode, die zu einem
          Laufzeitfehler führen? \newline\textbf{-}
    \item Falls Anwendungen der Methode zu einem Laufzeitfehler führen, ist dies
          `akzeptabel' oder sollte er durch Änderung der Methode vermieden werden?
          \newline \textbf{-}
    \item Fällt Ihnen sonst noch etwas Bemerkenswertes an dem Programmstück auf? \newline
          \textbf{-}
\end{enumerate}

\begin{lstlisting}
public class D {
  private String s;

  public boolean equals(Object obj) {
    return obj.hashCode() == this.hashCode();
  }
}
\end{lstlisting}
\begin{enumerate}
    \item Ist das Programmstück compilierbar? Falls nein, was ist der Fehler? \newline
          \textbf{Ja}
    \item Falls es compilierbar ist, gibt es Anwendungen der Methode, die zu einem
          Laufzeitfehler führen? \newline\textbf{wenn obj == null}
    \item Falls Anwendungen der Methode zu einem Laufzeitfehler führen, ist dies
          `akzeptabel' oder sollte er durch Änderung der Methode vermieden
          werden?\newline \textbf{Nein, weil die Anforderungen an die Equals 
          Methode der Object-Klasse nicht erfüllt werden}
    \item Fällt Ihnen sonst noch etwas Bemerkenswertes an dem Programmstück auf?\newline
          \textbf{Gleicher HashCode muss nicht gleiches Objekt bedeuten.}
\end{enumerate}