\chapter{Übung Collection Klassen}

\section{Aufgabe 1}
Gegeben sei die Klasse Point:

\begin{lstlisting}
public class Point {
  private int x;
  private int y;

  public Point(int x, int y) {
    this.x = x;
    this.y = y;
  }

  public int getX() {
    return this.x;
  }

  public int getY() {
    return this.y;
  }
}
\end{lstlisting}

Realisieren Sie eine statische Methode \newline 
\lstinline{boolean doesInclude(LinkedList<Point> points, Point p)} um zu prüfen, ob eine
verkettete Liste ein Point-Objekt enthält, das hinsichtlich seiner Koordinaten
mit dem übergebenen Punkt übereinstimmt. Sie dürfen auch den Code der Klasse
Point erweitern.

\begin{lstlisting}
public static boolean doesInclude(LinkedList<Point> points, Point p) {
    return points.contains(p);
}

@Override
public boolean equals(Object obj) {
    if (obj instanceof Point p) {
        this.x == p.x && this.y == p.y;
    }
    return false;
}

@Override 
public int hashCode() {
    return this.x + this.y;
}
\end{lstlisting}

\section{Aufgabe 2}

Die Menge $M \subset \mathbb{N}$ sei wie folgt indukttiv definiert:

\begin{enumerate}
    \item $1 \in M$
    \item falls $m \in M$, dann sind auch $2m + 1 \in M$ und $3m \in M$
\end{enumerate}

Realisieren Sie eine statische Methode \lstinline{ArrayList<Integer> upTo(int n)}, die
aufsteigend sortiert alle Elemente der Menge M bis zum ersten Element liefert,
das größer oder gleich n ist. Beispiel: upTo(20) liefert eine Liste mit den
Elementen 1, 3, 7, 9, 15, 19 und 21.

\begin{lstlisting}
public static ArrayList<Integer> upTo(int n) {
    ArrayList<Integer> list = new ArrayList<>();
    list.add(1);

    int i = 0;

    while (list.get(i) < n) {
        int num = list.get(i);
        list.add(2 * num + 1);
        list.add(3 * num);
        i++;
    }

    Collections.sort(list);

    while (list.size() > 1 && list.get(list.size() - 2) > n) {
        list.removeLast();
    }

    list = new ArrayList<>(new TreeSet<>(list));

    return list;
}
\end{lstlisting}

\section{Aufgabe 3}

Diese Aufgabe bezieht sich auf die praktische Aufgabe `Car Sharing'. Ein
VehicleManager verwaltet viele Fahrzeuge an vielen Standorten, ein Fahrzeug
kann für viele Zeiträume gebucht werden. Überlegen Sie, wie man diese Daten der
Car Sharing-Anwendung am besten repräsentieren kann. Berücksichtigen Sie
insbesondere, welche Zugriffe auf diese Daten aufgrund der Methoden der Klasse
VehicleManager erforderlich sind, und denken Sie an Laufzeiteffizienz.

HashMap.

\section{Aufgabe 4}
Skizzieren Sie die notwendigen Erweiterungen der Klasse Fraction (Praktikum zu
EPR), damit ein Objekt der Klasse \lstinline{HashSet<Fraction>} keine zwei Brüche mit
gleichem Wert enthalten kann.

\begin{lstlisting}
@Override
public boolean equals(Object obj) {
    if (obj instanceof Fraction f) {
        return this.numerator == f.numerator && this.denominator == f.denominator;
    }
    return false;
}

@Override
public int hashCode() {
    return this.numerator + this.denominator;
}
\end{lstlisting}

\section{Aufgabe 5}
Diese Aufgabe können Sie erst bearbeiten, wenn Sie das Kapitel `Hüllklassen'
durchgearbeitet haben. Ein Objekt der Klasse Zeitgeschichte soll Ereignisse der
Zeitgeschichte verwalten. Welche Instanzvariablen wählen Sie (es kommt vor
allem auf die Typen an), sodass eine Instanzmethode
\lstinline{gibEreignisse(int jahr)} möglichst effizient die Ereignisse eines
Jahres liefert? Skizzieren Sie den Quellcode dieser Methode, ebenso wie den
Quellcode des Konstruktors sowie einer Methode 
\lstinline{void fuegeEreignisHinzu(int jahr, String ereignis)}.

Nimmt man ein TreeMap<Integer, String> als Collection, so sind die Ereignisse 
sehr effizient gespeichert und sie sind direkt sortiert.

\begin{lstlisting}
public class Zeitgeschichte {
  TreeMap<Integer, Set<String>> ereignisse;

  public Zeitgeschichte() {
    this.ereignisse = new TreeMap<>();
  }

  public Set<String> gibEreignis(int jahr) {
    return ereignisse.get(jahr);
  }

  public void fuegeEreignisHinzu(int jahr, String ereignis) {
    Set<String> set = ereignisse.getOrDefault(jahr, new HashSet<>());
    set.add(ereignis);
    ereignisse.put(jahr, set);
  }
}
\end{lstlisting}

\section{Aufgabe 6}
Diese Aufgabe bezieht sich auf die praktische Aufgabe `Staff'. Realisieren Sie
in der Klasse Employee eine statische Methode \newline
\lstinline{...getHierarchy(ArrayList<Employee>)}, die aus den übergebenen
Mitarbeitern und allen Vorgesetzten, auf die darüber zugegriffen werden kann,
eine Personalhierarchie erstellt. Eine passende Datenstruktur dafür müssen Sie
selbst finden, weshalb der Ergebnistyp der Methode offen gelassen ist. In der
Personalhierarchie sollen alle Personen enthalten sein, die oben genannt sind,
und jeder Person sollen ihre direkt untergebenen Personen zugeordnet sein. Für
diese Aufgabe ist es zulässig, dass verschiedene Mitarbeiter den gleichen Namen
haben. Anders ausgedrückt: Haben zwei Objekte der Klasse Employee und Superior
verschiedene Identitäten, handelt es sich um verschiedene Personen. Diese
Aufgabe ist ein Beispiel dafür, dass man nicht zwingend hashCode und equals in
einer Klasse überschreiben muss, wenn man Objekte dieser Klasse als Schlüssel
in einer HashMap verwendet. Warum nicht? Weil in Object beide Methoden
implementiert sind - und manchmal ist die Implementierung dort genau die, die
man benötigt.

\begin{lstlisting}
HashMap<Superior, HashSet<Employee>> getHierarchy(ArrayList<Employee> employees) {
    HashMap<Superior, HashSet<Employee>> hierarchie = new HashMap<>();

    for (Employee employee : employees) {
        Employee superior = employee.getSuperior();

        HashSet<Employee> list = hierarchie.getOrDefault(superior, new HashSet<>());
        list.add(employee);

        hierarchie.put(super, list);
    }

    return hierarchie;
}

\end{lstlisting}