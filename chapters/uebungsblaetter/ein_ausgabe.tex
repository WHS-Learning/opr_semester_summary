\chapter{Übung Ein-/Ausgabe}

\section{Aufgabe 1}
Byteorientierte Datenquellen (Oberklasse: \lstinline{InputStream}) und zeichenorientierte Datenquellen (Oberklasse: \lstinline{Reader}) bieten mit ihren beiden \lstinline{read}-Methoden die Möglichkeit, sequentiell auf ihre Inhalte
zuzugreifen. Ein anderer als ein sequentieller Zugriff ist nicht möglich.
Für manche Anwendungen kann es wünschenswert sein, Bytes oder Zeichen in die Datenquelle
zurückschieben zu können, sodass diese beim nächsten read daraus gelesen werden.
Skizzieren Sie die Realisierung einer Klasse \lstinline{PushBackReader} mit einem Konstruktor
\lstinline{PushBackReader(Reader)}. Ein Objekt dieser Klasse ermöglicht den Zugriff auf den übergebenen
Reader, bietet jedoch zusätzlich die Möglichkeit, Zeichen `zurückzuschreiben'. Das zuletzt zurückgeschriebene Zeichen ist dasjenige, was beim nächsten \lstinline{int read()} geliefert wird.

\section{Aufgabe 2}

Schauen Sie sich die Dokumentation der Methode \lstinline{int available()} der Klasse \lstinline{InputStream}
an. Gerne (und fälschlicherweise) wird diese Methode verwendet, um für einen erzeugten Stream
auf die Anzahl der darin enthaltenen Bytes (= Größe des Streams) zuzugreifen. Bereits am Methodenkopf kann man erkennen, dass \lstinline{int available()} diese Aufgabe gar nicht erfüllen kann.
Woran erkennen Sie es?

\section{Aufgabe 3}
In einer Firma wurde in der Klasse \lstinline{QuickAndDirty} eine Methode
\lstinline{long untersucheDateien(Path pfad1, Path pfad2)}
geschrieben. Die Methode arbeitet byteorientiert auf zwei Dateien, deren Pfade durch Objekte des
Typs \lstinline{java.nio.file.Path} übergeben werden.
Leider hat die Firma bei der Entwicklung der Methode nicht daran gedacht, dass diese getestet werden muss (und dass man die Arbeit an den Methoden mit dem Schreiben der Tests hätte beginnen
sollen).
Was raten Sie der Firma? Wie würden Sie die JUnit-Tests für die Methode schreiben? Es geht insbesondere um die Bereitstellung der Testdaten für die Testfälle. Ziehen Sie auch in Betracht, die
Schnittstelle der Methode zu ändern.

\section{Aufgabe 4}
Wir möchten besser verstehen, was die Bedeutung der Codierungen zwischen Zeichen und Bytes
ist.
Skizzieren (oder programmieren) Sie eine Methode
\ lstinline{byte[] inBytes(char, Charset)}
die zu einem Zeichen die Bytefolge liefert, durch die das Zeichen in dem übergebenen Charset
kodiert wird.
Verwenden Sie keine Methoden der Klasse \lstinline{Charset}, sondern denken Sie an Ausgabeströme.

\section{Aufgabe 5}
Skizzieren (oder programmieren) Sie eine Methode
\lstinline{IntStream chars(Reader)}
die den Inhalt eines \lstinline{Readers} als Stream liefert. Die Werte des Streams sind die Codes der Zeichen.
Hinweis: Hilfreich ist die Methode \lstinline{generate} der Klasse \lstinline{IntStream}. Hierdurch wird jedoch ein
unendlicher Stream erzeugt. Schauen Sie sich deshalb auch die Methode \lstinline{takeWhile} an. Nützlich
ist auch dieses Realisierungsmuster:
\begin{lstlisting}
try {
    ...
} catch (IOException e) {
    throw new UncheckedIOException(e);
}
\end{lstlisting}
Das Muster ist nützlich, wenn man eine Ausnahme nicht behandeln kann, jedoch auch nicht deklarieren kann, dass man sie weiterwirft. Dies ist z. B. der Fall, wenn die Checked Exception innerhalb
eines Lambda-Ausdrucks auftritt. In diesem Fall wirft man anstelle der Checked eine Unchecked
Exception.