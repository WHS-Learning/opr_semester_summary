\chapter{Übungsblatt Klassenhierarchie und Polymorphie}

\section{Aufgabe 1}
Bearbeiten Sie diese Aufgabe ohne Zuhilfenahme des Rechners. Betrachten Sie
folgende Klassen \lstinline{A}, \lstinline{B} und \lstinline{C}:

\begin{lstlisting}
public class A {
  public A() {
    System.out.println("A");
  }

  public A(int parameter) {
    System.out.println("A2");
  }
}

public class B extends A {
}

public class C extends B {
  public C() {
    System.out.println("C");
  }

  public C(int parameter) {
    System.out.println("C2");
  }
}
\end{lstlisting}

Beantworten Sie nun folgende Fragen:

\begin{enumerate}
    \item Welche Ausgaben (Reihenfolge beachten!) erfolgen bei Auswertung der Ausdrücke
          \lstinline{new C()} und \lstinline{new C(5)}?\newline \textit{Antwort:} \begin{itemize}
              \item[\lstinline{new C()}] AC
              \item[\lstinline{new C(5)}] AC2
          \end{itemize} \pagebreak
    \item Lassen sich noch alle Klassen \lstinline{A}, \lstinline{B} und \lstinline{C} fehlerfrei compilieren, wenn in \lstinline{B}
          folgender Konstruktor hinzugefügt wird?
          \begin{lstlisting}
public B(int parameter) {
    System.out.println("B");
}
            \end{lstlisting}
          Falls ja, welche Ausgabe erfolgt bei Auswertung von \lstinline{new C(5)}? Falls nein,
          welche Klasse(n) lassen sich aus welchen Gründen nicht compilieren?\newline
          \textit{Antwort:} Die Klasse \lstinline{C} lässt sich nicht mehr compilieren.
\end{enumerate}

\section{Aufgabe 2}
Bearbeiten Sie diese Aufgabe ohne Zuhilfenahme des Rechners. Gegeben seien die
Klassen \lstinline{P}, \lstinline{Q} und \lstinline{R}.

\begin{lstlisting}
public class P {
  public String m1(P par) {
    return "[Pm1" + par.m2(this) + "]";
  }

  public String m2(P par) {
    return "(Pm2" + this.m3() + par.m3() + ")";
  }

  public String m3() {
    return "Pm3";
  }
}

public class Q extends P {
  public String m1(P par) {
    return "[Qm1" + super.m1(par) + "]";
  }

  public String m2(P par) {
    return "(Qm2" + super.m2(par) + ")";
  }

  public String m3() {
    return "Qm3";
  }
}

public class R extends Q {
  public String m3() {
    return "Rm3";
  }
}
\end{lstlisting}

Welchen Wert hat der Ausdruck \lstinline{new R().m1(new Q())}?

[Qm1[Pm1(Qm2(Pm2Qm3Rm3))]]