\chapter{Übung Streams}

\section{Aufgabe 1}
Gegeben sei folgende Methode. Schreiben Sie die Methode unter Verwendung von
Streams derart um, dass darin weder eine explizite Iteration, noch eine
if-Anweisung vorkommt

\begin{lstlisting}
public static List<String> m1(Collection<String> woerter) {
  List<String> ergebnis = new ArrayList<>();
  for (String wort : woerter) {
    if (wort.length() >= 5) {
      String s = wort.substring(wort.length() / 2);
      if (s.charAt(0) >= 'a') {
        ergebnis.add(s);
      }
    }
  }
  return ergebnis;
}
\end{lstlisting}

\begin{lstlisting}
public static List<String> m1(Collection<String> woerter) {
  return woerter.stream()
      .filter(s -> s.length() >= 5)
      .map(s -> s.substring(s.length() / 2))
      .filter(s -> s.charAt(0) >= 'a')
      .toList();
}
\end{lstlisting}

\section{Aufgabe 2}
Schauen Sie sich die API-Dokumentation der Klasse \lstinline{OptionalInt},
insbesondere der Methode \lstinline{ifPresent} an. Ändern Sie dann das folgende
Code-Stück derart, dass es bei gleichem Verhalten keine if-Anweisung mehr
enthält. Die Variable is ist vom Typ IntStream.

\begin{lstlisting}
OptionalInt zahlOpt = is
        .filter(n -> n % 2 == 0)
        .findAny();
if(zahlOpt.isPresent()){
  int zahl = zahlOpt.getAsInt();
  System.out.println(zahl *zahl);
}
\end{lstlisting}

\begin{lstlisting}
is.filter(n -> n % 2 == 0)
  .findAny()
  .ifPresent(i -> System.out.println(i * i));
\end{lstlisting}

\section{Aufgabe 3}
Schauen Sie sich die API-Dokumentation der Schnittstelle
\lstinline{LongStream}, insbesondere der Methode \lstinline{takeWhile} an.
Geben Sie unter Verwendung der Methode \lstinline{LongStream fibonaccis()} aus
dem Code-Beispiel der Vorlesung einen Ausdruck an, der den Stream aller
Fibonacci-Zahlen liefert, die mathematisch korrekt berechnet werden (d. h. bei
denen es zu keinen Überlauf kommt).

\begin{lstlisting}
fibonaccis().takeWhile(i -> i >= 0);
\end{lstlisting}

\section{Aufgabe 4}
Es sei s eine Variable des Typs \lstinline{Set<Integer>}.
\begin{enumerate}
  \item Geben Sie einen Ausdruck an, der die Summe aller Elemente der Menge s
        berechnet, die geradzahlig sind.
        \begin{lstlisting}
s.filter(s -> s % 2 == 0)
  .mapToInt(n -> n)
  .sum();
\end{lstlisting}
  \item Geben Sie einen Ausdruck des Typs \lstinline{IntStream} an, der alle
        natürlichen Zahlen enthält bis einschließlich 100, die nicht in der Menge s
        enthalten sind.
                \begin{lstlisting}
IntStream().rangeClosed(1, 100)
            .filter(i -> !s.contains(i));
\end{lstlisting}
\end{enumerate}

\section{Aufgabe 5}
Geben Sie einen Ausdruck des Typs \lstinline{IntStream} an, der folgende
Elemente enthält: 0, 1, 3, 7, 15, 31, \dots

\begin{lstlisting}
IntStream().iterate(0, i -> i + i + 1)
    .takeWhile(i -> i >= 0);
\end{lstlisting}

\section{Aufgabe 6}
Realisieren Sie eine Methode \lstinline{boolean istPalindrom(String s)} ohne
Verwendung von Schleifen, if-Anweisungen und Rekursion.

\begin{lstlisting}
public boolean istPalindrom(String s) {
  int[] chars = new int[s.length()];
  int[] index = new int[]{s.length() - 1};
  s.chars().forEach(i -> chars[index[0]--] = i);
  return Arrays.equals(chars, s.chars().toArray());
}
\end{lstlisting}

\textbf{Alternative, vernünftige Lösung}

\begin{lstlisting}
public static boolean istPalindrom(String s) {
  return IntStream.range(0, s.length() / 2)
          .allMatch(i -> s.charAt(i) == s.charAt(s.length() - 1 - i));
}
\end{lstlisting}

\section{Aufgabe 7}
Gegeben sei das folgende Code-Stück. Die Variable zeilen ist vom Typ
\lstinline{Stream<String>}.

\begin{lstlisting}
zeilen
    .map(w -> new StringTokenizer(w))
    .filter(st -> st.hasMoreTokens())
    .map(st -> st.nextToken())
    .filter(s -> s.length() >= 5)
    .findFirst();
\end{lstlisting}

\begin{enumerate}
  \item Was berechnet der Ausdruck? \newline
  \textit{Der Code gibt den ersten Token des Strings, der mehr als 4 buchstaben lang ist.}
  \item Jemand behauptet, das Code-Stück sei völlig ineffizient und würde, wenn zeilen
        einen unendlichen Stream enthält, überhaupt nicht terminieren. Er begründet
        dies damit, dass durch die erste Anwendung von map für alle Elemente des
        Streams zunächst ein StringTokenizer erzeugt wird. Ändern Sie das Code-Stück
        derart, dass durch Bildschirmausgaben dokumentiert wird, dass die Person
        Unrecht hat und nur soviele StringTokenizer erzeugt werden, wie für die
        Funktion des Code-Stücks benötigt werden.
\begin{lstlisting}
zeilen
    .map(w -> {
      System.out.println(w);
      return new StringTokenizer(w);
    })
    .filter(StringTokenizer::hasMoreTokens)
    .map(StringTokenizer::nextToken)
    .filter(s -> s.length() >= 5)
    .findFirst();
\end{lstlisting}
\end{enumerate}

