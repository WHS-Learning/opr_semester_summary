\chapter{Aufzählungstypen}\label{chap:aufzaehlungstypen}

Sollen Konstanten definiert werden, ist dies über \lstinline{static final} Variablen möglich. Soll beispielsweise einer von vier Zuständen gespeichert werden, so bietet sich ein Aufzählungstyp mehr an. 

\begin{lstlisting}[language=Java, caption={Beispiel für Aufzählungstypen}]
public enumm Direction {
    UP, DOWN, LEFT, RIGHT
}
\end{lstlisting}

Durch diesen Aufzählungstypen kann nun die Richtung von etwas sehr einfach gespeichert werden.

\begin{lstlisting}[language=Java, caption={Beispiel für Aufzählungstypen}]
private Direction direction = Direction.LEFT;

private void turnLeft() {
    if (direction == Direction.LEFT) {
        direction = Direction.DOWN;
    }
    else if (direction == Direction.DOWN) {
        direction = Direction.RIGHT;
    }
    else if (direction == Direction.RIGHT) {
        direction = Direction.UP;
    }
    else if (direction == Direction.UP) {
        direction = Direction.LEFT;
    }
}
\end{lstlisting}

Aufzählungstypen können, wie primitive Datentypen, über die \lstinline{==} Operation auf gleichheit geprüft werden. Dies liegt daran, dass keine neuen Objekte erstellt werden, sondern jedes \lstinline{Direction.UP} die gleiche Identität besitzt.

\section{Aufzählungstypen mit Konstruktor}

Sollen in dem Aufzählungstypen noch einige weitere werte gespeichert werden, so kann ihm ein Konstruktor zugewiesen werden.

\begin{lstlisting}[language=Java, caption={Beispiel für Aufzählungstypen}]
public enumm Module {
    OPR(false),
    FPR(true)

    private boolean isMasterModule;

    public Module(isMasterModule boolean) {
        this.isMasterModule = isMasterModule;
    }

    public boolean isMasterModule() {
        return this.isMasterModule;
    }
}
\end{lstlisting}