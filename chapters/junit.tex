\chapter{JUnit Tests (Sehr Prüfungsrelevant)}

JUnit Tests sind ein fundamentaler Bestandteil der modernen Softwareentwicklung und ein häufiges Thema in Prüfungen. Dieses Kapitel führt in die Grundlagen des Testens mit JUnit ein.

\section{Hintergrund von Tests}
Software zu testen ist unerlässlich, um deren korrekte Funktionalität sicherzustellen. Eine bewährte Praxis ist das Test-Driven Development (TDD), bei dem Tests vor der eigentlichen Implementierung der Anwendungslogik geschrieben werden. Tests definieren die Anforderungen an das Programm und dienen als ausführbare Spezifikation. Testfälle repräsentieren dabei konkrete Anwendungsszenarien.

\section{Tests schreiben mit JUnit}
Testklassen in JUnit dienen dazu, die Methoden einer Anwendungsklasse systematisch zu überprüfen. Idealerweise existiert für jede Anwendungsklasse eine korrespondierende Testklasse. Diese Testklassen bündeln Testmethoden, die einzelne Aspekte oder Testfälle der zu testenden Methoden abdecken. Es ist üblich und empfehlenswert, mehrere Testmethoden für eine einzelne Anwendungsmethode zu erstellen, um verschiedene Szenarien (z.B. Normalfall, Grenzfälle, Fehlerfälle) abzudecken. Testmethoden sollten voneinander unabhängig sein und isoliert ausgeführt werden können.

\subsection{Importe für JUnit}
Für die Verwendung von JUnit müssen spezifische Annotationen und Assertions-Methoden importiert werden. JUnit ist ein weit verbreitetes Framework für das Testen von Java-Code.

\begin{lstlisting}[language=Java, caption={Grundlegende JUnit 5 Importe}, label=lst:junit_imports, basicstyle=\ttfamily\footnotesize, breaklines=true, frame=tb, numbers=left]
// Annotationen für Test-Setup und Testmethoden
import org.junit.jupiter.api.BeforeEach; // Wird vor jeder Testmethode ausgeführt
import org.junit.jupiter.api.AfterEach;  // Wird nach jeder Testmethode ausgeführt (seltener benötigt)
import org.junit.jupiter.api.Test;       // Markiert eine Methode als Testmethode
import org.junit.jupiter.api.Disabled;   // Deaktiviert eine Testmethode oder -klasse

// Statische Importe für Assertions-Methoden (empfohlen für bessere Lesbarkeit)
import static org.junit.jupiter.api.Assertions.assertEquals;
import static org.junit.jupiter.api.Assertions.assertTrue;
import static org.junit.jupiter.api.Assertions.assertFalse;
import static org.junit.jupiter.api.Assertions.assertNull;
import static org.junit.jupiter.api.Assertions.assertNotNull;
import static org.junit.jupiter.api.Assertions.assertThrows;
// Es gibt viele weitere Assertions-Methoden in org.junit.jupiter.api.Assertions
\end{lstlisting}
Hinweis: Die Verwendung von \textit{AfterEach} ist seltener notwendig als \textit{BeforeEach}, da gut designte Tests oft keine explizite Aufräumarbeit nach jedem Test benötigen (z.B. wenn keine externen Ressourcen geöffnet werden).

\subsection{Struktur einer Testklasse}
\label{ssec:junit_klassenkopf}
Gemäß der Namenskonvention sollte eine Testklasse für eine Klasse `MyClass` den Namen `MyClassTest` tragen. Instanzvariablen in Testklassen sind üblich, um Testobjekte oder Testdaten zu halten, die von mehreren Testmethoden verwendet werden. Um die Unabhängigkeit der Tests zu gewährleisten, sollten diese Instanzvariablen typischerweise in einer mit `@BeforeEach` annotierten Methode `setUp()` vor jeder Testmethode neu initialisiert werden.

\begin{lstlisting}[language=Java, caption={Grundgerüst einer JUnit Testklasse}, label=lst:junit_test_class_structure, basicstyle=\ttfamily\footnotesize, breaklines=true, frame=tb, numbers=left]
// Angenommen, wir testen eine Klasse namens "Calculator"
class CalculatorTest {

    private Calculator calculatorInstance; // Instanzvariable für das Testobjekt

    @BeforeEach
    void setUp() {
        // Diese Methode wird vor jeder @Test Methode ausgeführt.
        // Ideal für die Initialisierung von Testobjekten.
        calculatorInstance = new Calculator();
        System.out.println("setUp() executed: New Calculator instance created.");
    }

    @AfterEach
    void tearDown() {
        // Diese Methode wird nach jeder @Test Methode ausgeführt.
        // Nützlich für Aufräumarbeiten, z.B. Schließen von Ressourcen.
        // In vielen Fällen nicht zwingend notwendig.
        calculatorInstance = null; // Beispiel: Objekt freigeben
        System.out.println("tearDown() executed.");
    }

    @Test
    void testAddition() {
        // Testlogik für die Additionsmethode
        int result = calculatorInstance.add(5, 3);
        assertEquals(8, result, "5 + 3 should be 8"); // Assertion mit optionaler Nachricht
    }

    @Test
    void testSubtraction() {
        // Testlogik für die Subtraktionsmethode
        int result = calculatorInstance.subtract(10, 4);
        assertEquals(6, result, "10 - 4 should be 6");
    }
    
    @Test
    @Disabled("This test is currently disabled due to ongoing refactoring.")
    void testMultiplication() {
        // Ein deaktivierter Test
        assertEquals(12, calculatorInstance.multiply(3,4));
    }
}

// Dummy-Klasse, die getestet wird (normalerweise in einer separaten Datei)
class Calculator {
    public int add(int a, int b) { return a + b; }
    public int subtract(int a, int b) { return a - b; }
    public int multiply(int a, int b) { return a * b; }
}
\end{lstlisting}

\subsection{Testdaten initialisieren mit \texttt{@BeforeEach}}
\label{ssec:junit_beforeeach}
Um sicherzustellen, dass jede Testmethode mit einem sauberen und definierten Zustand startet, werden Testdaten und -objekte häufig in einer Methode initialisiert, die mit `@BeforeEach` annotiert ist. Die Namenskonvention für diese Methode ist `setUp()`.

\begin{lstlisting}[language=Java, caption={Verwendung von \texttt{@BeforeEach} zur Initialisierung}, label=lst:junit_beforeeach_example, basicstyle=\ttfamily\footnotesize, breaklines=true, frame=tb, numbers=left]
class Book { // Beispielklasse, die getestet wird
    private String title;
    private String author;
    // Weitere Attribute und Konstruktor...
    public Book(String title, String author) { this.title = title; this.author = author; }
    public String getAsText() { return title + "; " + author; }
}

class BookTest {
    private Book testBook; // Instanzvariable für das Testobjekt

    @BeforeEach
    void setUp() {
        // Initialisiert das testBook Objekt vor jedem Test neu
        testBook = new Book("The Hitchhiker's Guide to the Galaxy", "Douglas Adams");
    }

    @Test
    void testBookCreation() {
        assertNotNull(testBook, "Book should be initialized by setUp()");
    }

    @Test
    void testGetAsText() {
        String expectedText = "The Hitchhiker's Guide to the Galaxy; Douglas Adams";
        assertEquals(expectedText, testBook.getAsText(), "getAsText() should return correct format.");
    }
}
\end{lstlisting}

\subsection{Testdaten abbauen mit \texttt{@AfterEach} (seltener)}
\label{ssec:junit_aftereach}
Falls nach der Ausführung jeder Testmethode Aufräumarbeiten notwendig sind (z.B. das Schließen von Dateien, Netzwerkverbindungen oder das Zurücksetzen von globalen Zuständen), kann dies in einer mit `@AfterEach` annotierten Methode geschehen. Die Namenskonvention hierfür ist `tearDown()`. Für die meisten Unit-Tests, die mit einfachen Objekten arbeiten, ist dies nicht erforderlich.

\begin{lstlisting}[language=Java, caption={Verwendung von \texttt{@AfterEach} für Aufräumarbeiten}, label=lst:junit_aftereach_example, basicstyle=\ttfamily\footnotesize, breaklines=true, frame=tb, numbers=left]
// @AfterEach // Selten benötigt für einfache Unit-Tests
// void tearDown() {
//     // Diese Methode wird nach jedem Testfall aufgerufen.
//     // Beispiel: Schließen einer Datei, Freigeben von Ressourcen.
//     // System.out.println("tearDown() called.");
// }
\end{lstlisting}

\subsection{Testmethoden und Assertions}
\label{ssec:junit_testmethoden}
Testmethoden werden mit `@Test` annotiert. Die Namenskonvention für Testmethoden ist oft `test<MethodenNameDieGetestetWird>[_Szenario]`, z.B. `testAdd_withPositiveNumbers` oder `testAdd_withNegativeNumbers`. Testmethoden haben in der Regel keine Rückgabewerte (`void`) und keine Modifikatoren (package-private Sichtbarkeit ist üblich). Sie sollten prägnant sein und sich auf die Überprüfung eines spezifischen Aspekts konzentrieren.

Innerhalb einer Testmethode werden Assertions verwendet, um das tatsächliche Verhalten des Codes mit dem erwarteten Verhalten zu vergleichen.
\begin{lstlisting}[language=Java, caption={Beispiel einer Testmethode mit Assertions}, label=lst:junit_testmethod_example, basicstyle=\ttfamily\footnotesize, breaklines=true, frame=tb, numbers=left]
// Innerhalb einer Testklasse, z.B. BookTest (siehe oben)

// @BeforeEach
// void setUp() {
//     testBook = new Book("Asterix der Gallier", "Uderzo", 1965, 9.8);
// }

// Angenommen, die Book-Klasse hat eine Methode getFormattedString():
// public String getFormattedString() {
//     return String.format("%s; %s; %d; %.1f", title, author, year, price);
// }
// Und der Konstruktor ist: public Book(String title, String author, int year, double price)

// @Test
// void testGetFormattedString_validBook() {
//     // Annahme: testBook wurde in setUp() initialisiert
//     // testBook = new Book("Asterix der Gallier", "Uderzo", 1965, 9.8); // Falls kein setUp
//     String expectedOutput = "Asterix der Gallier; Uderzo; 1965; 9.8";
//     // assertEquals(expectedOutput, testBook.getFormattedString());
// }
\end{lstlisting}
In diesem Beispiel würde `assertEquals` prüfen, ob der von `book.getFormattedString()` zurückgegebene String dem `expectedOutput` entspricht. Die `equals`-Methode des Vergleichsobjekts wird hierfür herangezogen (für Strings ist dies ein Inhaltsvergleich).

Einige gängige Assertions sind:
\begin{itemize}
    \item \texttt{assertEquals(expected, actual, [message])}: Überprüft, ob zwei Werte gleich sind.
    \item \texttt{assertTrue(condition, [message])}: Überprüft, ob eine Bedingung wahr ist.
    \item \texttt{assertFalse(condition, [message])}: Überprüft, ob eine Bedingung falsch ist.
    \item \texttt{assertNull(object, [message])}: Überprüft, ob ein Objekt \texttt{null} ist.
    \item \texttt{assertNotNull(object, [message])}: Überprüft, ob ein Objekt nicht \texttt{null} ist.
    \item \texttt{assertArrayEquals(expectedArray, actualArray, [message])}: Vergleicht zwei Arrays auf Gleichheit (Element für Element).
    \item \texttt{assertThrows(expectedThrowable, executable, [message])}: Überprüft, ob die Ausführung des `executable` Codes eine Exception vom Typ `expectedThrowable` wirft.
\end{itemize}
Die optionale `message` wird angezeigt, wenn die Assertion fehlschlägt, was die Fehlersuche erleichtert.

\subsection{Mock-Klassen zum Testen abstrakter Klassen oder Abhängigkeiten}
\label{ssec:junit_mock_klassen}
Um abstrakte Klassen zu testen oder Klassen, die komplexe Abhängigkeiten haben, werden oft Mock-Klassen (oder Mock-Objekte) verwendet. Eine Mock-Klasse ist eine vereinfachte Implementierung einer Abhängigkeit, die speziell für Testzwecke erstellt wird. Sie simuliert das Verhalten der echten Abhängigkeit in einer kontrollierten Weise.
Die Namenskonvention für eine Mock-Klasse für `AbstractDataProvider` könnte `MockAbstractDataProvider` oder `TestDataProviderImpl` sein.

\begin{lstlisting}[language=Java, caption={Konzept einer Mock-Klasse (vereinfacht)}, label=lst:junit_mock_concept, basicstyle=\ttfamily\footnotesize, breaklines=true, frame=tb, numbers=left]
// Abstrakte Klasse, die getestet werden soll (indirekt)
abstract class AbstractDataProcessor {
    protected abstract String fetchData(); // Abhängigkeit, die gemockt werden könnte

    public String processData() {
        String data = fetchData();
        return "Processed: " + data.toUpperCase();
    }
}

// Eine konkrete Implementierung der abstrakten Klasse für Testzwecke (eine Art Mock)
class TestableDataProcessor extends AbstractDataProcessor {
    private String dataToReturn;

    public TestableDataProcessor(String dataToReturn) {
        this.dataToReturn = dataToReturn;
    }

    @Override
    protected String fetchData() {
        // Simuliert das Verhalten der Abhängigkeit
        return this.dataToReturn;
    }
}

// Testklasse
class AbstractDataProcessorTest {
    @Test
    void testProcessData_withMockedData() {
        // Erstellt eine Instanz der Test-Implementierung mit kontrollierten Daten
        AbstractDataProcessor processor = new TestableDataProcessor("sample data");
        String result = processor.processData();
        assertEquals("Processed: SAMPLE DATA", result);
    }
}
\end{lstlisting}
Für komplexere Mocking-Szenarien werden oft Mocking-Frameworks wie Mockito oder EasyMock eingesetzt, die das Erstellen und Konfigurieren von Mock-Objekten erheblich vereinfachen. Diese sind jedoch über den Rahmen dieser Einführung hinausgehend.

